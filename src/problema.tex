\label{ch:problema}
\chapter{Descripción del Problema}
En Android, por defecto, el desarrollador no cuenta con mecanismos para definir
políticas de confidencialidad e integridad que regulen el flujo de información
de sus aplicaciones. Siendo complejo prevenir fugas de información del usuario,
puesto que, el desarrollador carece de herramientas que le garanticen la
ausencia de flujos indeseados. \newline
Precisamente, una de las principales preocupaciones de seguridad en aplicativos
Android, es la manipulación de información del usuario, puesto que, aún cuando
se trata de aplicaciones benignas, estas pueden presentar flujos de información
que comprometen la privacidad del usuario. \newline
Así lo evidencia un estudio reciente de seguridad en dispositivos móviles,
publicado por McAfee\cite{McAfeeReport}, este señala que entre las aplicaciones
que invaden la privacidad del usuario, existen aplicaciones no consideradas
malware que comprometen información privada o sensible del usuario,
puesto que, permiten que información como: su ubicación,
acciones en el dispositivo y sus contactos telefónicos, sea enviada hacia
servidores de compañías publicitarias\cite{McAfeeReport}(Pag 10).\newline
y lo siguiente sería discutir como poder darle garantías al usuario \newline
por ej. el usuario podría verificar el código fuente o parte de el usando la
herramienta propuesta, en aplicaciones open source.
%  mediante el uso
% de librerías ad.
% Las librerías ad son librerías de terceros que muestran información
% publicitaria, estas librerias sulen ser incluidas por el desarrollador para
% recibir bonificaciones por anuncio publicitario. Tales librerías acceden a
% información privada del usuario como
% 
% 
% De este modo, 80\% reúnen información de la ubicación, 82\%
% hacen seguimiento de alguna acción en el dispositivo, 57\%
% registran la forma de uso del celular(mediante Wi-Fi o
% mediante la red de telefonía), y 36\% conocen información de
% las cuentas de usuario.\newline
% XXXXXXXXXXXXXXXXXXXXXXXXXXXX\newline
% Así lo evidencia un
% estudio reciente de seguridad en dispositivos móviles, publicado por
% McAfee\cite{McAfeeReport}, este señala  que una importante cantidad de
% aplicaciones Android invaden la privacidad del usuario, reuniendo información
% detallada de su desplazamiento, acciones en el dispositivo, y su vida personal.
% De este modo, 80\% reúnen información de la ubicación, 82\%
% hacen seguimiento de alguna acción en el dispositivo, 57\%
% registran la forma de uso del celular(mediante Wi-Fi o
% mediante la red de telefonía), y 36\% conocen información de
% las cuentas de usuario.\newline
Si bien, el acceso a información privada del usuario, no necesariamente
implica acciones delictivas, el cuestionamiento de
fondo es la forma y finalidad con que una aplicación manipula dicha información,
y qué garantías puede ofrecer el desarrollador de que esa manipulación es
adecuada.

La falta de control sobre los flujos de información de la aplicación puede
ocasionar fugas de información, generando problemas de seguridad tanto para
quien la implementa como para quien la usa.\newline
Como contramedida a este problema, la API de Android ofrece herramientas de
seguridad basadas en políticas de control de acceso, y el desarrollador puede
implementarlas en su aplicación. Sin embargo, estos mecanismos se centran en
regular el acceso de los usuarios del sistema a determinados recursos, y no en
verificar qué sucede con la información una vez es accedida. 

Para superar dicha carencia, diferentes trabajos de investigación han abordado
el problema de fuga de información en aplicaciones Android, tanto desde un enfoque
dinámico como desde un enfoque estático, la literatura existente al
respecto(TaintDroid\cite{TaintDroid}, Flow-Droid\cite{FlowDroid-Thesis},
DidFail\cite{DidFail}, DroidForce\cite{DroidForce}), indica que la mayoría de
propuestas hacen data-flow analysis mediante técnicas de análisis 
tainting, partiendo del bytecode
Enfocándose en la precisión y eficiencia del análisis para detectar fugas de
datos en aplicaciones de terceros ya implementadas. Por consiguiente,
la finalidad del análisis no es garantizar el cumplimiento de políticas de
confidencialidad e integridad desde la construcción del aplicativo.

Partir de tales propuestas para analizar aplicaciones propias y garantizar
políticas de confidencialidad e integridad desde su construcción, puede implicar
incompletitud en el análisis(under-tainting) y no detección de flujos
implícitos. Esto debido a que,
% aún cuando el
% desarrollador conoce la funcionalidad de su propio código, las optimizaciones
% realizadas por el compilador pueden adicionar complejidad al
% mismo\cite[pag.~43]{SecureProgramming}; 
por un lado, al realizar análisis tainting de forma dinámica, el marcado de
datos se propaga únicamente a través de caminos del programa actualmente
ejecutados. Así, si existen datos que son influenciados por los datos marcados,
pero no están dentro de los actuales caminos de ejecución, quedan sin la
propagación de la marca, dando lugar al problema de
undertainting\cite{dynamic-tainting}\cite{Bit-Level-Taint-Analysis}. Es
decir, se obtiene precisión en el análisis, pero se pierde completitud.\newline
Por el otro, aún cuando se hace análisis tainting de forma estática, y el
marcado de datos puede ser propagado para todos los caminos posibles de
ejecución  del programa, superando el inconveniente de under-tainting, la
detección de flujos implícitos es posible si, en la construcción de la
herramienta de análisis se propaga el marcado de datos para flujos
implícitos\cite{taint-analysis}. 
Sin embargo, las propuestas que basan su análisis en data-flow estático,
suelen restringir la propagación del marcado de datos a flujos explícitos,
ganando eficiencia en el análisis. Las propuestas mencionadas anteriormente, no
son ajenas a tal generalidad(DidFail\cite{DidFail}[page 33],
FlowDroid\cite{FlowDroid-Thesis}[page 30]).

Ahora bien, la falta de garantías en el cumplimiento de determinadas
políticas de seguridad en la aplicación que se implementa, puede superarse
usando control de flujo de información, Information Flow Control(IFC), puesto
que, con esta técnica la aplicación  es analizada estáticamente  para
identificar todos los posibles caminos que podrían tomar sus flujos de
información, garantizando que a tiempo de ejecución, la aplicación respeta
determinadas políticas de seguridad.

Finalmente, partiendo del contexto que se plantea, donde es el propio
desarrollador Android quien requiere evaluar políticas de seguridad en su
aplicación, para  garantizarle al usuario que la aplicación las cumple. Resulta
apropiado proveerle una herramienta de apoyo, mediante la cual analice el flujo
de información de la aplicación que implementa, y verifique el cumplimiento
de políticas de seguridad.\newline

\section{Técnicas de análisis de código}
\label{sec:contexto}
%-(Lina: falta actualizar contenido)\newline
\subsection{Análisis estático y dinámico}

Las soluciones propuestas para detectar fuga de información en aplicaciones
Android, se enmarcan en el análisis estático o dinámico de la aplicación, en
algunos casos, se combinan ambos tipos.\newline 
En \textbf{análisis estático}\cite{Static-dynamic}, se estudia el código del
programa para inferir todos los posibles caminos de ejecución. Esto se logra
construyendo modelos de estado del programa, y determinando los estados posibles
a alcanzar por el programa.
No obstante, debido a que existen múltiples posibilidades de ejecución, se opta
por construir un modelo abstracto de los estados del programa. La consecuencia
de tener un modelo aproximado es pérdida de información y posibilidad de menor
precisión en el análisis.\newline 
Por otro lado, en \textbf{análisis dinámico} se ejecuta el programa y se analiza
su comportamiento, verificando el camino de ejecución que ha tomado el programa.
Esa exactitud en la ejecución que se verifica da precisión al análisis, porque
no es necesario construir un modelo aproximado de todos los posibles caminos de
ejecución.

Aunque los resultados del análisis estático pueden perder precisión, la ventaja
es que son generalizables, porque el modelo construido representa una
descripción del comportamiento del programa, independientemente de las entradas
y el contexto en que este se ejecute. Ahora, con el análisis dinámico, no es
posible generalizar sus resultados para futuras ejecuciones, porque no
existen garantías de que las entradas con que fue ejecutado el programa,
contengan características para todos los posibles caminos de ejecución.\newline 
Además de las ventajas y desventajas de ambas clases de análisis, cada uno
implica su propio reto. Mientras en el análisis estático la dificultad está
en construir el modelo de abstracción adecuado, en el análisis dinámico, es
complejo encontrar un conjunto de casos de prueba representativo.\newline
%, a analizar durante la ejecución del programa.\newline
Por otra parte, dependiendo de la finalidad con que se detecte la fuga de
información, un tipo de análisis puede ser más apropiado que otro. Si se busca
contener la fuga de información a tiempo de ejecución, análisis dinámico es el
camino apropiado. 
De lo contrario, si se busca garantizar que a tiempo de
ejecución la aplicación no incurre en fugas de información, resulta más
conveniente aplicar análisis estático, porque cumplir con tales garantías
implica definir políticas de confidencialidad y/o integridad desde la
implementación de la aplicación. REFERNCIA

Precisamente, el propósito fundamental del presente trabajo es ofrecer al
desarrollador de aplicaciones Android una herramienta para aplicar políticas de
confidencialidad en la aplicación que implementa, así, la aplicación se
ejecutará exitosamente, si y sólo si, cumple con las políticas definidas, de lo
contrario, el desarrollador puede revisar y corregir su código.\newline
 
\subsection{Detectar o garantizar políticas de seguridad } 
Generalmente, para verificar el cumplimiento de políticas de seguridad mediante
análisis estático, se aplican técnicas de seguridad de tipado
(Typed-Inference/Security-Typed Analysis) y técnicas de flujo de
datos(Data/Control Flow Analysis)\cite{Information-Flow-Java}.\newline 
Con \textbf{técnicas Security-Typed} las propiedades de confidencialidad e
integridad son anotadas en el código, y verificadas a tiempo de compilación,
garantizando su cumplimiento a tiempo de ejecución.\newline 
Con \textbf{técnicas de flujo de control} y \textbf{técnicas de flujo de datos},
las políticas de seguridad son verificadas haciendo seguimiento al control de
flujo, o al flujo de datos, respectivamente. Estás técnicas suelen utilizar
grafos de Control de Flujo CFG(Control Flow Graph), Grafos de Flujo de Datos
DFG( Data Flow Graph) y Grafos de llamadas CG (Call Graphs).

Acorde a literatura científica en el ámbito de seguridad de aplicativos
Android, parte importante de las propuestas para análisis de fuga de
información(TaintDroid\cite{TaintDroid}, Flow-Droid\cite{FlowDroid-Thesis},
DidFail\cite{DidFail}, DroidForce\cite{DroidForce}), parten del bytecode para
realizar data-flow analysis, mediante técnicas de análisis tainting. Las
técnicas de análisis tainting, son un tipo especial de análisis de flujo de
datos, donde se hace seguimiento al flujo de datos entre un conjunto de fuentes
consideradas privadas y/o sensibles; y un conjunto de destinos considerados no
confiables, sources y sinks, respectivamente.\newline 
Si bien, tales propuestas permiten detectar flujos de datos indebidos en
aplicaciones Android, están enfocadas a analizar aplicaciones ya implementadas,
y no, en garantizar el cumplimiento de determinadas políticas de seguridad desde
su construcción.
% Tales propuestas se enfocan en analizar aplicativos de terceros para detectar
% flujos de datos indebidos, y no: para garantizar el cumplimiento de determinadas
% políticas de seguridad. 
% En consecuencia, es complejo que el desarrollador
% garantice la ausencia de fugas de información en la aplicación que implementa,
% partiendo de tales herramientas. 
% 
% Puesto que, al seguir únicamente a los datos
% marcados, los datos no marcados para el análisis, pueden acarrear fugas de
% información(under-tainting). Adicionalmente, si no se hace seguimiento al flujo
% de control pueden existir fugas de información a través de flujos implícitos,
% ya que, el análisis estará centrado en flujos explícitos.\newline
% No obstante, las limitaciones propias de un análisis basado en flujo de datos,
% pueden superarse enfocando el análisis de la aplicación hacia técnicas de
% análisis basadas en control de flujo de información, ya que estas analizan el
% aplicativo de forma estática para identificar todos los posibles caminos que
% podría tomar la aplicación en tiempo de ejecución. 

Precisamente, mediante análisis basado en control de flujo de información y
técnicas Security-Typed, es posible garantizar el cumplimiento de políticas de
seguridad en las aplicaciones que se implementa, puesto que,
% Así, con análisis basado en
% control de flujo de información, no sólo es posible prevenir fugas por
% under-tainting y flujos implícitos; sino que también, es posible ofrecer
% garantías del cumplimiento de determinadas políticas de seguridad.\newline 
% Así, con análisis basado en
% control de flujo de información es posible garantizar el cumplimiento
% de determinadas políticas de seguridad, desde la construcción del
% aplicativo.\newline 
% Ahora, 
las reglas para evaluar control de flujo de
información pueden definirse mediante técnicas Security-Typed, por ejemplo como
se definen con Jif \ref{JIF-Tool}, un lenguaje tipado de seguridad para realizar
control de flujo de información en aplicativos Java.
% 
% el inconveniente es que está implementada para
% aplicaciones en Java, y no para aplicativos Android.\newline 
\subsection{Security Typed Languages}
En general, herramientas basadas en técnicas de análisis
Security-Typed, involucran conceptos como flujo de información, políticas de
confidencialidad e integridad, y chequeo de tipos.

\emph{Flujo de información}: el flujo de información describe el
comportamiento de un programa, desde la entrada de los datos hasta la salida de
los mismos. 

\emph{Políticas de confidencialidad e integridad}: confidencialidad e integridad
son políticas de seguridad aplicables mediante control de flujo de información.
Mientras la confidencialidad busca prevenir que la información fluya hacia
destinos no apropiados, la integridad busca prevenir que la información provenga
de fuentes no apropiadas\cite{LanguageIFS-2013}. Una importante diferencia
entre confidencialidad e integridad, es que la integridad de la información de un programa puede ser
alterada sin la interacción con agentes externos.\newline %\textcolor{red}{(por qué es importante?)}
Ambas políticas son fundamentales para garantizar propiedades de
seguridad.\newline 
Con políticas de confidencialidad, es posible garantizar ausencia de fugas de
información. Con políticas de integridad, la finalidad es evitar
modificación de la información, de forma no consentida.\newline 
Verificar que un programa utilice la información acorde a
tales políticas, implica analizar sus flujos de información de inicio a fin.
Para tal análisis se deben definir: políticas de flujo de información y
controles de flujo de información, es decir, las políticas de seguridad a
evaluar y los mecanismos para aplicarlas. 

\emph{Chequeo de tipos}: al usar un lenguaje tipado de seguridad, las políticas
son definidas a través del lenguaje, porque son expresadas mediante anotaciones
en el código fuente del programa a verificar, y su evaluación se realiza
mediante chequeo de tipos.\newline 
El chequeo de tipos consiste en una técnica estática,
también utilizada para analizar flujo de información durante la compilación de
un programa, más específicamente en la etapa de análisis semántico, el
compilador identifica el tipo para cada expresión del programa y verifica que
corresponda al contexto de la expresión.\newline
%  El
% chequeo de tipos también es una técnica estática utilizada para analizar flujo
% de información durante la compilación de un programa, más específicamente en la
% etapa de análisis semántico, el compilador identifica el tipo para cada
% expresión del programa y verifica que corresponda al contexto de la expresión.
Bajo este principio de chequeo, lenguajes tipados de seguridad aplican
políticas de control de flujo, definiendo para cada expresión del programa un
tipo de seguridad(security type), de la forma:  tipo de dato y label de
seguridad(security label). Donde el label de seguridad regula el uso del dato,
acorde a su tipo.\newline 
El compilador realiza el chequeo de tipos, partiendo del conjunto de labels de
seguridad. Así, si el programa pasa el chequeo de tipos y compila correctamente,
se espera que cumpla con las políticas de control de flujo evaluadas.  
\section{Trabajos Relacionados}
\label{sec:trabajo}
\subsection{JIF}
\label{JIF-Tool}
JIF(Java Information Flow), es un lenguaje tipado de seguridad que
permite extender el lenguaje de programación Java,  con control de flujo de
información y control de acceso, usando anotaciones de seguridad. El compilador
usa estas anotaciones durante el chequeo de tipos, verificando el
cumplimiento de la propiedad de seguridad non-interference.

Usar JIF para el análisis estático de flujo de información de un programa,
requiere implementar la versión del mismo, especificando mediante el conjunto de
labels de JIF, las políticas de seguridad a verificar. La implementación de
programas JIF está basada en el modelo de etiquetas DLM(Decentralized Label
Model), donde un principal es una entidad con autoridad para observar y cambiar
aspectos del sistema, así, un principal puede definir y hacer cumplir los
requerimientos de seguridad del dueño de la información. Para expresar una
relación de confianza entre principals, se define la relación acts-for, a partir
de la cual, se derivan dos tipos de principals: top principal y botton
principal, un top principal puede actuar para todos los principals, mientras
que, un botton principal permite que todos los principals actúen para el. Las
políticas de seguridad se condensan en Políticas de Confidencialidad y Políticas
de Integridad, con ellas se determina el conjunto de principals readers y
writes, y el comportamiento que deberían tener.
El compilador de JIF aplica chequeo de labels para verificar  el cumplimiento
de las políticas de seguridad definidas en el programa, cuando determina que
efectivamente las cumple, da paso al compilador de Java para generar su versión
ejecutable.

Además del modelo de labels en que se centra, JIF incluye mecanismos que
aportan características adicionales en la implementación de programas para
seguimiento de Flujo de información. La opción de flexibilizar las políticas
de seguridad de la información, hace parte de estas características adicionales,
y se logra aplicando el mecanismo Downgrading. Dependiendo del tipo política al
que se realiza downgrading, políticas de confidencialidad o políticas de
integridad, el proceso se conoce como Declasificación o Endorsement,
respectivamente.

\subsection{JOANA}
\label{JOANA-Tool}
JOANA (Java Object-sensitive ANAlysis)- Information Flow Control Framework for
Java\cite{JOANA}. Verifica si una aplicación java contiene fugas de
información, mediante análisis estático de flujos de información. El análisis parte  de anotaciones en
el código fuente de la aplicación. JOANA utiliza técnicas de análisis de flujo de
datos y técnicas de análisis de control de flujo. El frontend de la herramienta
está basado en el framework de análisis de programas WALA\cite{wala}, a partir
del cual obtiene la representación intermedia del código Java en forma SSA(Static
Single Assignement), lo que permite obtener información dinámica del programa.
Por otro lado, utiliza Grafos de Dependencia, System Dependence Graphs(SDG),
para detectar dependencias entre las sentencias del programa, es decir,
si existen caminos entre sentencias etiquetadas con nivel de seguridad
alto y sentencias con nivel de seguridad bajo. Para esta etapa del análisis
recurre a técnicas de slicing y chopping, reduciendo la cantidad de caminos
posibles sólo a los válidos. Así obtiene como resultado, una mayor precisión y
reducción de falsas alarmas en el análisis.\newline

Aunque JOANA provee sencillez a la hora de anotar el código a analizar, pues
sólo es necesario anotar inputs y outputs del programa, porque la herramienta se
encarga de propagar las anotaciones en el resto del programa; carece de
características adicionales ofrecidas por sistemas de tipado de seguridad, por
ejemplo, el mecanismo downgrading facilitado por JIF.\newline 

Si bien, al igual que JOANA, la herramienta propuesta a través del presente
trabajo, aplica análisis de control de flujo de información, esta última busca
analizar aplicaciones implementadas en código Android, aprovechando las ventajas
del sistema de anotaciones de JIF. Proporcionando una herramienta de apoyo al
desarrollador de aplicaciones Android, ya que por el momento, JOANA sólo analiza
aplicaciones en JAVA.

\subsection{JoDroid}
JoDroid\cite{JoDroid-Paper} es una extención a la herramienta de análisis JOANA
para soportar analisis de aplicaciones Android.\newline 
El análisis de JOANA está basado en Program Dependence Graphs(PDG) y técnicas
slicing. Con PDGs obtiene una representación del programa que
analiza, donde los nodos representan statements y expresiones; y las aristas
modelan las dependencias sintacticas entre los statements y expresiones:
dependencias de datos y dependencias de control, por tanto el grafo está en
capacidad de modelar, tanto flujos explícitos como flujos implícitos.\newline
Con técnicas slicing provee sensibilidad al contexto, puesto que el PDG se
construye de manera tal que al hacer el backwards slice de un determinado nodo,
se obtiene cada nodo que es alcanzable por caminos del grafo que conservan
llamadas al contexto.\newline
El PDG es generado mediante el Front-end de WALA, framework que analiza bytecode
de Java. Así, los ajustes hechos a JOANA adaptan parte del Front-end de WALA
para generar el PDG de aplicaciones Android.\newline
JoDroid detecta tanto flujos explícitos como flujos implícitos.

\subsection{FlowDroid}
\label{FlowDroid-Tool}
FlowDroid es una herramienta para análisis estático de flujo de datos en
Aplicaciones Android. También permite el análisis de aplicaciones Java.\newline
Esta herramienta utiliza un tipo especial de análisis de flujo de datos:
análisis tainting, que hace seguimiento al flujo de datos entre un conjunto de
sources y un conjunto de sinks. Define tales conjuntos a partir de
SuSi[\ref{sec:susi}], un clasificador automático de sources y sinks para la Api
de Android.\newline 
FlowDroid provee un alto recall y precisión\cite{FlowDroid-Thesis} en el
análisis. El recall, mediante un fiel modelamiento del ciclo de vida de una
aplicación Android; la precisión, incluyendo elementos de análisis como:
context-, flow-, field- y object-sensitive. Para proveer sensibilidad al flujo y
al contexto, recurre a grafos de llamada; y con grafos que modelan todos los
procedimientos del programa(inter-procedural control-flow graph), analiza el
flujo de datos entre procedimientos, proporcionando field- y object-sensitive.\newline
Los autores de esta propuesta, alcanzan precisión en la construcción del grafo
de llamadas extendiendo Soot\cite{Soot}, un framework que genera código
intermedio para código Java y código ejecutable Android(dex). Adicionalmente,
con el framework Heros\cite{heros}, incluyen llamadas multihilos en el análisis
de flujo de datos entre procedimientos.\newline

Entre las limitaciones de FlowDroid está el over-tainting y la no detección
de flujos implícitos. Por tanto, la herramienta no distingue elementos marcados
ni dentro de arrays, ni dentro de collections, si se inserta un elemento marcado
dentro de alguna de estas estructuras, inmediatamente se marca el resto de
elementos. La herramienta tampoco identifica flujos implícitos,    
% causados por dependencias entre control de flujo.\newline
puesto que, según los resultados de evaluación de
DroidBench\cite{DroidBenchBenchmarks}, su benchmark; cuando Flowdroid analiza el
conjunto de aplicaciones de prueba para la identificación de flujos implícitos, no
detecta fuga de datos, generando falsos negativos en la detección de flujos
implícitos\cite[pags 32-36]{FlowDroid-Thesis}.\newline

Aún cuando el problema a atacar es el mismo: fuga de información, la propuesta
que se expone a través del presente trabajo difiere en el enfoque de análisis de
FlowDroid, mientras FlowDroid se concentra en detectar si la aplicación de un
tercero presenta fugas de información, la herramienta planteada aborda el
análisis del lado del desarrollador de la aplicación, apoyándolo en
la verificación del cumplimiento de políticas de seguridad. Así, resulta más
conveniente guiar el análisis mediante control de flujo de información, ya que
se previene fuga por datos no marcados para el análisis(under-tainting) y por
la no detección de flujos implícitos, siendo posible garantizar el cumplimiento
de políticas de seguridad.
 
\subsection{TaintDroid, Dinamic Taint Tracking, para la detección de fugas de
Información}
\label{TaintDroid-Tool}
A diferencia de las propuestas expuestas anteriormente, caracterizadas
por ejecutar el análisis de manera estática, TaintDroid es una herramienta de
análisis dinámico. Está herramienta extiende la plataforma de dispositivos
celulares Android, con el fin de verificar el uso dado por aplicaciones de
terceros a datos sensibles del usuario. El análisis aplica técnicas de análisis
tainting, marcando automáticamente como sources, datos provenientes de fuentes
consideradas privadas y/o sensibles; y como sinks, canales que permiten salir
datos de la aplicación, como por ejemplo internet.
Cada vez que un dato marcado como source sale de la aplicación, se genera un log.\newline 
Para reducir sobrecarga en el dispositivo, pues el análisis es ejecutado a nivel
de instrucciones, instrumentan la máquina virtual de Android con marcas de
propagación a nivel de: variables, métodos, mensajes y archivos. Las marcas de
variable hacen seguimiento a datos dentro de aplicaciones consideras no
confiables. Las marcas de mensaje siguen mensajes entre aplicaciones. Debido a
que TaintDroid no hace seguimiento a la ejecución de código nativo, utiliza las
marcas de métodos para hacer seguimiento a lo retornado luego de invocar métodos
de librerías nativas. Las marcas de archivo son utilizadas para verificar la
persistencia de los datos, acorde a las políticas de seguridad.\newline 
Otra medida para reducir sobrecarga en la ejecución del análisis, consiste en no
hacer seguimiento a flujos de control, generando no detección de flujos
implícitos\cite[pag 12]{TaintDroid}.\newline
Si bien, TaintDroid supera el inconveniente de sobrecarga en la ejecución del
análisis, un inconveniente característico en análisis dinámico, está limitado
para detectar fuga de datos mediante flujos implícitos, puesto que se
enfoca en hacer seguimiento a flujos de datos directos(flujos
explícitos).\newline

Al ser una herramienta de análisis dinámico, TaintDroid sólo detecta fugas de
información correspondiente a las ejecuciones presentadas por el programa, y
para la finalidad de su análisis: informar al usuario de posibles fugas de
información, se puede decir que es adecuado. No obstante, para los propósitos de
la propuesta planteada a través del presente trabajo, con la que se pretende
brindar una herramienta de análisis para que el desarrollador verifique el
cumplimiento de políticas de seguridad en la aplicación que implementa, no
resulta viable aplicar análisis dinámico, ni técnicas de análisis tainting para
hacer seguimiento a flujos de datos.
%\subsection{STAMP Análisis estático de aplicaciones}

\subsection{Comparación de técnicas}
Las técnicas utilizadas para análisis de seguridad en aplicaciones, pueden
aplicarse estática o dinámicamente, dependiendo de las propiedades del programa
en que se centre el análisis.\newline
La ejecución dinámica o estática del análisis, trae sus propias ventajas y
desventajas. En el caso de análisis estático, completitud en el análisis es una
de sus principales ventajas. Esto debido a qué, el análisis contempla todas los
caminos de ejecución en que podría incurrir el programa. Evitando que se pierdan
casos a analizar. Por otra parte, al carecer de información que sólo se puede
obtener a tiempo de ejecución, por ejemplo, las entradas que el programa
recibe, el análisis estático suele generar falsos positivos.\newline
En el análisis dinámico, una de las principales ventajas es la baja generación
de falsos positivos, puesto que, el análisis no se centra en los posibles casos
de ejecución, sino que verifica el caso de ejecución que efectivamente está
ocurriendo. No obstante, el análisis dinámico podría incurrir en incompletitud,
porque sólo verifica los casos de ejecución que se presenten, es decir, el
aplicativo podría presentar fugas de información no reportadas por el análisis,
como consecuencia de la no ejecución de los casos que permiten identificarlos.\newline 
Así, el análisis dinámico genera menor cantidad de falsos positivos que el
análisis estático, sin embargo, el análisis estático ofrece mayor completitud en
el análisis.\newline
% Ahora, partiendo del contexto de análisis planteado en el presente trabajo,
% donde el desarrollador cuenta con el código fuente de su propia aplicación y
% pretende garantizar que esta cumple con determinadas políticas de seguridad, la
% característica de completitud en el análisis estático, es cable para garantizar
% el cumplimiento de políticas de seguridad.\newline
Adicional a la forma en que son aplicadas, estática o dinámicamente, las
técnicas de análisis pueden enfocarse en hacer seguimiento al flujo de datos a
través del programa, o en verificar flujos de información. Las técnicas basadas
en tanting análisis, permiten hacer análisis de flujo de datos, marcando los
datos de interés y verificando su flujo entre sources(fuentes del programa
consideradas sensibles y/o confidenciales) y sinks(destinos considerados no
confiables). Entre las desventajas de está técnica, esta el under-tainting, es
decir, la posibilidad de fugas a través de datos no marcados para el
análisis.\newline
Las técnicas para aplicar análisis mediante control de flujo de información,
generalmente permiten definir anotaciones de seguridad en el código fuente de la
aplicación, para verificar sus flujos de información. Estas generalmente se 
basan en técnicas de seguridad de tipado(Security-Typed Analyses), o en grafos
que describen el comportamiento del programa, como Contol Dependence Graphs(PDG)
y System Dependence Graphs(SDG).
Ambas técnicas recurren a etapas de análisis de compilación(se basan en
técnicas de compilación), sin embargo, mientras las técnicas de Security-Typed
sólo requieren llegar hasta el chequeo de tipos; las basadas en grafos de
dependencia deben llegar hasta la representación de código intermedio para
generar los respectivos grafos. Si bien, con grafos de dependencia se tiene
mayor precisión en el análisis, su ejecución es costosa, ya que genera una
complejidad de orden polinomial, O(N)3\cite[page 3]{FCO-PDG}.
Las motivaciones para guiar el análisis bajo una u otra perspectiva, implica
poner a consideración tanto el nivel de precisión requerido por las propiedades
de seguridad a evaluar, como el costo de implementación y de ejecución del
análisis. \newline


 
%profundizar en las de análisis
% estático, security Typed y control flow \begin{itemize}

% 	  \item El uso de lenguajes de seguridad tipados para el análisis de flujo de
% 	  información en tiempo de ejecución, puede generar sobrecargas.\cite[pag.~1]{LanguageIFS-2013}
% 	  \item Detección de implicit information flows mediante: static enforcements
% 	  of information-flow control versus, run-time enforcement mechanisms.
% 	  \item 
% 	\end{itemize}
% 
% Dentro de las técnicas existentes para adelantar análisis de seguridad en
% aplicativos 
% Para verificar propiedades de seguridad en los aplicativos que implementa, , 
% \begin{itemize}
% 	  \item Information Flow Control
% 	  \item 
% 	  \item 
% 	\end{itemize}
	
\subsection{Clasificación de Sources y Sinks}
\label{sec:susi}
En el ámbito de análisis de flujo de información de aplicaciones,
independientemente del tipo de análisis, estático o dinámico, el punto de
partida es la definición de políticas de privacidad, los pasos sucesivos para 
detectar la perdida de información giran en torno a las políticas de privacidad
definidas.
Muchas de las propuestas para análisis de flujo de información en aplicaciones
Android, parten de un listado de sources y sinks para definir sus políticas de
privacidad. Así, en el grupo de sources se incluyen las fuentes de datos
sensibles, mientras que en el grupo de sinks, se incluyen los medios o canales
que podrían filtrar información sensible de forma no autorizada. 
La efectividad del análisis se ve limitada al listado de sources y sinks, y la
veracidad de los mismos. El inconveniente con estos sources y sinks, es que su
clasificación suele hacerse de forma manual, por tanto, existe mayor
probabilidad de error u omisión.\newline
Con el fin de precisar dicha clasificación, el trabajo de investigación SuSi
propone el uso de machine-learning para la clasificación y categorización de
sources y sinks, partiendo del código fuente de la API Android.
La propuesta de análisis se materializa en una herramienta, que recibe como
entrada métodos de Android y devuelve una lista con la respectiva
categorización de sources y sinks.\newline
La construcción del modelo de
análisis propuesto, parte definiendo los elementos necesarios para el
reconocimiento de sources y sinks; inicialmente define:
Sources y sinks, respectivamente, como las entradas y salidas de flujo de datos del
programa; un dato como un valor o una referencia a un valor; un Resource Method
como un método que lee o escribe datos en un recurso compartido. Seguidamente,
define el concepto de sources y sinks, considerando el contexto de Android:
Android Sources como llamadas a métodos tipo resources(Resources method) que
retornan valores no constantes al código de la aplicación. Android Sinks como
llamadas a methods resource, aceptando como argumento al menos un valor no
constante desde el código de la aplicación, y qué además adicionen o modifiquen
valores del recurso invocado.
El modelo de entrenamiento de SuSi usa el clasificador SMO, una implementación
del clasificador SVM(Support Vector Machines) para Weka, al que inicialmente
enseña a clasificar partiendo de ejemplos entrenados manualmente.
Adicionalmente, lo adapta utilizando la técnica de clasificación
one-againts-all, de modo que pueda representar, tanto los ejemplos de
entrenamiento, en tres clases: sources, sinks, o ninguno; como las
categorías de los sources y sinks identificados.\newline 
Los criterios de clasificación están basados en un conjunto de características,
es decir, funciones que asocian ejemplos de entrenamiento o ejemplos de prueba,
con un determinado valor.\newline
El proceso de análisis se compone de dos rondas secuenciales: clasificación y
categorización. Cada una se compone de las fases input, preparation,
classification y output. Así, la salida de la primera ronda: sources y sinks, se
convierte en entrada para la ronda de categorización, donde se definen
diferentes tipos de categorías, 12 para sources y 15 para sinks.
\section{Background}
\label{sec:back}

\subsection{Aplicaciones Android}
Explicar composición de aplicaciones Android, actividades, servicios, etc.

\subsection{Estructura de trabajo en JIF}
- estructura de los directorios del compilador Jif y estructura de trabajo en
Jif(para entender cómo funciona y cómo afecta el diseño de la
solución).

\subsection{Sintaxis de Anotación en Jif}
\label{subsec:JifSintax}
-Definición de variables: \newline 
\emph{ type\{L\} varName; }\newline 
donde type especifica el tipo de dato que
almacena la variable, \{L\} el label de seguridad  para especificar quien es el
dueño de la variable, y name, el respectivo nombre de la variable.

-Definición de arrays:\newline
en jif un array cuenta con dos labels de seguridad, Base Label(BL) y Size
Label(SL). BL indica el nivel de seguridad de los elementos que almacena el
array, controlando quien puede conocer la información del mismo. SL especifica
quienes pueden conocer la número de elementos almacenados.

-Definición de métodos.\newline
\emph{ type \{RTL\} methodName \{BL\} (arg1\{AL\},,, argn\{AL\}) :\{EL\}
}\newline 
RTL, Return Type Label, indica el label de seguridad con que
queda el tipo de dato devuelto por el método.\newline 
BL begin label, representa el máximo nivel se seguridad del pc label desde donde
se invoca el método, de este modo, el program counter label desde donde
se invoca el método debe ser menor o igual de restrictivo que el BL del
método.\newline 
AL argument label, indica el máximo nivel de seguridad  para los argumentos con
que se llama el método, así, los labels de los argumentos con que se invoca el
método deben ser menor o igual de restrictivos que los AL con que han
definido el método.\newline
EL end label, indica el pc label en el punto de terminación del método, y
representa la información que puede ser conocida.\newline
Cuando un label no es especificado, Jif define unos por defecto. En el caso de
RTL, jif hace un join entre los diferentes AL con que ha sido definido el
método.\newline














