\label{ch:introduccion}
\chapter{Introducción}
En la introducción tratar de seguir una organización similar a la presentación:
contexto general, problema particular, contexto (propuestas previas y sus
limitaciones), propuesta, y evaluación\newline

En aplicativos Android, el manejo de la información del usuario, es una de las
principales preocupaciones de seguridad. Según un estudio reciente de seguridad
en dispositivos móviles, publicado por McAfee\cite{McAfeeReport}, una importante
cantidad de aplicaciones Android invaden la privacidad del usuario, reuniendo
información detallada de su desplazamiento, acciones en el dispositivo, y su
vida personal.\newline 
Por otro lado, para controlar el acceso a información manipulada por sus
aplicaciones, el desarrollador cuenta con los mecanismos de seguridad proveídos
por la API de Android, sin embargo, al estar basados en políticas de control de
acceso, se limitan a verificar el uso de los recursos del sistema acorde a los
privilegios del usuario, lo que suceda con la información una vez sea accedida,
está fuera del alcance de este tipo de controles. Al no contar con herramientas
de análisis de flujo de información en aplicaciones Android, o al utilizar
librerías de terceros, para el desarrollador es difícil verificar
el cumplimiento de políticas de confidencialidad e integridad en la aplicación
próxima a liberar. Por consiguiente, el desarrollador no tiene cómo asegurar la
ausencia de fugas de información en la aplicación.\newline 
Si bien, en el campo de aplicativos Android existen diferentes propuestas para
detectar fuga de información, en su mayoría  se enfocan en precisión y
eficiencia del análisis para detectar fugas de datos en aplicaciones de terceros
ya implementadas. Estas propuestas
no abordan el problema del lado del desarrollador, analizando flujos de
información de la aplicación para verificar el cumplimiento de políticas de
confidencialidad e integridad.\newline

Ante esto, y con el fin de proveer una herramienta de apoyo al desarrollador, de
modo que verifique el cumplimiento de políticas de seguridad en sus
aplicaciones, el presente trabajo aborda el problema de fugas de información en
aplicaciones Android, analizando flujos de información de la aplicación mediante
técnicas de lenguajes tipados de seguridad.



