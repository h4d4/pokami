\chapter{Introducción}
\label{ch:introduccion}
El presente capítulo introduce el tema de investigación de la tesis, el
problema, las limitaciones de los trabajos relacionados,  la propuesta de
solución planteada y la evaluación realizada.\newline

En aplicativos Android, el manejo de la información del usuario es una de las
principales preocupaciones de seguridad. Según un estudio reciente de seguridad
en dispositivos móviles publicado por McAfee\cite{McAfeeReport}, una importante
cantidad de aplicaciones Android invaden la privacidad del usuario, recolectando
información detallada de su desplazamiento, acciones en el dispositivo, y su
vida personal, con el fin de enviar tal información hacia servidores de
compañías publicitarias.\newline 
Por otro lado, para controlar el acceso a información manipulada por sus
aplicaciones, el desarrollador cuenta con los mecanismos de seguridad proveídos
por la API de Android, sin embargo, al estar basados en políticas de control de
acceso, se limitan a verificar el uso de los recursos del sistema acorde a los
privilegios del usuario, lo que suceda con la información una vez sea accedida,
está fuera del alcance de este tipo de controles.\newline 
Al no contar con herramientas de análisis de flujo de información para
aplicaciones Android, o al utilizar librerías de terceros, para el desarrollador
es difícil verificar el cumplimiento de políticas de confidencialidad e
integridad en la aplicación próxima a liberar. Por consiguiente, el
desarrollador no tiene cómo garantizarle al usuario la ausencia de fugas de
información en la aplicación que le provee.\newline 
Ahora bien, aunque en el campo de aplicativos Android se han propuesto
diferentes trabajos para detectar fuga de información(propuestas como
Flow-Droid\cite{FlowDroid-Thesis}, JoDroid\cite{JoDroid-Paper},
TaintDroid\cite{TaintDroid}, DidFail\cite{DidFail},
DroidForce\cite{DroidForce}), en su mayoría, las propuestas existentes se
enfocan en precisión y eficiencia del análisis, con el fin de detectar
fugas de datos en aplicaciones de terceros ya implementadas.
Estas propuestas no abordan el problema del lado del desarrollador, de
manera tal que se analice el flujo de información de la aplicación verificando
el cumplimiento de políticas de seguridad.\newline 
Ante esto, y con el fin de proveerle al desarrollador una
herramienta para verificar políticas de seguridad desde la
construcción de sus aplicativos, el presente trabajo aborda el problema de fugas
de información en aplicaciones Android, analizando flujos de información de la
aplicación con técnicas de lenguajes tipados de seguridad.\newline 
Así pues, se propone una herramienta para análisis de flujo de información de
aplicativos Android mediante el sistema de anotaciones de Jif.\newline 
El diseño ideal para implementar la herramienta de análisis, sugiere anotar
todas las clases de la API Android, sin embargo, para efectos de la presente
tesis se parte de un conjunto reducido de clases de la API Android y un conjunto
específico de sources y sinks, acorde a una política de seguridad establecida. 
De este modo, el desarrollador define la política de seguridad en su aplicativo
mediante el sistema de anotaciones, y verifica el cumplimiento de la misma con
el compilador de Jif.\newline 
Teniendo la respectiva implementación de la propuesta planteada, se parte de
Droibench \cite{DroidBenchBenchmarks}, un benchmark diseñado específicamente
para verificar fugas de información en aplicativos Android. De este benchmark se
define un conjunto que permite evaluar la política de seguridad definida.\newline
Los resultados de evaluar el conjunto de pruebas coinciden con las hipótesis
iniciales, puesto que, al estar basada en control de flujo de información, el análisis es más rápido,
menos preciso pero a la vez completo. Es decir, se reportan más falsos positivos
pero a la vez, se pierden menos fugas de información.

Finalmente, dado que Jif permite
anotar código Java pero no código Android, es decir, las anotaciones Jif son
válidas para clases del lenguaje Java estándar, no para clases específicas de la
API del framework Android, las cuales son indispensables para
implementar las funcionalidades de aplicativos Android; la principal
contribución de la presente tesis consiste en: proveerle al desarrollador de
aplicativos Android una herramienta que permite definir y verificar políticas de
confidencialidad en sus aplicativos, con el sistema de anotaciones de
Jif.\newline


