\label{ch:evaluacion}
\chapter{Evaluación}
% Ventajas y limitaciones de la solución.\newline 
% Si aplica, evaluación de desempeño.  \newline 
% Si aplica, evaluación de usabilidad.  
% Hay otras soluciones similares? \newline 
% Cuáles son las diferencias y las ventajas y desventajas con respecto a esas soluciones.

\section{Consideraciones de evaluación}
El flujo de información es analizado al interior de una sóla aplicación, no se
consideran flujos de información vía interApp, es decir, varias aplicaciones que
se comunican entre sí.

\section{Conjunto de evaluación}
\label{sec:evalSet}
Para la evaluación se parte de DroidBech versión 1.0\cite{DroidBenchBenchmarks},
benchmark integrado por 39 casos de prueba para aplicaciones Android, cuyos
autores son los mismos creadores de FlowDroid. Se opta por utilizar DroidBench
puesto que, en la literatura científica consultada al respecto, no se encuentran
otros benchmarks diseñados específicamente para evaluar aplicaciones Android.\newline 
De DroidBech se toma un grupo de testcases evaluables frente a la política de
seguridad establecida\ref{subsection:politica}, este grupo está integrado por 20
testcases. De los cuales, 14 presentan fugas de información.

La tabla \ref{tab:descripApps0} describe parte del grupo de testcases a
evaluar. En los casos en que se requiere, se precisan observaciones entre los
resultados de evaluación esperados para la técnica de análisis utilizada por
FlowDroid(análisis de flujo de datos) y la técnica de análisis propuesta en el
presente trabajo(análisis de flujo de información).
En la sección\ref{sec:testcases} de los anexos, se encuentra la
descripción del grupo de prueba completo.

El conjunto de prueba es analizado con FlowDroid, JoDroid y con el Prototipo. Los
resultados del análisis que devuelve cada herramienta son calificados como:
Falso Positivo(FP) cuando se detecta un leak que no existe; Falso Negativo(FN)
cuando no se detecta un leak existente; Verdadero Positivo(TP) cuando se detecta
un leak existente; Verdadero Negativo(TN) cuando no existe leak que detectar.

En base a estos resultados se calcula la Precisión y el Recall, para cada una de
las herramientas evaluadas. La Precisión mide la cantidad de verdaderos
Positivos(TP) frente a la cantidad de Falsos positivos(FP), reportados por la
herramienta. A mayor Precisión, la herramienta detecta menos falsos
positivos(FP).\newline 
El Recall(\textit{r}) mide la cantidad de verdaderos Positivos(TP) frente a la
cantidad de Falsos negativos(FN), reportados por la herramienta. A mayor Recall,
menos leaks deja pasar la herramienta, es decir menor reporte de falsos
negativos. Las formulas \ref{p} y \ref{r}, permiten el calculo de ambas métricas
de seguridad.\newline
Adicionalmente, para medir el desempeño de cada herramienta se utiliza el
comando \textit{time}\cite{time-man} de unix. 

\begin{table}[H]
\small\addtolength{\tabcolsep}{-3pt}
\begin{tabular}{|p{13cm}|p{1cm}|}
	\hline
	\multicolumn{2}{|>{\columncolor[gray]{0.8}}c|}{\textbf{AndroidSpecific\_DirectLeak1}}\\
	\hline
	\textbf{Descripción} & \textbf{Leaks}\\
	\hline
	La variable \textit{mrg} tiene un nivel de seguridad alto,
	almacena información retornada por el método source \textit{getDeviceId}. Se
	genera flujo de información directo entre información con nivel de seguridad alto e
	información con nivel de seguridad bajo, al enviar como parámetro del método
	\textit{sendTextMessage}, información de la variable \textit{mrg}. & 1 \\
	\hline
	\multicolumn{2}{|>{\columncolor[gray]{0.8}}c|}{\textbf{AndroidSpecific\_LogNoLeak}}\\
	\hline
	\textbf{Descripción} & \textbf{Leaks}\\
	\hline
	El caso de prueba no presenta información con niveles de seguridad alto. Se
	presentan flujos de información entre información con el mismo nivel de
	seguridad, en este caso bajo, lo cual es permitido. & 0 \\
	\hline
	\multicolumn{2}{|>{\columncolor[gray]{0.8}}c|}{\textbf{ArraysAndLists\_ArrayAccess1}}\\
	\hline
	\textbf{Descripción} & \textbf{Leaks}\\
	\hline
	Se tiene un array en que se almacena información tanto proveniente como no
	proveniente de sources, parte de la información que almacena es enviada como
	parámetro del método \textit{sendTextMessage}. \textit{Observación:}
	Para la técnica de análisis de FlowDroid(taint analysis), se marca únicamente el
	índice del array donde se almacena el dato considerado como source, así,
	cuando se envía como parámetro del método \textit{sendTextMessage},
	el dato de un índice no marcado, no se genera leak. Para nuestra técnica
	de análisis(flujo de información mediante JIF), para que un array almacene
	información con nivel de seguridad alto, primero debe ser catalogo(anotado)
	con nivel de seguridad alto, lo que implica que el array podrá almacenar
	información tanto de nivel de seguridad alto como bajo, pero toda la
	información quedará con nivel de seguridad alto. En consecuencia, al enviar
	cualquier índice del array como parámetro del método 
	\textit{sendTextMessage} se presenta un flujo de información no
	permitido. & 0
	\\
	\hline
	\multicolumn{2}{|>{\columncolor[gray]{0.8}}c|}{\textbf{GeneralJava\_Exceptions2}}\\
	\hline
	\textbf{Descripción} & \textbf{Leaks}\\
	\hline
	La variable \textit{imei} es de nivel de seguridad alto, almacena información
	devuelta por el método \textit{getDeviceId}. El control de flujo del
	programa conduce de manera implícita a la captura de una excepción tipo
	RuntimeException, desde allí se utiliza información proveída por la variable
	\textit{imei}, como parámetro para invocar el método \textit{sendTextMessage}.
	Generando un flujo de información indebido. & 1
	\\
	\hline
	\multicolumn{2}{|>{\columncolor[gray]{0.8}}c|}{\textbf{ImplicitFlows\_ImplicitFlow2}}\\
	\hline
	\textbf{Descripción} & \textbf{Leaks}\\
	\hline
	 La variable \textit{userInputPassword} con nivel de seguridad alto, almacena
	 información de un campo EditText tipo textPassword(password suministrado por
	 el usuario). Se generan flujos de información indebidos: al tratar de asignar
	 información a la variable passwordCorrect con nivel de seguridad bajo, a
	 partir de la comparación de información con nivel de seguridad alto(variable
	 textPassword), después, al tratar de mostrar en el \textit{log} información
	 que depende de tal comparación. & 1\\
	\hline
\end{tabular}
\caption{Descripción aplicaciones de prueba.\newline
Se considera con nivel de seguridad alto, variables y métodos que almacenan y
modifican(respectivamente), información catalogada como privada(Sources).\newline 
Se considera con nivel de seguridad bajo, canales para envío de mensajes,
muestra de logs y canales creados durante el control de flujo del programa.\newline }
\label{tab:descripApps0}
\end{table}

\subsection{Evaluación FlowDroid y Prototipo } 
\label{subsec:fvsp}
Del total de testcases(20), 14 presentan fugas de información mientras que 6 de
ellos no. Los resultados de evaluación para FlowDroid y el Prototipo, son
presentados en la tabla \ref{tb:resultados}. En esta, por cada caso de prueba
se indica la cantidad de leaks que presenta, el resultado y el tiempo que tarda el
análisis.

\begin{table}[H]
\begin{center}
\small\addtolength{\tabcolsep}{-3pt}
\begin{tabular}{|p{6cm}|p{1cm}|p{1cm}|p{1cm}|p{1cm}|p{1cm}|}
	\hline
	\textbf{Testcase} & \textbf{Leaks} & \textbf{F} &
	\textbf{P} & \textbf{ tF} & 
	\textbf{tP}\\
	\hline
	AndroidSpecific\_DirectLeak1 & 1 & TP & TP &5.371s &2.063s\\
	\hline
	AndroidSpecific\_InactiveActivity & 0 & TN & FP  &3.255s &2.469s\\
	\hline
	AndroidSpecific\_LogNoLeak & 0 & TN & TN &5.505s &2.946s\\
	\hline
	AndroidSpecific\_Obfuscation1 & 1 & TP & TP &6.734s &2.706s\\
	\hline
	 AndroidSpecific\_PrivateDataLeak2 & 1 & TP & TP & 6.144s &2.644s\\
	\hline
	 ArraysAndLists\_ArrayAccess1 & 0 & FP & FP & 4.708s & 1.278s\\
	\hline
	 ArraysAndLists\_ArrayAccess2 & 0 & FP & FP & 4.4s &1.361s\\
	 \hline
	 GeneralJava\_Exceptions1 & 1 & TP & TP &6.397s &2.755s\\
	\hline
	 GeneralJava\_Exceptions2 & 1 & TP & TP &5.887s &1.980s\\
	\hline
	GeneralJava\_Exceptions3 & 0 & FP & FP &6.008s &2.032s\\
	\hline
	GeneralJava\_Exceptions4 & 1 & TP & TP &5.731s &2.313s\\
	\hline
	GeneralJava\_Loop1 & 1 & TP & TP &5.605s &2.800s\\
	\hline
	GeneralJava\_Loop2 & 1 & TP & TP &4.719s &1.361s\\
	\hline
	GeneralJava\_UnreachableCode & 0 & TN & FP &3.792s &1.197s\\
	\hline
	ImplicitFlows\_ImplicitFlow1 & 1 & FN & TP &4.853s &1.331s\\
	\hline
	ImplicitFlows\_ImplicitFlow2 & 1 & FN & TP &4.496s &1.212s\\
	\hline
	ImplicitFlows\_ImplicitFlow4 & 1 & FN & TP &4.375s &1.224s\\
	\hline
	Lifecycle\_ActivityLifecycle3 & 1 & TP & TP &4.792s &1.222s\\
	\hline
	Lifecycle\_BroadcastReceiverLifecycle1 & 1 & TP & TP &4.456s &1.061s\\
	\hline
	Lifecycle\_ServiceLifecycle1 & 1 & TP & TP &5.225s &1.180s\\
	\hline
\end{tabular}
\end{center}
\caption{Resultados de evaluación para FlowDroid y Prototipo. Donde
\textit{Testcase} especifica el nombre de la aplicación que se está evaluando;
\textit{Leaks} indica si el testcase presenta fugas de información;
\textit{F} y  \textit{P} muestran los resultados devueltos por FlowDroid y por
el Prototipo; \textit{tF} y \textit{tP}, señalan el tiempo(en segundos) que toma
el análisis para Flowdroid y para el Prototipo, respectivamente.}
\label{tb:resultados}
\end{table}
\subsubsection{Análisis de evaluación entre FlowDroid y Prototipo}
-\textit{Resultados de desempeño}\newline
Acorde a los datos señalados en los campos tF y tP de la tabla
\ref{tb:resultados}, en promedio, FlowDroid tarda 3,3 segundos más que el
Prototipo para ejecutar el análisis.

% PULIR IDEAAAAA***\newline
% Tal diferencia de tiempo va en concordancia con los costos de ejecución que
% implica una u otra técnica de análisis, mientras el análisis de flujo de
% información basado en lenguajes tipados de seguridad(que es lo que se evalua
% mediante el prototipo) sólo requiere llegar hasta el chequeo de tipos, la
% técnica de analisis de flujo de datos en que se basa FlowDroid utiliza
% algoritmos de orden O(ED)3 , cuya complejidad es de orden polinomial,
% O(ED)3\cite[page 3]{FCO-PDG}.\newline
% Se podría destacar como positivo que el análisis de flujo de información
% mediante técnicas de tipado de seguridad, requiere menos tiempo que la técnica
% de marcado de datos utilizada por FlowDroid.
-\textit{Resultados de precisión}\newline
En lo que respecta a los resultados del Prototipo, los FP correspondientes a
AndroidSpecific\_InactiveActivity y GeneralJava\_UnreachableCode, surgen como
consecuencia de un análisis pesimista, donde se asume que el desarrollador
utiliza lo que implementa, es decir, que los métodos serán invocados.\newline 
Por otro lado, en el caso de ArraysAndLists: ArrayAccess1 y ArrayAccess2, no es
sencillo calificar los resultados como FP, puesto que, para lo que está
analizando FlowDroid(verificar que su técnica de análisis diferencie entre los
elementos marcados y no marcados de un array), efectivamente se presentan FP,
sin embargo, para la forma en que se deben implementar los programas en Jif,
donde se suele definir un nivel de seguridad para todo el array antes de
almacenar los elementos en el mismo, podría decirse que no se trata de un FP,
porque se revelo información que había sido catalogada con nivel de seguridad alto.

A diferencia de FlowDroid, el Prototipo detecta fugas de información través de
flujos implícitos. La no detección de Flujos implícitos por parte de FlowDroid,
responde al tipo de análisis y las técnicas en que se fundamenta la herramienta:
análisis de flujo de datos mediante técnicas tainting. Basada en 
%Puesto que, 
% 
% \newline
% --el Prototipo, la anotación generada por el prototipo, o la técnica de análisis
% en que se basa el diseño de la solución: lenguajes tipados de seguridad???
% 
% -\textit{Acerca de por qué FlowDroid no detecta flujos implícitos}\newline
%el análisis de FlowDroid utiliza  técnicas DataFlow, específicamente, utiliza
% tainting análisis.
% Para hacer seguimiento al flujo de información de un
% programa, la técnica de análisis tainting se basa en: 
asociar una o más marcas
con el valor de los datos en el programa, y en propagarlas. Dependiendo de los
criterios definidos para el análisis, la marca puede ser propagada a causa de
flujos explícitos o de flujos implícitos, o a causa de ambos. En flujos
explícitos la propagación ocurre cuando el valor de una variable marcada está
implicada en el calculo de otra variable. En flujos implícitos la propagación
tiene lugar a través de dependencias en el control de flujo del programa, por
ejemplo, cuando el valor de un dato marcado afecta indirectamente otra variable.\newline 
En el caso de FlowDroid, los criterios que fundamentan el análisis de la
herramienta, hacen que el marcado de datos se propague para flujos explícitos y
y no para flujos implícitos. Por consiguiente, FlowDroid no detecta flujos
implícitos.\newline
Con esto presente, se definen dos escenarios de análisis: Precisión y Recall,
incluyendo flujos implícitos; y  Precisión y Recall, exlcuyendo flujos
implícitos, donde se incluyen u omiten los testcases para flujos
implícitos(ImplicitFlows\_ImplicitFlow 1, 2 y 4).
La tabla \ref{tb:FlowDroidPrototipoFI} muestra los resultados de evaluación para
ambos casos.\newline
%En el caso de TaintDroid, tampoco detecta flujos implícitos, porque entre las
% desiciones de diseño de la herramienta está enfocarse en el seguimeinto al
% flujo de datos y no al flujo de control, puesto que si incluyen seguimiento al
% flujo de control, se adiciona sobrecarga a la herramienta, la cual es de tipo
% dinámico.

\begin{table}[t!]
\begin{center}
\begin{tabular}{c|c|c|c|c|}
\cline{2-5}
& \multicolumn{2}{>{\columncolor{gray!30}} c  }{\multirow{1}{*}{Con FI}}
& \multicolumn{2}{|c|}{\multirow{1}{*}{\cellcolor{gray!50} {Sin FI}}}\\
\cline{2-5}
& \cellcolor{gray!30}FlowDroid & \cellcolor{gray!30}Prototipo &
\cellcolor{gray!50}FlowDroid & \cellcolor{gray!50}Prototipo \\
\cline{1-5}
\multicolumn{0}{ |c|  }{\multirow{0}{*}{FP} }  & 3 & 5 & 3 & 5\\ \cline{0-4}
\multicolumn{0}{ |c|  }{\multirow{0}{*}{FN} }  & 3 & 0 & 0 & 0\\ \cline{0-4}
\multicolumn{0}{ |c|  }{\multirow{0}{*}{TP} }  & 11 & 14 & 11 & 11 \\
\cline{0-4}
\multicolumn{0}{ |c|  }{\multirow{0}{*}{TN} }  & 3 & 1 &  3 & 1\\ \cline{0-4}
\end{tabular}
\end{center}
\caption{Resultados de precisión para FlowDroid y Prototipo, de acuerdo al
escenario, incluyendo o excluyendo flujos implicitos(FI). Resume el total de
falsos positivos(FP), verdaderos positivos(TP), verdaderos negativos(TN) y falsos
negativos(FN); obtenidos tanto con FlowDroid como con el Prototipo.}
\label{tb:FlowDroidPrototipoFI}
\end{table}
\textit{Precisión y Recall, incluyendo flujos implícitos}\newline
Los resultados obtenidos en la tabla \ref{tb:FlowDroidPrototipoFI} señalan que
de las 14 fugas existentes, el Prototipo las detecta todas, presenta 14
TP(verdaderos positivos); mientras que, FlowDroid deja pasar 3.\newline
Por otro lado, el Prototipo presenta más falsos positivos que FlowDroid, de los
6 testcases que no presentan leaks, el prototipo reporta 5 como si fuesen
fugas, mientras que FlowDroid reporta 3.\newline
Así, en lo que respecta a Precisión, FlowDroid presenta un porcentaje del 78\%,
siendo más preciso frente al Prototipo, que presenta un porcentaje del
73\%.\newline 
Por otro lado, el Prototipo presenta un porcentaje en Recall del 100\%,
mientras que FlowDroid presenta un porcentaje del 78\%.\newline

\textit{Precisión y Recall, exlcuyendo flujos implícitos}\newline
Excluyendo los testcases para flujos implícitos, el conjunto de prueba se reduce
a 17 casos. De los cuales, 11 presentan leaks.\newline 
En este contexto, el porcentaje de Precisión para FlowDroid es de 78,57\%,
mientras que para el Prototipo es del 68,75\%. El porcentaje de Recall es igual
para FlowDroid y el Prototipo: 100\%.


\subsection{Evaluación JoDroid y Prototipo}
% Del conjunto de casos de prueba JoDroid ignora las excepciones, ya que su actual
% implementación no soporta análisis del flujo de información a través de
% tales sentencias[pag 93 \cite{JoDroid-Thesis}]. Por tanto, los testcases
% GeneralJava\_Exceptions1 a GeneralJava\_Exceptions4, no son evaluados. 
% No obstante, para calcular la precisión y el recall se consideran dos
% escenarios, uno en que se incluyen y otra en que no.
A continuación se ilustran los resultados de evalución para JoDroid y el
Prototipo, en base a los cuales, se calculan las métricas de precisión y recall. 
\label{subsec:jvsp}
\begin{table}[H]
\begin{center}
\small\addtolength{\tabcolsep}{-3pt}
\begin{tabular}{|p{6cm}|p{1cm}|p{1cm}|p{1cm}|p{1,8cm}|p{1cm}|}
	\hline
	\textbf{Testcase} & \textbf{Leaks} & \textbf{J} &
	\textbf{P} & \textbf{ tJ} & 
	\textbf{tP}\\
	\hline
	AndroidSpecific\_DirectLeak1 & 1 & TP & TP & 22m11.991s &2.063s\\
	\hline
	AndroidSpecific\_InactiveActivity & 0 & FP & FP  & 22m25.617s &2.469s\\
	\hline
	AndroidSpecific\_LogNoLeak & 0 & TN & TN & 21m6.548s &2.946s\\
	\hline
	AndroidSpecific\_Obfuscation1 & 1 & TP & TP &22m46.541s&2.706s\\
	\hline
	 AndroidSpecific\_PrivateDataLeak2 & 1 & TP & TP &21m32.447s&2.644s\\
	\hline
	 ArraysAndLists\_ArrayAccess1 & 0 & FP & FP &22m01.926s& 1.278s\\
	\hline
	 ArraysAndLists\_ArrayAccess2 & 0 & FP & FP &22m11.023s&1.361s\\
	 \hline
	 GeneralJava\_Exceptions1 & 1 & FN & TP & 22m52.134s &2.755s\\
	\hline
	 GeneralJava\_Exceptions2 & 1 & FN & TP & 21m4.434s&1.980s\\
	\hline
	GeneralJava\_Exceptions3 & 0 & TN\tablefootnote{Al igual que en
	los testcases GeneralJava\_Exceptions(1, 2 y 4), la herramienta no detecta
	leaks, la diferencia para el presente caso, es que efectivamente no existe
	leak. Por tanto se califica como TN.} & FP & 21m37.040s &2.032s\\
	\hline
% 	\hline
% 	GeneralJava\_Exceptions3 & 0 & TN & FP & 21m37.040s &2.032s\\
% 	\hline
	GeneralJava\_Exceptions4 & 1 & FN  & TP & 21m10.240s &2.313s\\
	\hline
	GeneralJava\_Loop1 & 1 & TP & TP &21m15.30s&2.800s\\
	\hline
	GeneralJava\_Loop2 & 1 & TP & TP &21m41.224s&1.361s\\
	\hline
	GeneralJava\_UnreachableCode & 0 & TN & FP &22m84.138s&1.197s\\
	\hline
	ImplicitFlows\_ImplicitFlow1 & 1 & TP & TP &22m55.645s&1.331s\\
	\hline
	ImplicitFlows\_ImplicitFlow2 & 1 & TP & TP &22m32.231s&1.212s\\
	\hline
	ImplicitFlows\_ImplicitFlow4 & 1 & TP & TP &22m43.110s&1.224s\\
	\hline
	Lifecycle\_ActivityLifecycle3 & 1 & TP & TP &22m54.651s&1.222s\\
	\hline
	Lifecycle\_BroadcastReceiverLifecycle1 & 1 & TP & TP &22m42.347s&1.061s\\
	\hline
	Lifecycle\_ServiceLifecycle1 & 1 & TP & TP &22m92.722s&1.180s\\
	\hline
\end{tabular}
\end{center}
\caption{Resultados de evaluación para JoDroid y Prototipo. Donde
\textit{Testcase} especifica el nombre de la aplicación que se está evaluando;
\textit{Leaks} indica si el testcase presenta fugas de información; \textit{J} y
\textit{P} muestran los resultados devueltos por JoDroid y por el Prototipo;
\textit{tJ} y \textit{tP}, señalan el tiempo que toma el análisis para JoDroid
y para el Prototipo, respectivamente.}
\label{tab:JoDroid-Prototipo}
\end{table}

\subsubsection{Analisis de evaluación entre JoDroid y Prototipo}
-\textit{Resultados de desempeño}\newline
Para el analisis mediante JoDroid se deben seguir una seríe de pasos, tal como
se describen en el manual de referencia\cite{joDroidManual}, de estos,
únicamente se contabiliza el tiempo correspondiente al paso para generar el
grafo de dependencia del programa(PDG), del cual parte el análisis. En general,
el tiempo que tarda la generación del PDG para cada aplicación analizada, oscila
entre 21 y 22 minutos. Cabe anotar que estos valores podrían cambiar con otras
características de hardware, sin embargo, asignando 1048 MB de Ram a la máquina
virtual de Java, para la generación del PDG, es ese el rango de tiempo obtenido.
En consecuencia, para los valores de tiempo que señala la presente evaluación,
es posible decir que la herramienta es costosa en desempeño.

-\textit{Resultados de Precisión}\newline
La tabla \ref{tab:JoDroid-Prototipo} muestra que al igual que el Prototipo,
JoDroid detecta fugas de información presentes en flujos implícitos.\newline
Por otro lado, es importante resaltar que del conjunto de casos de prueba,
JoDroid ignora el control de flujo de información para
excepciones(GeneralJava\_Exceptions1 a GeneralJava\_Exceptions4), puesto que, su
actual implementación no soporta análisis del flujo de información a través de
tales sentencias[pag 93 \cite{JoDroid-Thesis}]. En la
tabla\ref{tb:jodroidP2exce} se muestran dos excenarios para los resultados de
evaluación: Con  excepciones y Sin excepciones. Con base en tales resultados se
calcula la Precisión y Recall para cada uno de los escenarios.

\begin{table}[t!]
\begin{center}
\begin{tabular}{c|c|c|c|c|}
\cline{2-5}
& \multicolumn{2}{>{\columncolor{gray!30}} c  }{\multirow{1}{*}{Con
excepciones}} & \multicolumn{2}{|c|}{\multirow{1}{*}{\cellcolor{gray!50} {Sin
excepciones}}}\\
\cline{2-5}
& \cellcolor{gray!30}JoDroid & \cellcolor{gray!30}Prototipo &
\cellcolor{gray!50}JoDroid & \cellcolor{gray!50}Prototipo \\
\cline{1-5}
\multicolumn{0}{ |c|  }{\multirow{0}{*}{FP} }  & 3 & 5 & 3 & 4\\ \cline{0-4}
\multicolumn{0}{ |c|  }{\multirow{0}{*}{FN} }  & 3 & 0 & 0 & 0\\ \cline{0-4}
\multicolumn{0}{ |c|  }{\multirow{0}{*}{TP} }  & 11 & 14 & 11 & 11 \\
\cline{0-4}
\multicolumn{0}{ |c|  }{\multirow{0}{*}{TN} }  & 3 & 1 &  2 & 1\\ \cline{0-4}
\end{tabular}
\end{center}
\caption{Resultados de precisión para JoDroid y Prototipo. Muestra los
escenarios en que mide. Resume el total de falsos positivos(FP), verdaderos
positivos(TP), verdaderos negativos(TN) y falsos negativos(FN); obtenidos tanto
con JoDroid como con el Prototipo.}
\label{tb:jodroidP2exce}
\end{table}

\textit{Precisión y Recall incluyendo excepciones}\newline
Del total de testcases(20), 14 presentan fugas de información. 
De los casos con fuga de información, 3 corresponden a las excepciones
incluidas(GeneralJava\_Exceptions), y se califican como falsos negativos(FN)
puesto que JoDroid no los detecta. La tabla\ref{tb:jodroidP2exce} ilustra los
resultados de evaluación.\newline
En cuanto a la Precisión(p), JoDroid presenta un porcentaje del 78,57\%,
mientras que el Prototipo presenta un porcentaje del  73,68\%.\newline
En cuanto a Recall(r), el Prototipo presenta un porcentaje del 100\%, frente a
un porcentaje del 78,57\% presentado por JoDroid.


\textit{Precisión y Recall excluyendo exceptions}\newline
Omitiendo los casos de prueba de GeneralJava\_Exceptions1 a
GeneralJava\_Exceptions4, el total de testcases(20) queda reducido a 16. De
estos, 11 presentan fugas de información.\newline
En lo que respecta a la métrica de Precisión, JoDroid
presenta un porcentaje del 78,57\%; frente al Prototipo que presenta un
porcentaje del 73,33\%.\newline 
Para la métrica de Recall, tanto JoiDroid como el Prototipo, presentan el mismo
porcentaje esto es 100\%.\newline



\subsection{Análisis de evaluación FlowDroid, JoDroid, Prototipo}
\label{subsec:fjp}
%las tablas confirman lo que ya se sabia segun la literatura existe.

En base a los resultados de evaluación para el conjunto de
evaluación(compuesto por 20 testcases, de los cuales 14 presentan leaks),
obtenidos en los anteriores apartados, 
% donde se presentó un análisis detallado para la evaluación entre FlowDroid vs
% Prototipo, y, JoDroid vs Prototipo; 
se comparan las tres herramientas(FlowDroid, JoDroid y Prototipo) frente a
Precisión, Recall, y la detección de fugas de información mediante flujos
implícitos. La tabla \ref{tb:porcentajes} ilustra todos los resultados y la
tabla \ref{tb:comparacion} ilustra los respectivos porcentajes.

\begin{table}[t!]
\begin{center}
\begin{tabular}{c|c|c|c|}
\cline{2-4}
& \cellcolor{gray!30}FlowDroid & \cellcolor{gray!30}JoDroid &
\cellcolor{gray!30}Prototipo \\
\cline{1-4}
\multicolumn{0}{ |c|  }{\multirow{0}{*}{FP} }  & 3 & 5 & 3\\ \cline{0-3}
\multicolumn{0}{ |c|  }{\multirow{0}{*}{FN} }  & 3 & 0 & 0\\ \cline{0-3}
\multicolumn{0}{ |c|  }{\multirow{0}{*}{TP} }  & 11 & 14 & 11\\\cline{0-3}
\multicolumn{0}{ |c|  }{\multirow{0}{*}{TN} }  & 3 & 1 &  3\\ \cline{0-3}
\end{tabular}
\end{center}
\caption{Resultados de precisión para FlowDroid y Prototipo. Resume el total de
falsos positivos(FP), verdaderos positivos(TP), verdaderos negativos(TN) y
falsos negativos(FN).}
\label{tb:porcentajes}
\end{table}

\begin{table}[t!]
\begin{center}
\begin{tabular}{cc|c|c|c}
\cline{2-4}
& \multicolumn{0}{ |c|  }{\multirow{1}{*}{\cellcolor{gray!30} FlowDroid} } &
\cellcolor{gray!30}JoDroid & \cellcolor{gray!30}Prototipo \\
\cline{1-4}
\multicolumn{0}{ |c|  }{\multirow{0}{*}{Precisión} }  & 78\% & 78,57\% & 73,68\%
\\
\cline{0-3}
\multicolumn{0}{ |c|  }{\multirow{0}{*}{Recall} }  & 78\% & 78,57\% &  100\%\\
\cline{0-3}
\multicolumn{0}{ |c|  }{\multirow{0}{*}{Detección Flujos Implícitos} }  & No &
Si & Si\\
\cline{0-3}
\end{tabular}
\end{center}
\caption{Comparación entre FlowDroid, JoDroid y Prototipo. Ilustra los
porcentajes para Precisión, Recall, y la detección de leaks mediante
flujos implícitos.\newline}
\label{tb:comparacion}
\end{table}

-\textit{Desempeño}\newline 
Como muestran las tablas \ref{tb:resultados} y \ref{tab:JoDroid-Prototipo}, el
Prototipo presenta mejor desempeño frente FlowDroid y JoDroid. En el caso de
FlowDroid, en promedio tarda 3,3 segundos más que el Prototipo para ejecutar el
análisis. En el caso de JoDroid, el tiempo de análisis es costoso en comparación
a las otras herramientas, puesto que su tiempo de ejecución oscila entre 21 y 22
minutos.\newline
% EXPLICAR POR QUE, DE ACUERDO A LA TECNICA DE ANALISIS?\newline
% Como muestran las tablas \ref{tb:resultados} y \ref{tab:JoDroid-Prototipo}, el
% análisis de flujo de información basado en lenguajes tipados de seguridad(en que
% se fundamenta la propuesta de análisis) presenta un mejor desempeño, frente a
% las técnicas en que se basan FlowDroid y JoDroid, análisis de flujo de datos
% mediante técnicas tainting y IFC(Information Flow Control) mediante PDG y
% slicing, respectivamente.

-\textit{Precisión y Recall}\newline
Tanto FlowDroid como JoDroid presentan mejor Precisión que el Prototipo, es
decir que el Prototipo presenta más falsos positivos(FP).\newline 
Por otro lado, el Prototipo presenta mayor Recall frente a FlowDroid y JoDroid,
por tanto, el Prototipo detecta mayor cantidad de fugas existentes (reporta
menos FN).
Para este caso particular, el Prototipo detecta todos los TP.\newline 
En consecuencia, es posible decir que aunque el Prototipo presenta mayor
cantidad de FP frente a FlowDroid y JoDroid, deja pasar menos fugas de
información.\newline
En lo que respecta a flujos implícitos, a diferencia de FlowDroid, tanto JoDroid
como el Prototipo detectan fugas de información a través de Flujos
implícitos.\newline

% EXPLICAR POR QUE DE ACUERDO A LAS TECNICAS DE ANALISIS?\newline
% El análisis de flujo de información mediante lenguajes tipados de
% seguridad(en que se basa el Prototipo), ofrece un mejor Recall que
% FlowDroid, sin embargo FlowDroid es más preciso. Esto se traduce en que la
% propuesta de análisis evaluada a través del Prototipo, presenta más falsos
% positivos que FlowDroid, pero no deja pasar fugas de información.\newline
% Por otro lado, el Prototipo detecta fugas de información presentes en Flujos
% implicitos, FlowDroid No.\newline
% El análisis de flujo de información mediante lenguajes tipados de seguridad,
% ofrece igual Recall que la técnica de PDG utilizada por JoDroid, sin
% embargo, JoDroid presenta mejor Precisión.\newline
Analizando los resultados para las métricas de desempeño, precisión y recall;
descritas anteriormente, acorde al tipo de análisis y técnicas en que se basa
cada herramienta, es posible anotar:\newline 
- Desempeño: la técnica de lenguajes tipados se seguridad en que se basa el
prototipo aprovecha las ventajas de optimización con que se construyen los
compiladores, permitiendo que el tiempo para ejecución del análisis sea
despreciable.\newline 
-Precisión, Recall y detección de Flujos implícitos: el análisis pesimista en
que se basa el Prototipo, donde se asume que todos los métodos implementados en
la aplicación seran invocados, hace que los resultados del análisis sean menos
precisos.
%(EL ANÁLISIS PESIMISTA O LA TECNICA DE LENGUAJES TIPADOS?)
Al basar su técnica de análisis en flujo de información, el Prototipo y JoDroid
ofrecen mayor Recall que FlowDroid, que se basa en flujo de datos.\newline
Una ventaja de las técnicas basadas en control de flujo de información es la
detección de fugas de información a través de flujos implícitos.

  

\subsection{Comparación técnicas de análisis evaluadas}
% En las subsecciones anteriores(4.2.1 a 4.2.3), se analizaron los resultados de
% evaluación con respecto a un conjunto de aplicaciones específico. En la presente
% sección, el análisis se basa en las herramientas previamente evaluadas, pero
% haciendo enfásis en las técnicas utilizadas por las mismas.
%tipo de analisis y técnica
FlowDroid se fundamenta en análisis de flujo de datos, mediante técnicas
tainting.\\
El código .dex a ser analizadado es transformado a una representación
intermedia(Jimple representation).\\
El análisis parte de la construcción de un super-grafo del programa que se
analiza, el super-grafo es una colección de grafos dirijidos, mediante los
cuales se representa el programa, donde los nodos asocian las sentencias del
programa y las aristas, la forma en que estas se conectan. Para recorrer el
super-grafo utiliza un algoritmo basado en el problema de
graph-reachability\cite{Graph-reachability}; cuyo costo computacional es de
orden polinomial O(ED3), donde E representa funciones de flujo de datos(dataflow
functions) y D conjunto de elementos para guiar el seguimiento de los
datos marcados(set of data flow facts).\newline
Para propagar la marca en los datos que análiza omite el control de flujo de
información, sólo se centra en el flujo de datos marcados como sources y
sinks.\newline
La herramienta recibe como entrada el apk del aplicativo, detecta
automáticamente los sources y sinks del programa mediante el uso de SuSi y
genera un reporte del análisis.

JoDroid se fundamenta en analísis de control de flujo de información, aplicando
técnicas de grafos de dependencia(PDG) y técnicas slicing.\newline 
El código .dex es transformado a código de representación intermedio(SSA-form).
Construye un grafo PDG, donde los nodos representan statements y expresiones, y
las aristas modelan las dependencias sintacticas entre los statements y
expresiones. Este PDG permite modelar flujos explícitos e implícitos.\newline
El costo computacional un análisis basado en PDG es de orden polinomial
O(N)3\cite[page 3]{FCO-PDG}.\newline 
Para hacer seguimiento al control de flujo de información, utiliza labels de
seguridad, estos califican con nivel de seguridad alto o bajo información de
variables y statements.\newline
Los procedimientos para usar la herramienta comprenden: generar el punto de
entrada del análisis, generar el PDG, ejecutar el respectivo análisis. Primero,
recibe como entrada el apk y manifest del aplicativo para generar un archivo con
el punto de entrada del análisis; luego, a partir del archivo devuelto
anteriormente genera el PDG, finalmente, recibe como entrada el PDG, lista los
statements y variables del aplicativo para que se indique manualmente los
sources y sinks, y genera el respectivo análisis.\newline

La propuesta está basada en análisis de flujo de información mediante lenguajes
tipados de seguridad, más específicamente mediante Jif.\newline
Para cada programa a analizar se debe implementar la versión Jif, es decir
el programa debe estar implementado acorde al sistema de anotaciones de Jif. A
partir de tales anotaciones el compilador verifica la generación de flujos de
información que incumplan la política de seguridad establecida, para reportalos
como flujos de información indebidos. 
Al ser evaluado directamente por un
compilador, obtiene los beneficios de bajo costo computacional del mismo.\newline
La generación del análisis para verificar la política de seguridad
definida, requiere dos pasos. Primero, se genera la versión Jif del aplicativo a
analizar usando el prototipo de anotación, este recibe como entrada el
código fuente del aplicativo a analizar. No requiere la espcificación de sources
y sinks.\\
Segundo, se compila el .jif, para obtener el reporte de análisis.






En el cuadro \ref{tab:comparacion} se resumen los puntos comparados
anteriormente.
\begin{table}[H]
\begin{center}
\small\addtolength{\tabcolsep}{-3pt}
\begin{tabular}{|p{2,2cm}|p{1,3cm}|p{5cm}|p{2cm}|p{2cm}|}
	\hline
	\textbf{Herramienta} & \textbf{Tipo} & \textbf{Técnicas} & \textbf{Costo
	computacional} & \textbf{ Entradas} \\
	\hline
	FlowDroid & Flujo de datos & 
	Tainting; super-grafo integrado por grafos dirigidos; Representación intermedia
	Jimple; algoritmo graph-reachability & Polinomial
	O(ED3)\cite{Graph-reachability} & apk\\
	\hline
	JoDroid & Flujo de información & PDG; slicing; Representación intermedia(SSA-
	form) & polinomial O(N)3\cite{FCO-PDG} & apk; Manifest; sources y sinks
	\\
	\hline
	Prototipo & Flujo de información  & Lenguajes tipados de seguridad; Type
	checking & Tiempo de compilación(Tiempo realmente bajo) & código fuente
	\\
	\hline
\end{tabular}
\end{center}
\caption{Generalidades técnicas de análisis evaluadas}
\label{tab:comparacion}
\end{table}	


\begin{table}[H]
\begin{center}
\small\addtolength{\tabcolsep}{-3pt}
\begin{tabular}{|p{2,3cm}|p{2cm}|p{2,5cm}|p{3cm}|p{3cm}|}
	\hline
	\textbf{Herramienta} & \textbf{ventajas } & \textbf{desventajas} 
	& \textbf{similitudes} & \textbf{diferencias}\\
	\hline
	Prototipo vs FlowDroid &  Mejor Recall.  &  Menor Precisión. &
	Detección automática de sources y sinks & Tipo de análisis(flujo de
	informacion; flujo de datos)\\
	 & Detección de flujos implícitos.  & No soporte para análisis interApp. & Bajo
	 costo en desempeño. & \\
	\hline
	Prototipo vs JoDroid & Mejor Recall. &  Menor Precisión. &  & \\ 
	 & Detección automática de sources y sinks. &  & No soporte para análisis
	 interApp &
	\\
	& Menor costo en desempeño & & Tipo de análisis IFC & Técnica de análisis: PDG,
	slicing\\
	\hline
\end{tabular}
\end{center}
\caption{Comparación técnicas de FlowDroid y JoDroid, frente a técnicas
del Prototipo.\newline Ventajas y desventajas comparando el prototipo(Propuesta
de solución)}
\label{tab:comparacion}
\end{table}	

\begin{table}[H]
\begin{center}
\small\addtolength{\tabcolsep}{-3pt}
%\begin{tabular}{|p{4cm}|p{1cm}|p{1cm}|}
\begin{tabular}{|p{4cm}|p{1cm}|p{0,5cm}|p{0,5cm}|p{0,5cm}|p{0,5cm}|p{0,5cm}|p{0,5cm}|p{0,5cm}|p{0,5cm}|p{0,5cm}|}
	\hline
	\textbf{Item} & \textbf{P vs F} & \textbf{P vs J}\\
	%\multirow{4}{*}\textbf{Item} & \textbf{P vs F} & \textbf{P vs J}\\
	\hline
	 & & \\
	\hline
	Mayor Precisión & & & & & & & & & &\\
	\hline
	Menor Precisión & & \\
	\hline
	Mayor Recall & & \\
	\hline
	Mayor Recall & & \\
	\hline
	Mayor costo en desempeño & & \\
	\hline
	Bajo costo en desempeño & & \\
	\hline
	Detección automática de sources y sinks & & \\
	\hline
	No soporte para Análisis interApp & & \\
	\hline
	Tipo de análisis(flujo de informacion; flujo de datos) & & \\
	\hline
	Tipo de análisis IFC & & \\
	\hline
	Técnica de análisis: PDG, slicing & & \\
	\hline
\end{tabular}
\end{center}
\caption{CoOMPARACION técnicas de FlowDroid y JoDroid, frente a técnicas
del Prototipo.\newline Ventajas y desventajas comparando el prototipo(Propuesta
de solución)}
\label{tab:comparacion}
\end{table}

\subsection{Qué tanto cambia la anotación del código original}
(\textbf{Martín: esta parte debería ir en esta sección, o en la
discusión?})\newline 
En el diseño de la solución se describieron los retos
técnicos\ref{sec:limitaciones} que implica anotar código Android, y cómo superar
algunos de estos. Específicamente, las limitaciones para las que se propone un
mecanismo que permita soportarlas son: Statement @Override; Casting entre tipos
EditText y View; Clase Nested R y Enhanced for loop.

% 
% Dentro de las limitaciones técnicas \ref{sec:limitaciones} descritas en el
% diseño de la solución, se describieron una serie de limitaciones para generar la
% versión Jif del programa Android a analizar. Para algunas de tales limitaciones
% se propuso un mécanismo que permita soportarlas, estas son:
% \begin{itemize}
%   \item Statement @Override
%   \item Casting entre tipos EditText y View
%   \item Clase Nested R 
%   \item Enhanced for loop
% \end{itemize}
Adicionalmente, se describió que el compilador de Jif exige la declaración del
chequeo de excepciones tipo runtime, en programas que lo
requieran\ref{subsec:consVerPol}. Ahora, cumplir con los requisitos del
compilador de Jif y aplicar mecanismos que soporten tales limitaciones, implica
una serie de transformaciones en el código fuente del programa Android a
analizar, tanto en la etapa previa a la anotación como en la anotación
misma.\newline
% para soportar tales limitaciones y
% cumplir con los requisitos del compilador de Jif, el código fuente del programa
% Android a analizar, pasa por una serie de transformaciones tanto en la etapa
% previa a la anotación como en la anotación misma.\newline
Prevía a la anotación el desarrollador debe garantizar dos cosas, primero debe
adicionar las runtime exceptions que su programa necesite, segundo cuando
requiera el uso de bucles for, debe usar la versión sencilla y no la versión
mejorada del mismo(Enhanced for loop).\newline 
Durante la anotación, además de aplicar los criterios de anotación definidos en
el diseño de la solución \ref{subsec:pasosSol}, se aplican los mecanismos
propuestos para soportar el Statement @Override, Casting entre tipos EditText y
View; y Clase Nested R. Así, en el caso de  la sentencia Statement @Override, se
comenta la línea que lo contenga; para el casting entre tipos EditText y View,
la información implicada en este tipo de casting es abstraída mediante un tipo
de dato String; finalmente para la clase nested R se crea una clase que define
los define los recursos utilizados por la aplicación a través de
variables.\newline 
Si bien, la idea que fundamenta el diseño de la solución consiste en generar la
versión Jif del aplicativo Android a analizar, lo cual se traduce en adicionar
las anotaciones correspondientes para evaluar determinada política de seguridad,
de modo que el compilador de Jif entienda el programa y permita analizar el
flujo de información en el mismo; no es suficiente con sólo anotar el código, en
otras palabras, sin los ajustes previamente mencionados, el compilador genera
error. Por otro lado, lo positivo es que tales ajustes no alteran la lógica del
programa.
