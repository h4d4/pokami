\label{ch:evaluacion}
\chapter{Evaluación}
% Ventajas y limitaciones de la solución.\newline 
% Si aplica, evaluación de desempeño.  \newline 
% Si aplica, evaluación de usabilidad.  
% Hay otras soluciones similares? \newline 
% Cuáles son las diferencias y las ventajas y desventajas con respecto a esas soluciones.

\section{Consideraciones de evaluación}
El flujo de información es analizado al interior de una sola aplicación, no se
consideran flujos de información vía interApp, es decir, la activación de
componentes de aplicaciones externas vía intents.

Cabe anotar que los resultados de evaluación que se presentan a continuación son
los obtenidos en el siguiente ambiente de pruebas: una máquina virtual con
sistema operativo GNU/Linux; 2,7 GHz de procesador y 1GB de Memoria RAM. En esa
misma máquina, se tiene instalada la versión 3.4.2 del compilador de Jif;
el artefacto de prueba para FlowDroid y JoDroid, obtenidos desde,
\cite{FlowDroid-artifact} y \cite{joDroidManual}, respectivamente.



\section{Conjunto de evaluación}
\label{sec:evalSet}

\begin{table}[t]
\small\addtolength{\tabcolsep}{-3pt}
\begin{tabular}{|p{13cm}|p{1cm}|}
	\hline
	\multicolumn{2}{|>{\columncolor[gray]{0.8}}c|}{\textbf{AndroidSpecific\_DirectLeak1}}\\
	\hline
	\textbf{Descripción} & \textbf{Leaks}\\
	\hline
	La variable \textit{mrg} tiene un nivel de seguridad alto,
	almacena información retornada por el método source \textit{getDeviceId}. Se
	genera flujo de información directo entre información con nivel de seguridad alto e
	información con nivel de seguridad bajo, al enviar como parámetro del método
	\textit{sendTextMessage}, información de la variable \textit{mrg}. & 1 \\
	\hline
	\multicolumn{2}{|>{\columncolor[gray]{0.8}}c|}{\textbf{AndroidSpecific\_LogNoLeak}}\\
	\hline
	\textbf{Descripción} & \textbf{Leaks}\\
	\hline
	El caso de prueba no presenta información con niveles de seguridad alto. Se
	presentan flujos de información entre información con el mismo nivel de
	seguridad, en este caso bajo, lo cual es permitido. & 0 \\
	\hline
	\multicolumn{2}{|>{\columncolor[gray]{0.8}}c|}{\textbf{ArraysAndLists\_ArrayAccess1}}\\
	\hline
	\textbf{Descripción} & \textbf{Leaks}\\
	\hline
	Se tiene un array en que se almacena información tanto proveniente como no
	proveniente de sources, parte de la información que almacena es enviada como
	parámetro del método \textit{sendTextMessage}. \textit{Observación:}
	Para la técnica de análisis de FlowDroid(taint analysis), se marca únicamente el
	índice del array donde se almacena el dato considerado como source, así,
	cuando se envía como parámetro del método \textit{sendTextMessage},
	el dato de un índice no marcado, no se genera leak. Para nuestra técnica
	de análisis(flujo de información mediante JIF), para que un array almacene
	información con nivel de seguridad alto, primero debe ser catalogo(anotado)
	con nivel de seguridad alto, lo que implica que el array podrá almacenar
	información tanto de nivel de seguridad alto como bajo, pero toda la
	información quedará con nivel de seguridad alto. En consecuencia, al enviar
	cualquier índice del array como parámetro del método
	\textit{sendTextMessage} se presenta un flujo de información no
	permitido. & 0
	\\
	\hline
% 	\multicolumn{2}{|>{\columncolor[gray]{0.8}}c|}{\textbf{GeneralJava\_Exceptions2}}\\
% 	\hline
% 	\textbf{Descripción} & \textbf{Leaks}\\
% 	\hline
% 	La variable \textit{imei} es de nivel de seguridad alto, almacena información
% 	devuelta por el método \textit{getDeviceId}. El control de flujo del
% 	programa conduce de manera implícita a la captura de una excepción tipo
% 	RuntimeException, desde allí se utiliza información proveída por la variable
% 	\textit{imei}, como parámetro para invocar el método \textit{sendTextMessage}.
% 	Generando un flujo de información indebido. & 1
% 	\\
% 	\hline
	\multicolumn{2}{|>{\columncolor[gray]{0.8}}c|}{\textbf{ImplicitFlows\_ImplicitFlow2}}\\
	\hline
	\textbf{Descripción} & \textbf{Leaks}\\
	\hline
	 La variable \textit{userInputPassword} con nivel de seguridad alto, almacena
	 información de un campo EditText tipo textPassword(password suministrado por
	 el usuario). Se generan flujos de información indebidos: al tratar de asignar
	 información a la variable passwordCorrect con nivel de seguridad bajo, a
	 partir de la comparación de información con nivel de seguridad alto(variable
	 textPassword), después, al tratar de mostrar en el \textit{log} información
	 que depende de tal comparación. & 1\\
	\hline
\end{tabular}
\caption{Aplicaciones de prueba.\newline
Describe parte del conjunto de aplicaciones de prueba. 
% Se considera con nivel de seguridad alto, variables y métodos que almacenan y modifican(respectivamente), información catalogada como privada(Sources).\newline
% Se considera con nivel de seguridad bajo, canales para envío de mensajes,
% muestra de logs y canales creados durante el control de flujo del programa.\newline 
}
\label{tab:descripApps0}
\end{table}

Para la evaluación se parte de DroidBech versión 1.0\cite{DroidBenchBenchmarks},
% benchmark integrado por 39 casos de prueba para aplicaciones Android, 
benchmark integrado por casos de prueba para aplicaciones Android, cuyos
autores son los mismos creadores de FlowDroid. Se opta por
utilizar DroidBench puesto que, en la literatura científica consultada al respecto, no se encuentran
otros benchmarks diseñados específicamente para evaluar aplicaciones Android.\newline 
De DroidBech se toma un grupo de testcases evaluables frente a la política de
seguridad establecida \ref{subsection:politica}, este grupo está integrado por
20 testcases. De los cuales, 14 presentan fugas de información.

La tabla \ref{tab:descripApps0} describe parte del grupo de testcases a
evaluar. En los casos en que se requiere, se precisan observaciones entre los
resultados de evaluación esperados para la técnica de análisis utilizada por
FlowDroid(análisis de flujo de datos) y la técnica de análisis propuesta en el
presente trabajo(análisis de flujo de información).
En la sección \ref{sec:testcases} de los anexos, se encuentra la
descripción del grupo de prueba completo.

El conjunto de prueba es analizado con FlowDroid, JoDroid y con el
Prototipo.\newline 
Los fundamentos\cite{Precision-Recall} para calificar los resultados del
análisis son los siguientes:\newline 
True Positive(TP) y False Positive(FP), para referenciar el número de Casos
Positivos esperados que son correcta o incorrectamente identificados.\newline 
True Negatives(TN) y False Negatives(FN), para referenciar el número de 
Casos Negativos esperados que son correcta o incorrectamente
identificados.\newline
Ahora, para el contexto del presente análisis, donde se evalúa la detección de
fugas de información, los resultados del análisis reportados por cada
herramienta son calificados como:\newline 
True Positive(TP): cuando se reporta un leak que efectivamente existe.\newline
False Positive(FP): cuando se reporta un leak que no existe.\newline 
True Negative(TN): cuando no se reporta leak y efectivamente no existe.\newline
False Negative(FN): cuando no se reporta un leak existente. 

Una vez se tienen los resultados de análisis reportados por cada herramienta, se
calcula la Precisión y el Recall para cada una de ellas.

La \textbf{Precisión} hace referencia a los Casos Positivos esperados(correctos
e incorrectos: TP, FP), en contraste con la proporción de verdaderos Positivos(TP)
detectados\cite{Precision-Recall}. Una alta Precisión indica que la herramienta
reporta más correctos Positivos(TP) que incorrectos Positivos(FP). 

El \textbf{Recall} indica la proporción de Casos Positivos detectados(TP),
frente a los Casos Positivos esperados como correctos\cite{Precision-Recall}. Un
alto Recall indica que la herramienta reporta más correctos Positivos(TP) que incorrectos
Negativos(FN). Es decir, la herramienta deja pasar menos errores.\newline
% La Precisión(\ref{pre}) mide la cantidad de respuestas válidas(TP), frente al
% total de respuestas, esto es, respuestas válidas(TP) y respuestas no válidas(FP).
% A mayor Precisión, la herramienta reporta más verdaderos positivos(TP) y menos
% falsos positivos(FP). En otras palabras, de los leaks existentes, la
% herramienta reporta una mayor cantidad.
% 
% Con el Recall(\ref{rec}) se mide la cantidad de respuestas válidas(TP), frente
% al total de respuestas válidas(TP) y respuestas válidas no detectadas(FN). A
% mayor Recall, menor reporte de falsos negativos.\newline

Las fórmulas para calcular Precisión(p) y Recall(r), son:
\begin{equation}
\label{pre}
	p = TP/(TP +FP) 
\end{equation}
\begin{equation}
\label{rec}
	r = TP/(TP+FN)
\end{equation}
Donde: \newline
TP representa el total de verdaderos positivos; FP corresponde al total de
falsos positivos y  FN representa el total de falsos
negativos; reportados por la herramienta.

% Las formulas \ref{p} y \ref{r}, permiten el calculo de ambas métricas
% de seguridad.\newline
Además de las fórmulas anteriormente descritas, se utiliza el comando
\textit{time}\cite{time-man} de unix, para medir el desempeño de cada
herramienta.

\section{Evaluación FlowDroid y Prototipo } 
\label{subsec:fvsp}
Del total de testcases(20), 14 presentan fugas de información mientras que 6 de
ellos no. Los resultados de evaluación para FlowDroid y el Prototipo, son
presentados en la tabla \ref{tb:resultados}. En esta, por cada caso de prueba
se indica la cantidad de leaks que presenta, el resultado devuelto por la
herramienta y el tiempo que tarda el análisis.

\begin{table}[H]
\begin{center}
\small\addtolength{\tabcolsep}{-3pt}
\begin{tabular}{|p{1cm}|p{6cm}|p{1cm}|p{1cm}|p{1cm}|p{1cm}|p{1cm}|}
	\hline
	\textbf{Item} & \textbf{Testcase} & \textbf{Leaks} & \textbf{F} &
	\textbf{P} & \textbf{ tF} & 
	\textbf{tP}\\
	\hline
	1 & AndroidSpecific\_DirectLeak1 & 1 & TP & TP &5.371s &2.063s\\
	\hline
	2 & AndroidSpecific\_InactiveActivity & 0 & TN & FP  &3.255s &2.469s\\
	\hline
	3 & AndroidSpecific\_LogNoLeak & 0 & TN & TN &5.505s &2.946s\\
	\hline
	4 & AndroidSpecific\_Obfuscation1 & 1 & TP & TP &6.734s &2.706s\\
	\hline
	5 & AndroidSpecific\_PrivateDataLeak2 & 1 & TP & TP & 6.144s &2.644s\\
	\hline
	6 & ArraysAndLists\_ArrayAccess1 & 0 & FP & FP & 4.708s & 1.278s\\
	\hline
	7 & ArraysAndLists\_ArrayAccess2 & 0 & FP & FP & 4.4s &1.361s\\
	 \hline
	8 & GeneralJava\_Exceptions1 & 1 & TP & TP &6.397s &2.755s\\
	\hline
	9 &  GeneralJava\_Exceptions2 & 1 & TP & TP &5.887s &1.980s\\
	\hline
	10 & GeneralJava\_Exceptions3 & 0 & FP & FP &6.008s &2.032s\\
	\hline
	11 & GeneralJava\_Exceptions4 & 1 & TP & TP &5.731s &2.313s\\
	\hline
	12 & GeneralJava\_Loop1 & 1 & TP & TP &5.605s &2.800s\\
	\hline
	13 & GeneralJava\_Loop2 & 1 & TP & TP &4.719s &1.361s\\
	\hline
	14 & GeneralJava\_UnreachableCode & 0 & TN & FP &3.792s &1.197s\\
	\hline
	15 & ImplicitFlows\_ImplicitFlow1 & 1 & FN & TP &4.853s &1.331s\\
	\hline
	16 & ImplicitFlows\_ImplicitFlow2 & 1 & FN & TP &4.496s &1.212s\\
	\hline
	17 & ImplicitFlows\_ImplicitFlow4 & 1 & FN & TP &4.375s &1.224s\\
	\hline
	18 & Lifecycle\_ActivityLifecycle3 & 1 & TP & TP &4.792s &1.222s\\
	\hline
	19 & Lifecycle\_BroadcastReceiverLifecycle1 & 1 & TP & TP &4.456s &1.061s\\
	\hline
	20 & Lifecycle\_ServiceLifecycle1 & 1 & TP & TP &5.225s &1.180s\\
	\hline
\end{tabular}
\end{center}
\caption{Resultados de evaluación para FlowDroid y Prototipo. Donde
\textit{Item} indica el testcase que se evalúa, \textit{Testcase} especifica el
nombre de la aplicación para el caso de prueba; \textit{Leaks} indica si el
testcase presenta fugas de información; \textit{F} y  \textit{P} muestran los
resultados devueltos por FlowDroid y por el Prototipo; \textit{tF} y
\textit{tP}, señalan el tiempo(en segundos) que toma el análisis para Flowdroid
y para el Prototipo, respectivamente.}
\label{tb:resultados}
\end{table}

\subsection{Análisis de evaluación entre FlowDroid y Prototipo}
\subsubsection{Resultados de desempeño}
% -\textit{Resultados de desempeño}\newline
Acorde a los tiempos señalados en los campos tF y tP de la tabla
\ref{tb:resultados}, en promedio, FlowDroid tarda 3,3 segundos más que el
Prototipo para ejecutar el análisis.

% PULIR IDEAAAAA***\newline
% Tal diferencia de tiempo va en concordancia con los costos de ejecución que
% implica una u otra técnica de análisis, mientras el análisis de flujo de
% información basado en lenguajes tipados de seguridad(que es lo que se evalua
% mediante el prototipo) sólo requiere llegar hasta el chequeo de tipos, la
% técnica de analisis de flujo de datos en que se basa FlowDroid utiliza
% algoritmos de orden O(ED)3 , cuya complejidad es de orden polinomial,
% O(ED)3\cite[page 3]{FCO-PDG}.\newline
% Se podría destacar como positivo que el análisis de flujo de información
% mediante técnicas de tipado de seguridad, requiere menos tiempo que la técnica
% de marcado de datos utilizada por FlowDroid.x
\subsubsection{Falsos positivos, precisiones}
% -\textit{Resultados de precisión}\newline
En lo que respecta a los resultados del Prototipo, los FP correspondientes a los
items 2 y 14 de la tabla \ref{tb:resultados}, surgen como consecuencia de un
diseño de análisis pesimista (evaluación del flujo de información
\ref{subsec:consVerPol}), donde se asume que todos los métodos definidos en
la clase serán invocados. Por consiguiente, todos los métodos son incluidos para
el análisis del flujo de información. Así, si los métodos conllevan a flujos de
datos indebidos, independientemente de si son invocados o no, son considerados
como generadores de fugas de información.\newline 
% En lo que respecta a los resultados del Prototipo, los FP correspondientes a los
% items 2(AndroidSpecific\_InactiveActivity) y 14(GeneralJava\_UnreachableCode) de
% la tabla \ref{tb:resultados}, los cuales presentan flujos de información
% indebidos, pero en realidad nunca tendran lugar, porque se presentan en: una
% actividad que no está activada en el Manifest, y un método que nunca es llamado.
% Dan lugar al reporte de falsos positivos por parte del Prototipo, puesto que
% al basarse en un análisis pesimista (evaluación del flujo de información
% \ref{subsec:consVerPol}), donde se asume que todos los métodos definidos en la
% clase serán invocados. Por consiguiente, todos los métodos son incluidos para el
% análisis del flujo de información. Así, si los métodos conllevan a flujos
% de datos indebidos, independientemente de si son invocados o no, son
% considerados como generadores de fugas de información.\newline 
Por otro lado, en el caso de ArraysAndLists: items 6 y 7, de la tabla
\ref{tb:resultados}, no es sencillo calificar los resultados como FP, puesto
que, para lo que está analizando FlowDroid(verificar que su técnica de análisis
diferencie entre los elementos marcados y no marcados de un array),
efectivamente se presentan FP, sin embargo, para la forma en que se deben
implementar los programas en Jif, donde se suele definir un nivel de seguridad
para todo el array antes de almacenar los elementos en el mismo, podría decirse
que no se trata de un FP, porque se revelo información que había sido catalogada
con nivel de seguridad alto.

\subsubsection{Detección de flujos implícitos}
A diferencia de FlowDroid, el Prototipo detecta fugas de información través de
flujos implícitos. La no detección de Flujos implícitos por parte de FlowDroid,
responde al tipo de análisis y las técnicas en que se fundamenta la herramienta:
análisis de flujo de datos mediante técnicas tainting. Basada en 
%Puesto que, 
% 
% \newline
% --el Prototipo, la anotación generada por el prototipo, o la técnica de análisis
% en que se basa el diseño de la solución: lenguajes tipados de seguridad???
% 
% -\textit{Acerca de por qué FlowDroid no detecta flujos implícitos}\newline
%el análisis de FlowDroid utiliza  técnicas DataFlow, específicamente, utiliza
% tainting análisis.
% Para hacer seguimiento al flujo de información de un
% programa, la técnica de análisis tainting se basa en: 
asociar una o más marcas
con el valor de los datos en el programa, y en propagarlas. Dependiendo de los
criterios definidos para el análisis, la marca puede ser propagada a causa de
flujos explícitos o de flujos implícitos\cite{taint-analysis}, o a causa de
ambos. En flujos explícitos la propagación ocurre cuando el valor de una variable marcada está
implicada en el calculo de otra variable. En flujos implícitos la propagación
tiene lugar a través de dependencias en el control de flujo del programa, por
ejemplo, cuando el valor de un dato marcado afecta indirectamente otra variable.\newline 
En el caso de FlowDroid, los criterios que fundamentan el análisis de la
herramienta, hacen que el marcado de datos se propague para flujos explícitos y
y no para flujos implícitos. Por consiguiente, FlowDroid no detecta flujos
implícitos.\newline
Con esto presente, se definen dos escenarios de análisis: Precisión y Recall,
incluyendo flujos implícitos; y  Precisión y Recall, excluyendo flujos
implícitos, donde se incluyen u omiten los testcases para flujos
implícitos(items 15 a 17, tabla \ref{tb:resultados}).\newline
La tabla \ref{tb:FlowDroidPrototipoFI} muestra los resultados de evaluación para
ambos casos.\newline
%En el caso de TaintDroid, tampoco detecta flujos implícitos, porque entre las
% desiciones de diseño de la herramienta está enfocarse en el seguimeinto al
% flujo de datos y no al flujo de control, puesto que si incluyen seguimiento al
% flujo de control, se adiciona sobrecarga a la herramienta, la cual es de tipo
% dinámico.
\begin{table}[H]
\begin{center}
\begin{tabular}{c|c|c|c|c|}
\cline{2-5}
& \multicolumn{2}{>{\columncolor{gray!30}} c  }{\multirow{1}{*}{Con FI}}
& \multicolumn{2}{|c|}{\multirow{1}{*}{\cellcolor{gray!50} {Sin FI}}}\\
\cline{2-5}
& \cellcolor{gray!30}FlowDroid & \cellcolor{gray!30}Prototipo &
\cellcolor{gray!50}FlowDroid & \cellcolor{gray!50}Prototipo \\
\cline{1-5}
\multicolumn{0}{ |c|  }{\multirow{0}{*}{FP} }  & 3 & 5 & 3 & 5\\ \cline{0-4}
\multicolumn{0}{ |c|  }{\multirow{0}{*}{FN} }  & 3 & 0 & 0 & 0\\ \cline{0-4}
\multicolumn{0}{ |c|  }{\multirow{0}{*}{TP} }  & 11 & 14 & 11 & 11 \\
\cline{0-4}
\multicolumn{0}{ |c|  }{\multirow{0}{*}{TN} }  & 3 & 1 &  3 & 1\\ \cline{0-4}
\end{tabular}
\end{center}
\caption{Resultados de precisión para FlowDroid y Prototipo, de acuerdo al
escenario, incluyendo o excluyendo flujos implícitos(FI). Resume el total de
falsos positivos(FP), verdaderos positivos(TP), verdaderos negativos(TN) y falsos
negativos(FN); obtenidos tanto con FlowDroid como con el Prototipo.}
\label{tb:FlowDroidPrototipoFI}
\end{table}


\subsubsection{Precisión y Recall, incluyendo flujos implícitos}
% \textit{Precisión y Recall, incluyendo flujos implícitos}\newline
Los resultados obtenidos en la tabla \ref{tb:FlowDroidPrototipoFI} señalan que
de las 14 fugas existentes, el Prototipo las detecta todas, presenta 14
TP(verdaderos positivos); mientras que, FlowDroid deja pasar 3.\newline
Por otro lado, el Prototipo presenta más falsos positivos que FlowDroid, de los
6 testcases que no presentan leaks, el prototipo reporta 5 como si fuesen
fugas, mientras que FlowDroid reporta 3.\newline
Así, en lo que respecta a Precisión, FlowDroid presenta un porcentaje del
78,57\%, siendo más preciso frente al Prototipo, que presenta un porcentaje del
73,68\%.\newline 
Por otro lado, el Prototipo presenta un porcentaje en Recall del 100\%,
mientras que FlowDroid presenta un porcentaje del 78,75\%.

\subsubsection{Precisión y Recall, excluyendo flujos implícitos}
% \textit{Precisión y Recall, excluyendo flujos implícitos}\newline
Excluyendo los testcases para flujos implícitos, el conjunto de prueba se reduce
a 17 casos. De los cuales, 11 presentan leaks.\newline 
En este contexto, el porcentaje de Precisión para FlowDroid es de 78,57\%,
mientras que para el Prototipo es del 68,75\%. El porcentaje de Recall es igual
para FlowDroid y el Prototipo: 100\%.


\section{Evaluación JoDroid y Prototipo}
% Del conjunto de casos de prueba JoDroid ignora las excepciones, ya que su actual
% implementación no soporta análisis del flujo de información a través de
% tales sentencias[pag 93 \cite{JoDroid-Thesis}]. Por tanto, los testcases
% GeneralJava\_Exceptions1 a GeneralJava\_Exceptions4, no son evaluados. 
% No obstante, para calcular la precisión y el recall se consideran dos
% escenarios, uno en que se incluyen y otra en que no.
La tabla \ref{tab:JoDroid-Prototipo} ilustra los resultados de evaluación para JoDroid y el
Prototipo, en base a los cuales, se calculan las métricas de precisión y recall. 
\label{subsec:jvsp}
\begin{table}[t]
\begin{center}
\small\addtolength{\tabcolsep}{-3pt}
\begin{tabular}{|p{0.8cm}|p{6cm}|p{1cm}|p{0.8cm}|p{0.8cm}|p{1,8cm}|p{1cm}|}
	\hline
	\textbf{Item} & \textbf{Testcase} & \textbf{Leaks} & \textbf{J} &
	\textbf{P} & \textbf{ tJ} & \textbf{tP}\\
	\hline
	1 & AndroidSpecific\_DirectLeak1 & 1 & TP & TP & 22m11.991s &2.063s\\
	\hline
	2 & AndroidSpecific\_InactiveActivity & 0 & FP & FP  & 22m25.617s &2.469s\\
	\hline
	3 & AndroidSpecific\_LogNoLeak & 0 & TN & TN & 21m6.548s &2.946s\\
	\hline
	4 & AndroidSpecific\_Obfuscation1 & 1 & TP & TP &22m46.541s&2.706s\\
	\hline
	5 & AndroidSpecific\_PrivateDataLeak2 & 1 & TP & TP &21m32.447s&2.644s\\
	\hline
	6 & ArraysAndLists\_ArrayAccess1 & 0 & FP & FP &22m01.926s& 1.278s\\
	\hline
	7 & ArraysAndLists\_ArrayAccess2 & 0 & FP & FP &22m11.023s&1.361s\\
	 \hline
	8 & GeneralJava\_Exceptions1 & 1 & FN & TP & 22m52.134s &2.755s\\
	\hline
	9 & GeneralJava\_Exceptions2 & 1 & FN & TP & 21m4.434s&1.980s\\
	\hline
	10 & GeneralJava\_Exceptions3 & 0 & TN\tablefootnote{Al igual que en
	el resto de testcases para GeneralJava\_Exceptions(items 8, 9 y 11), la
	herramienta no detecta leaks, la diferencia para el presente caso, es que efectivamente no existe leak. Por tanto se califica como TN.} & FP & 21m37.040s &2.032s\\
	\hline
% 	\hline
% 	GeneralJava\_Exceptions3 & 0 & TN & FP & 21m37.040s &2.032s\\
% 	\hline
	11 & GeneralJava\_Exceptions4 & 1 & FN  & TP & 21m10.240s &2.313s\\
	\hline
	12 & GeneralJava\_Loop1 & 1 & TP & TP &21m15.30s&2.800s\\
	\hline
	13 & GeneralJava\_Loop2 & 1 & TP & TP &21m41.224s&1.361s\\
	\hline
	14 & GeneralJava\_UnreachableCode & 0 & TN & FP &22m84.138s&1.197s\\
	\hline
	15 & ImplicitFlows\_ImplicitFlow1 & 1 & TP & TP &22m55.645s&1.331s\\
	\hline
	16 & ImplicitFlows\_ImplicitFlow2 & 1 & TP & TP &22m32.231s&1.212s\\
	\hline
	17 & ImplicitFlows\_ImplicitFlow4 & 1 & TP & TP &22m43.110s&1.224s\\
	\hline
	18 & Lifecycle\_ActivityLifecycle3 & 1 & TP & TP &22m54.651s&1.222s\\
	\hline
	19 & Lifecycle\_BroadcastReceiverLifecycle1 & 1 & TP & TP &22m42.347s&1.061s\\
	\hline
	20 & Lifecycle\_ServiceLifecycle1 & 1 & TP & TP &22m92.722s&1.180s\\
	\hline
\end{tabular}
\end{center}
\caption{Resultados de evaluación para JoDroid y Prototipo. Donde
\textit{Testcase} especifica el nombre de la aplicación que se está evaluando;
\textit{Leaks} indica si el testcase presenta fugas de información; \textit{J} y
\textit{P} muestran los resultados devueltos por JoDroid y por el Prototipo;
\textit{tJ} y \textit{tP}, señalan el tiempo que toma el análisis para JoDroid
y para el Prototipo, respectivamente.}
\label{tab:JoDroid-Prototipo}
\end{table}

\subsection{Análisis de evaluación entre JoDroid y Prototipo}
\subsubsection{Resultados de desempeño}
% -\textit{Resultados de desempeño}\newline
Para el análisis mediante JoDroid se deben seguir una serie de pasos, tal como
se describen en el manual de referencia\cite{joDroidManual}, de estos,
únicamente se toma el tiempo correspondiente al paso para generar el
grafo de dependencia del programa(PDG), del cual parte el análisis. En general,
el tiempo que tarda la generación del PDG para cada aplicación analizada, oscila
entre 21 y 22 minutos. Cabe anotar que estos valores podrían cambiar con otras
características de hardware, sin embargo, asignando 1 GB de Ram a la máquina
virtual de Java, para la generación del PDG, es ese el rango de tiempo obtenido.
En consecuencia, para los valores de tiempo que señala la presente evaluación,
es posible decir que la herramienta es costosa en desempeño.

\subsubsection{Detección de Flujos implícitos}
% -\textit{Resultados de Precisión}\newline
La tabla \ref{tab:JoDroid-Prototipo} muestra que al igual que el Prototipo,
JoDroid detecta fugas de información a través de flujos implícitos. Ya que,
en los testcases correspondientes a flujos implícitos, items 15, 16 y 17, ambas
herramientas detectan que efectivamente, se presentan fugas de
información.\newline

\subsubsection{Fugas a través de excepciones}
Es importante resaltar que del conjunto de casos de prueba, JoDroid ignora el
control de flujo de información para excepciones(items 8 a 11 de la tabla
\ref{tab:JoDroid-Prototipo}), puesto que, su actual implementación no soporta
análisis del flujo de información a través de tales sentencias[pag 93
\cite{JoDroid-Thesis}].
En la tabla \ref{tb:jodroidP2exce} se muestran dos escenarios para los
resultados de evaluación: Con  excepciones y Sin excepciones. 
Con base en tales resultados se calcula la Precisión y Recall para cada uno de
los escenarios.

\begin{table}[t]
\begin{center}
\begin{tabular}{c|c|c|c|c|}
\cline{2-5}
& \multicolumn{2}{>{\columncolor{gray!30}} c  }{\multirow{1}{*}{Con
excepciones}} & \multicolumn{2}{|c|}{\multirow{1}{*}{\cellcolor{gray!50} {Sin
excepciones}}}\\
\cline{2-5}
& \cellcolor{gray!30}JoDroid & \cellcolor{gray!30}Prototipo &
\cellcolor{gray!50}JoDroid & \cellcolor{gray!50}Prototipo \\
\cline{1-5}
\multicolumn{0}{ |c|  }{\multirow{0}{*}{FP} }  & 3 & 5 & 3 & 4\\ \cline{0-4}
\multicolumn{0}{ |c|  }{\multirow{0}{*}{FN} }  & 3 & 0 & 0 & 0\\ \cline{0-4}
\multicolumn{0}{ |c|  }{\multirow{0}{*}{TP} }  & 11 & 14 & 11 & 11 \\
\cline{0-4}
\multicolumn{0}{ |c|  }{\multirow{0}{*}{TN} }  & 3 & 1 &  2 & 1\\ \cline{0-4}
\end{tabular}
\end{center}
\caption{Resultados de precisión para JoDroid y Prototipo. Muestra los
escenarios en que mide. Resume el total de falsos positivos(FP), verdaderos
positivos(TP), verdaderos negativos(TN) y falsos negativos(FN); obtenidos tanto
con JoDroid como con el Prototipo.}
\label{tb:jodroidP2exce}
\end{table}

\subsubsection{Precisión y Recall incluyendo excepciones}
% \textit{Precisión y Recall incluyendo excepciones}\newline
Del total de testcases(20), 14 presentan fugas de información. 
De los casos con fuga de información, 3 corresponden a las excepciones
incluidas(items 8 a 11, tabla \ref{tab:JoDroid-Prototipo}), y se califican como
falsos negativos(FN) porque JoDroid no los detecta. La tabla
\ref{tb:jodroidP2exce} ilustra los resultados de evaluación.\newline 
En cuanto a la Precisión(p), JoDroid presenta un porcentaje del 78,57\%,
mientras que el Prototipo presenta un porcentaje del  73,68\%.\newline
En cuanto a Recall(r), el Prototipo presenta un porcentaje del 100\%, frente a
un porcentaje del 78,57\% presentado por JoDroid.

\subsubsection{Precisión y Recall excluyendo excepciones}
% \textit{Precisión y Recall excluyendo exceptions}\newline
Omitiendo los testcases para excepciones(items 8 a 11, tabla
\ref{tab:JoDroid-Prototipo}), el total de testcases(20) queda reducido a 16. 
De estos, 11 presentan fugas de información.\newline 
En lo que respecta a la métrica de Precisión, JoDroid presenta un porcentaje del
78,57\%; frente al Prototipo que presenta un porcentaje del 73,33\%.\newline 
Para la métrica de Recall, tanto JoiDroid como el Prototipo, presentan el mismo
porcentaje esto es 100\%.\newline



\section{Análisis de evaluación FlowDroid, JoDroid, Prototipo}
\label{subsec:fjp}
%las tablas confirman lo que ya se sabia segun la literatura existe.

\begin{table}[h]
\begin{center}
\begin{tabular}{c|c|c|c|}
\cline{2-4}
& \cellcolor{gray!30}FlowDroid & \cellcolor{gray!30}JoDroid &
\cellcolor{gray!30}Prototipo \\
\cline{1-4}
\multicolumn{0}{ |c|  }{\multirow{0}{*}{FP} }  & 3 & 3 & 5\\ \cline{0-3}
\multicolumn{0}{ |c|  }{\multirow{0}{*}{FN} }  & 3 & 3 & 0\\ \cline{0-3}
\multicolumn{0}{ |c|  }{\multirow{0}{*}{TP} }  & 11 & 11 & 14\\\cline{0-3}
\multicolumn{0}{ |c|  }{\multirow{0}{*}{TN} }  & 3 & 3 &  1\\ \cline{0-3}
\end{tabular}
\end{center}
\caption{Resultados de precisión para FlowDroid y Prototipo. Resume el total de
falsos positivos(FP), verdaderos positivos(TP), verdaderos negativos(TN) y
falsos negativos(FN).}
\label{tb:porcentajes}
\end{table}

\begin{table}[h]
\begin{center}
\begin{tabular}{cc|c|c|c}
\cline{2-4}
& \multicolumn{0}{ |c|  }{\multirow{1}{*}{\cellcolor{gray!30} FlowDroid} } &
\cellcolor{gray!30}JoDroid & \cellcolor{gray!30}Prototipo \\
\cline{1-4}
\multicolumn{0}{ |c|  }{\multirow{0}{*}{Precisión} }  & 78,57\% & 78,57\% & 73,68\%
\\
\cline{0-3}
\multicolumn{0}{ |c|  }{\multirow{0}{*}{Recall} }  & 78,57 & 78,57\% &  100\%\\
\cline{0-3}
\multicolumn{0}{ |c|  }{\multirow{0}{*}{Detección Flujos Implícitos} }  & No &
Si & Si\\
\cline{0-3}
\end{tabular}
\end{center}
\caption{Comparación entre FlowDroid, JoDroid y Prototipo. Ilustra los
porcentajes para Precisión, Recall, y la detección de leaks mediante
flujos implícitos.\newline}
\label{tb:comparacion}
\end{table}


En base a los resultados para el conjunto de
evaluación(compuesto por 20 testcases, de los cuales 14 presentan leaks),
obtenidos en los anteriores apartados, 
% donde se presentó un análisis detallado para la evaluación entre FlowDroid vs
% Prototipo, y, JoDroid vs Prototipo; 
se comparan las tres herramientas: FlowDroid, JoDroid y Prototipo, frente a 
Precisión, Recall, y la detección de fugas de información mediante flujos
implícitos. La tabla \ref{tb:porcentajes} ilustra todos los resultados y la
tabla \ref{tb:comparacion} ilustra los respectivos porcentajes.\newline

% \begin{table}[b]
% \begin{center}
% \begin{tabular}{c|c|c|c|}
% \cline{2-4}
% & \cellcolor{gray!30}FlowDroid & \cellcolor{gray!30}JoDroid &
% \cellcolor{gray!30}Prototipo \\
% \cline{1-4}
% \multicolumn{0}{ |c|  }{\multirow{0}{*}{FP} }  & 3 & 3 & 5\\ \cline{0-3}
% \multicolumn{0}{ |c|  }{\multirow{0}{*}{FN} }  & 3 & 3 & 0\\ \cline{0-3}
% \multicolumn{0}{ |c|  }{\multirow{0}{*}{TP} }  & 11 & 11 & 14\\\cline{0-3}
% \multicolumn{0}{ |c|  }{\multirow{0}{*}{TN} }  & 3 & 3 &  1\\ \cline{0-3}
% \end{tabular}
% \end{center}
% \caption{Resultados de precisión para FlowDroid y Prototipo. Resume el total de
% falsos positivos(FP), verdaderos positivos(TP), verdaderos negativos(TN) y
% falsos negativos(FN).}
% \label{tb:porcentajes}
% \end{table}
% 
% \begin{table}[b]
% \begin{center}
% \begin{tabular}{cc|c|c|c}
% \cline{2-4}
% & \multicolumn{0}{ |c|  }{\multirow{1}{*}{\cellcolor{gray!30} FlowDroid} } &
% \cellcolor{gray!30}JoDroid & \cellcolor{gray!30}Prototipo \\
% \cline{1-4}
% \multicolumn{0}{ |c|  }{\multirow{0}{*}{Precisión} }  & 78\% & 78,57\% & 73,68\%
% \\
% \cline{0-3}
% \multicolumn{0}{ |c|  }{\multirow{0}{*}{Recall} }  & 78\% & 78,57\% &  100\%\\
% \cline{0-3}
% \multicolumn{0}{ |c|  }{\multirow{0}{*}{Detección Flujos Implícitos} }  & No &
% Si & Si\\
% \cline{0-3}
% \end{tabular}
% \end{center}
% \caption{Comparación entre FlowDroid, JoDroid y Prototipo. Ilustra los
% porcentajes para Precisión, Recall, y la detección de leaks mediante
% flujos implícitos.\newline}
% \label{tb:comparacion}
% \end{table}

\textbf{\textit{Desempeño}}\newline 
Como muestran las tablas \ref{tb:resultados} y \ref{tab:JoDroid-Prototipo}, el
Prototipo presenta mejor desempeño frente FlowDroid y JoDroid. En el caso de
FlowDroid, en promedio tarda 3,3 segundos más que el Prototipo para ejecutar el
análisis. En el caso de JoDroid, el tiempo de análisis es costoso en comparación
a las otras herramientas, puesto que su tiempo de ejecución oscila entre 21 y 22
minutos.\newline

% EXPLICAR POR QUE, DE ACUERDO A LA TECNICA DE ANALISIS?\newline
% Como muestran las tablas \ref{tb:resultados} y \ref{tab:JoDroid-Prototipo}, el
% análisis de flujo de información basado en lenguajes tipados de seguridad(en que
% se fundamenta la propuesta de análisis) presenta un mejor desempeño, frente a
% las técnicas en que se basan FlowDroid y JoDroid, análisis de flujo de datos
% mediante técnicas tainting y IFC(Information Flow Control) mediante PDG y
% slicing, respectivamente.
\textbf{\textit{Precisión y Recall}}\newline
Tanto FlowDroid como JoDroid presentan mejor Precisión que el Prototipo, es
decir que el Prototipo presenta más falsos positivos(FP).\newline 
Por otro lado, el Prototipo presenta mayor Recall frente a FlowDroid y JoDroid,
por tanto, el Prototipo detecta mayor cantidad de fugas existentes (reporta
menos FN).
Para este caso particular, el Prototipo detecta todos los TP.\newline 
En consecuencia, es posible decir que aunque el Prototipo presenta mayor
cantidad de FP frente a FlowDroid y JoDroid, deja pasar menos fugas de
información.\newline
En lo que respecta a flujos implícitos, a diferencia de FlowDroid, tanto JoDroid
como el Prototipo detectan fugas de información a través de Flujos
implícitos.\newline

% EXPLICAR POR QUE DE ACUERDO A LAS TECNICAS DE ANALISIS?\newline
% El análisis de flujo de información mediante lenguajes tipados de
% seguridad(en que se basa el Prototipo), ofrece un mejor Recall que
% FlowDroid, sin embargo FlowDroid es más preciso. Esto se traduce en que la
% propuesta de análisis evaluada a través del Prototipo, presenta más falsos
% positivos que FlowDroid, pero no deja pasar fugas de información.\newline
% Por otro lado, el Prototipo detecta fugas de información presentes en Flujos
% implicitos, FlowDroid No.\newline
% El análisis de flujo de información mediante lenguajes tipados de seguridad,
% ofrece igual Recall que la técnica de PDG utilizada por JoDroid, sin
% embargo, JoDroid presenta mejor Precisión.\newline
\textbf{\textit{Análisis métricas acorde al tipo de análisis}}\newline
Analizando los resultados para las métricas de desempeño, precisión y recall;
descritas anteriormente, acorde al tipo de análisis y técnicas en que se basa
cada herramienta, es posible anotar:\newline 

- Desempeño:\newline 
El prototipo presenta mejor desempeño, como resultado de analizar flujo de
información mediante lenguajes tipados de seguridad, más específicamente a
través de Jif. Dado que Jif recurre a técnicas de compilación(label checking) y
no requiere la generación de grafos de dependencia, el análisis toma menos
tiempo.

-Precisión, Recall y detección de Flujos implícitos:\newline 
el análisis pesimista en que se basa el Prototipo, donde se asume que todos los
métodos implementados en la aplicación serán invocados, hace que los resultados
del análisis sean menos precisos, generando más falsos positivos(FP). 
%En consecuencia, se tiene un análisis unsound?.\newline 
%%VERIFICAR SENSIBILIDAD AL CONTEXTO,,,,
%(EL ANÁLISIS PESIMISTA O LA TECNICA DE LENGUAJES TIPADOS?)
Para la evaluación realizada, el análisis de flujo de información mediante el
sistema de anotaciones de Jif, ofrece un mejor recall, frente a: el análisis de
flujo de datos en que se basa FlowDroid y el análisis de flujo de información
mediante System Dependences Graphs  en que se basa JoDroid. Esto hace que para
el experimento, la técnica de análisis del prototipo sea completa(completeness),
puesto que, dentro de los leaks detectados están todos los leaks que
efectivamente existen.\newline
Una ventaja de las técnicas basadas en control de flujo de información es que al
analizar tanto flujos explícitos como flujos implícitos, detectan la generación
de leaks mediante flujos implícitos casi de forma natural, contrario a lo que
sucede en las técnicas de análisis de flujos de datos, en estas, si la
construcción del análisis no define casos de propagación para el marcado de
datos mediante flujos implícitos, el análisis carece de criterios para la
detección de fugas a través de los mismos.

Los resultados de evaluación confirman las hipótesis iniciales del presente
trabajo, según las cuales se esperaba que: al hacer análisis de flujo de
información mediante lenguajes tipados de seguridad, los resultados del análisis
fuesen más rápidos pero menos precisos, reportando más falsos positivos que
JoDroid y FlowDroid.\newline
Tal resultado refleja lo ilustrado por la teoría \cite{taghdiri-etal-2010} y
\cite{hammer09ijis}.\newline

La tabla \ref{tab:resumen} resume 
las ventajas, desventajas, similitudes y
diferencias entre el Prototipo y las herramientas de comparación, FlowDroid y
JoDroid, respectivamente. En esta se ilustra por ejemplo, como el Prototipo
detecta la fuga de información a través de flujos implícitos, mientras que
FlowDroid no.

\begin{table}[H]
%\begin{center}
\small\addtolength{\tabcolsep}{-3pt}
%\begin{tabular}{|p{4cm}|p{1cm}|p{1cm}|}
\begin{tabular}{|p{4cm}|p{1cm}|p{1cm}|p{1cm}|p{1cm}|p{1cm}|p{1cm}|p{1cm}|p{1cm}|}
	\hline
	{\multirow{2}{*}{\textbf{Item}}}
	&\multicolumn{4}{c|}{\cellcolor{gray!30}\textbf{Prototipo vs FlowDroid}} &
	\multicolumn{4}{c|}{\cellcolor{gray!55}\textbf{Prototipo vs JoDroid}}\\
	\cline{2-9}
	 & \cellcolor{gray!30}\tiny{\textbf{ventaja}} &
	 \cellcolor{gray!30}\tiny{\textbf{desvent}} &
	 \cellcolor{gray!30}\tiny{\textbf{similit}}&
	 \cellcolor{gray!30}\tiny{\textbf{diff}} &
	 \cellcolor{gray!55}\tiny{\textbf{ventaja}} &
	 \cellcolor{gray!55}\tiny{\textbf{desvent}} &
	 \cellcolor{gray!55}\tiny{\textbf{similit}}&
	 \cellcolor{gray!55}\tiny{\textbf{diff}}\\
	\hline
	\footnotesize{Menor Precisión} & &\checkmark & & & &\checkmark& &\\
	\hline
	Mayor Recall &\checkmark& & & &\checkmark& & &\\
	\hline
	Menor costo en desempeño & & & & &\checkmark& & &\\
	\hline
	Bajo costo en desempeño & & &\checkmark& & & & & \\
	\hline
	Detección de flujos implícitos & \checkmark& & & & & &\checkmark&\\
	\hline
	No detección automática de sources y sinks & &\checkmark&& &&&\checkmark& \\
	\hline
	No soporte para Análisis interApp & &\checkmark& & & & &\checkmark&\\
	\hline
	\footnotesize{Tipo de análisis(flujo de información; flujo de datos)} & & &
	&\checkmark& & & &\\
	\hline
	Tipo de análisis IFC & & & & & & &\checkmark&\\
	\hline
	Técnica de análisis: PDG, slicing & & & & & & & &\checkmark\\
	\hline
\end{tabular}
%\end{center}
\caption{Síntesis ventajas, desventajas, similitudes y diferencias; del
Prototipo frente a FlowDroid y JoDroid(respectivamente).\newline}
\label{tab:resumen}
\end{table}

\section{Tipos de análisis y técnicas evaluadas}
% En las subsecciones anteriores(4.2.1 a 4.2.3), se analizaron los resultados de
% evaluación con respecto a un conjunto de aplicaciones específico. En la presente
% sección, el análisis se basa en las herramientas previamente evaluadas, pero
% haciendo enfásis en las técnicas utilizadas por las mismas.
%tipo de analisis y técnica
\textbf{FlowDroid} se fundamenta en análisis de flujo de datos, mediante
técnicas tainting.\\
El código .dex a ser analizado es transformado a una representación
intermedia(Jimple representation).\\
El análisis parte de la construcción de un super-grafo del programa que se
analiza, el super-grafo es una colección de grafos dirigidos, mediante los
cuales se representa el programa, donde los nodos asocian las sentencias del
programa y las aristas, la forma en que estas se conectan. Para recorrer el
super-grafo utiliza un algoritmo basado en el problema de
graph-reachability\cite{Graph-reachability}; cuyo costo computacional es de
orden polinomial O(ED3), donde E representa funciones de flujo de datos(dataflow
functions) y D conjunto de elementos para guiar el seguimiento de los
datos marcados(set of data flow facts).\newline
Para propagar la marca en los datos que analiza omite el control de flujo de
información, sólo se centra en el flujo de datos marcados como sources y
sinks.\newline
La herramienta recibe como entrada el apk del aplicativo, detecta
automáticamente los sources y sinks del programa mediante el uso de SuSi y
genera un reporte del análisis.

\textbf{JoDroid} se fundamenta en análisis de control de flujo de información,
aplicando técnicas de grafos de dependencia(PDG) y técnicas slicing.\newline 
El código .dex es transformado a código de representación intermedio(SSA-form).
Construye un grafo PDG, donde los nodos representan statements y expresiones, y
las aristas modelan las dependencias sintácticas entre los statements y
expresiones. Este PDG permite modelar flujos explícitos e implícitos.\newline
El costo computacional un análisis basado en PDG es de orden polinomial
O(N)3\cite[page 3]{FCO-PDG}.\newline 
Para hacer seguimiento al control de flujo de información, utiliza labels de
seguridad, estos califican con nivel de seguridad alto o bajo información de
variables y statements.\newline
Los procedimientos para usar la herramienta comprenden: generar el punto de
entrada del análisis, generar el PDG, ejecutar el respectivo análisis. Primero,
recibe como entrada el apk y manifest del aplicativo para generar un archivo con
el punto de entrada del análisis; luego, a partir del archivo devuelto
anteriormente genera el PDG, finalmente, recibe como entrada el PDG, lista los
statements y variables del aplicativo para que se indique manualmente los
sources y sinks, y genera el respectivo análisis.

\textbf{La propuesta} está basada en análisis de flujo de información mediante
lenguajes tipados de seguridad, más específicamente mediante Jif.\newline
Para cada programa a analizar se debe implementar la versión Jif, es decir,
el programa implementado acorde al sistema de anotaciones de Jif. A
partir de tales anotaciones el compilador verifica la generación de flujos de
información que incumplan la política de seguridad establecida, para reportarlos
como flujos de información indebidos. 
Al ser evaluado directamente por un
compilador, obtiene los beneficios de bajo costo computacional del mismo.\newline
La generación del análisis para verificar la política de seguridad
definida, requiere dos pasos. Primero, se genera la versión Jif del aplicativo a
analizar; Segundo, se compila el .jif, para obtener el reporte de análisis.

En el cuadro \ref{tab:comparacion} se resumen los puntos comparados
anteriormente.
\begin{table}[H]
\begin{center}
\small\addtolength{\tabcolsep}{-3pt}
\begin{tabular}{|p{2,2cm}|p{1,3cm}|p{5cm}|p{2cm}|p{2cm}|}
	\hline
	\textbf{Herramienta} & \textbf{Tipo} & \textbf{Técnicas} & \textbf{Costo
	computacional} & \textbf{ Entradas} \\
	\hline
	FlowDroid & Flujo de datos & 
	Tainting; super-grafo integrado por grafos dirigidos; Representación intermedia
	Jimple; algoritmo graph-reachability & Polinomial
	O(ED3)\cite{Graph-reachability} & apk\\
	\hline
	JoDroid & Flujo de información & PDG; slicing; Representación intermedia(SSA-
	form) & polinomial O(N)3\cite{FCO-PDG} & apk; Manifest; sources y sinks
	\\
	\hline
	Prototipo & Flujo de información  & Lenguajes tipados de seguridad; Type
	checking & Tiempo de compilación(Tiempo realmente bajo) & código fuente
	\\
	\hline
\end{tabular}
\end{center}
\caption{Generalidades técnicas de análisis evaluadas}
\label{tab:comparacion}
\end{table}	
