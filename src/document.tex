\documentclass[a4paper, 11pt, oneside]{report}

%%muy util para salto de línea
%\vspace{0.5 cm}
%Para garantizar la división correcta de palabras en castellano
\usepackage[spanish,activeacute]{babel}
%%\hyphenrules{nohyphenation} 
%algotith enviroment
\usepackage{algorithm}   
 
% Conservar division de palabras, listarlas
\hyphenation{
OnCreate
}    


\usepackage[pdftex]{graphicx}    
%avoid put imagen in specific space 
\usepackage{float}
  % declare the path(s) where your graphic files are
  % \graphicspath{{../pdf/}{../jpeg/}} 
  \graphicspath{{../imagenes/}}
  % and their extensions so you won't have to specify these with
  % every instance of \includegraphics
\DeclareGraphicsExtensions{.png,.jpeg,}    
%Paragraph
%\usepackage{blindtext}

%\usepackage{epsfig}% para los ejemplos con postscript.
%\usepackage{epstopdf}

% inserción url's notas de pie.
\usepackage{url}

\usepackage{footnote} 

\usepackage{tablefootnote}  


%url
\usepackage[colorlinks=true,urlcolor=blue,linkcolor=blue]{hyperref}

%Numeración de página
\setcounter{page}{1}
\pagenumbering{arabic}

%insertat source code
\usepackage{listings}  
%%defino atributos para code bash
%%\lstset{language=bash, basicstyle=\ttfamily, frame=single}  \footnotesize
\lstset{basicstyle=\footnotesize\ttfamily, escapechar={', --} }

%comentarios que incluyen varias líneas
\usepackage{verbatim}

\usepackage{perpage} 

%negrita
\usepackage{bold-extra}

% idioma
\usepackage[utf8]{inputenc}
\usepackage[spanish]{babel}

%tablas
\usepackage{booktabs}
\usepackage{multirow}

% Personalizar listas
\usepackage{paralist}

%rotar tablas
%\usepackage{rotating}
%formate texto tablas 
\usepackage{array}
%color tablas
\usepackage{colortbl}
\usepackage[table]{xcolor} 
\usepackage{longtable}
%espaciado 
\usepackage{setspace}
%%\onehalfspacing
\setlength{\parindent}{0pt}
\setlength{\parskip}{2.0ex plus0.5ex minus0.2ex}

%margenes según n. icontec
\usepackage{vmargin}
\setmarginsrb           { 4.0cm}  % left margin
                        { 3.0cm}  % top margcm
                        { 3.0cm}  % right margcm
                        { 2.0cm}  % bottom margcm
                        {   0pt}  % head height
                        {0.0 cm}  % head sep
                        {   9pt}  % foot height
                        { 1.0cm}  % foot sep

% Paquetes de la AMS:
\usepackage{amsmath, amsthm, amsfonts}
\addto\captionsspanish{\def\refname{\textsc{Bibliografía}}}

\newcommand\portada{
	\begin{titlepage} 
		\begin{center}
			{\large \bf ANÁLISIS DE FLUJOS DE INFORMACIÓN EN APLICACIONES ANDROID  \par }
			\vfill
			{\large\bf LINA MARCELA JIMÉNEZ BECERRA \par}
			\vfill
			{\large\bf UNIVERSIDAD DE LOS ANDES  \par}
			{\large\bf FACULTAD DE INGENIERÍA \par}
			{\large\bf DEPARTAMENTO DE INGENIERÍA DE SISTEMAS Y COMPUTACIÓN \par}
			{\large\bf  BOGOTÁ 2015 \par}
		\end{center}
	\end{titlepage}
}

\newcommand\contraportada{
	\begin{titlepage}
		\begin{center}
			{\large \bf ANÁLISIS DE FLUJOS DE INFORMACIÓN EN APLICACIONES ANDROID \par}
			\vfill
			{\large\bf LINA MARCELA JIMÉNEZ BECERRA \par}
			\vfill
			{\large\bf Asesores\par}
			{\large\bf Martín Ochoa, Ph. D. \par}
			{\large\bf Researcher at the software engineering chair of the TU Munich\par}
			{\large\bf Sandra Julieta Rueda Rodriguez, Ph. D. \par}
			{\large\bf Profesora Asistente, DISC Universidad de los Andes\par}
			\vfill
			{\large\bf UNIVERSIDAD DE LOS ANDES  \par}
			{\large\bf FACULTAD DE INGENIERÍAS \par}
			{\large\bf DEPARTAMENTO DE INGENIERÍA DE SISTEMAS Y COMPUTACIÓN \par}
			{\large\bf BOGOTÁ 2015 \par}
		\end{center}
	\end{titlepage}
} 


\begin{document} 
\portada  
\contraportada
\tableofcontents
%Cambiar el nombre Cuadro por tabla
\renewcommand{\tablename}{Tabla}
%Modificar el nombre: Índice de cuadros, por Índice de tablas
% \renewcommand{\listtablename}{ÍNDICE DE TABLAS}\listoftables
% %Modificar el nombre a mayúsculas
% \renewcommand{\listfigurename}{ÍNDICE DE GRÁFICAS}\listoffigures
% %hacer que el título del capitulo aparezca en mayuscula
\renewcommand\chaptername{CAPÍTULO} 

\label{ch:resumen}
\chapter*{
\begin{center}
	Resumen
\end{center} }
Breve resumen del trabajo : contexo, problema, solución propuesta, resultados
alcanzados.

El presente trabajo de investigación plantea aplicar técnicas de análisis
basadas en control de flujo de información, con el fin de verificar la ausencia
de fugas de información en aplicaciones Android, desde su construcción. Puesto
que, controlar el acceso y uso de la información, representa una de las
principales preocupaciones de seguridad en dichos aplicativos.\newline 
Un estudio reciente de seguridad en dispositivos móviles, publicado por
McAfee\cite{McAfeeReport}, revela que en el contexto de aplicativos Android:
80\% reúnen información de la ubicación, 82\% hacen seguimiento de alguna acción
en el dispositivo, 57\% registran la forma de uso del celular(mediante Wi-Fi o
mediante la red de telefonía), y 36\% conocen información de las cuentas de
usuario.\newline
Diferentes trabajos de investigación han abordado el problema de pérdida de
información en aplicativos Android, sin embargo, literatura científica existente
al respecto señala que la mayoría de propuestas aplican técnicas data-flow
análisis a partir del bytecode. Enfocandose en detección de fugas de información
en aplicativos ya implementados, y no, en garantizar el cumplimiento de
determinadas políticas de seguridad desde la construcción del aplicativo. Por
lo tanto, el desarrollador de la aplicación carece de herramientas de apoyo para
verificar si la aplicación que implementa, cumple con determinadas políticas de
seguridad.\newline

Para contribuir en la verificación de políticas de seguridad desde la
implementación de aplicativos Android, se propone una herramienta para análisis
estático de flujo de información basada en el sistema de anotaciones de Jif. De
manera que, partiendo de las anotaciones de seguridad definidas por el
desarrollador en el código de su aplicación, se verifique si esta cumple 
con determinada política de seguridad.

El presente trabajo parte del diseño de solución ideal que se plantea
\ref{sec:sol-desig}, centrandose en un conjunto reducido de clases de la API
Android y un conjunto específico de sources y sinks, acorde a una política de
seguridad establecida.\newline 
De este modo, el desarrollador define la propiedad de segurirad en su
aplicativo, mediante el sistema de anotaciones y compila la respectiva versión
Jif para verificar el cumplimiento de la misma.\newline 
No obstante, para efectos de evalución, la anotación del desarrollador es
automatizada mediante un generador de anotaciones, que anota el aplicativo
acorde a la política de seguridad a evaluar.\\
Para la evaluación se especifica un conjunto de aplicaciones, estás son
analizadas con la la herramienta de análisis propuesta, y los resultados
obtenidos son comparados con FlowDroid y JoDroid, dos herramientas de análisis
estático basadas en flujo de datos y flujo de información, respectivamente.\newline

Los resultados de evaluación para la solución propuesta coindiden con las
hipótesis iniciales, puesto que, al estar basada en control de flujo de
información, el análisis es más rápido, menos preciso pero a la vez completo. Es
decir, reporta más falsos positivos pero identifica todas o la mayoría de fugas.


























\label{ch:introduccion}
\chapter{Introducción}
% Información más detallada del contexto en el que se presenta el problema.  
% 
% Datos estadísticos acerca del problema, de los costos que genera, etc.  
% 
% Presentación informal y breve, pero clara, del problema.

En aplicativos Android, el manejo de la información del usuario, es una de las
principales preocupaciones de seguridad. Según un estudio reciente de seguridad
en dispositivos móviles, publicado por McAfee\cite{McAfeeReport}, una importante
cantidad de aplicaciones Android invaden la privacidad del usuario, reuniendo
información detallada de su desplazamiento, acciones en el dispositivo, y su
vida personal.\newline
Por otro lado, para controlar el acceso a información manipulada por sus
aplicaciones, el desarrollador cuenta con los mecanismos de seguridad proveídos
por la API de Android, sin embargo, al estar basados en políticas de control de
acceso, se limitan a verificar el uso de los recursos del sistema acorde a los
privilegios del usuario, lo que suceda con la información una vez sea accedida,
está fuera del alcance de este tipo de controles. Al no contar con herramientas
de análisis de flujo de información en aplicaciones Android, o al utilizar
librerías de terceros, para el desarrollador es difícil verificar
el cumplimiento de políticas de confidencialidad e integridad en la aplicación
próxima a liberar. Por consiguiente, el desarrollador no tiene cómo asegurar la
ausencia de fugas de información en la aplicación.\newline 
Si bien, en el campo de aplicativos Android existen diferentes propuestas para
detectar fuga de información, en su mayoría  se enfocan en precisión y
eficiencia del análisis para detectar fugas de datos en aplicaciones de terceros
ya implementadas. Estas propuestas
no abordan el problema del lado del desarrollador, analizando flujos de
información de la aplicación para verificar el cumplimiento de políticas de
confidencialidad e integridad.\newline
%  hacen falta propuestas que aborden
% el problema análizando flujo de información, 
% mediante técnicas de lenguajes tipados de seguridad(NO ESTOY SEGURA DE ESTA
% AFIRMACIÓN, ES DECIR QUE SÓLO CON SECURITY-TYPED ES POSIBLE DETECTAR FUGAS
% DE INFORMACIÓN),
% lo que se traduce en
% imposibilidad para detección de fugas mediante sentencias de control, por
% ejemplo, la no detección de flujos implicitos.\newline 
Ante esto, y con el fin de proveer una herramienta de apoyo al desarrollador, de
modo que verifique el cumplimiento de políticas de seguridad en sus
aplicaciones, el presente trabajo aborda el problema de fugas de información en
aplicaciones Android, analizando flujos de información de la aplicación mediante
técnicas de lenguajes tipados de seguridad.

\section{Técnicas de análisis de código}
\label{sec:contexto}
%-(Lina: falta actualizar contenido)\newline
\subsection{Análisis estático y dinámico}

Las soluciones propuestas para detectar fuga de información en aplicaciones
Android, se enmarcan en el análisis estático o dinámico de la aplicación, en
algunos casos, se combinan ambos tipos.\newline 
En \textbf{análisis estático}\cite{Static-dynamic}, se estudia el código del
programa para inferir todos los posibles caminos de ejecución. Esto se logra
construyendo modelos de estado del programa, y determinando los estados posibles
a alcanzar por el programa.
No obstante, debido a que existen múltiples posibilidades de ejecución, se opta
por construir un modelo abstracto de los estados del programa. La consecuencia
de tener un modelo aproximado es pérdida de información y posibilidad de menor
precisión en el análisis.\newline 
Por otro lado, en \textbf{análisis dinámico} se ejecuta el programa y se analiza
su comportamiento, verificando el camino de ejecución que ha tomado el programa.
Esa exactitud en la ejecución que se verifica da precisión al análisis, porque
no es necesario construir un modelo aproximado de todos los posibles caminos de
ejecución.

Aunque los resultados del análisis estático pueden perder precisión, la ventaja
es que son generalizables, porque el modelo construido representa una
descripción del comportamiento del programa, independientemente de las entradas
y el contexto en que este se ejecute. Ahora, con el análisis dinámico, no es
posible generalizar sus resultados para futuras ejecuciones, porque no
existen garantías de que las entradas con que fue ejecutado el programa,
contengan características para todos los posibles caminos de ejecución.\newline 
Además de las ventajas y desventajas de ambas clases de análisis, cada uno
implica su propio reto. Mientras en el análisis estático la dificultad está
en construir el modelo de abstracción adecuado, en el análisis dinámico, es
complejo encontrar un conjunto de casos de prueba representativo.\newline
%, a analizar durante la ejecución del programa.\newline
Por otra parte, dependiendo de la finalidad con que se detecte la fuga de
información, un tipo de análisis puede ser más apropiado que otro. Si se busca
contener la fuga de información a tiempo de ejecución, análisis dinámico es el
camino apropiado. 
De lo contrario, si se busca garantizar que a tiempo de
ejecución la aplicación no incurre en fugas de información, resulta más
conveniente aplicar análisis estático, porque cumplir con tales garantías
implica definir políticas de confidencialidad y/o integridad desde la
implementación de la aplicación. REFERNCIA

Precisamente, el propósito fundamental del presente trabajo es ofrecer al
desarrollador de aplicaciones Android una herramienta para aplicar políticas de
confidencialidad en la aplicación que implementa, así, la aplicación se
ejecutará exitosamente, si y sólo si, cumple con las políticas definidas, de lo
contrario, el desarrollador puede revisar y corregir su código.\newline
 
\subsection{Detectar o garantizar políticas de seguridad } 
Generalmente, para verificar el cumplimiento de políticas de seguridad mediante
análisis estático, se aplican técnicas de seguridad de tipado
(Typed-Inference/Security-Typed Analysis) y técnicas de flujo de
datos(Data/Control Flow Analysis)\cite{Information-Flow-Java}.\newline 
Con \textbf{técnicas Security-Typed} las propiedades de confidencialidad e
integridad son anotadas en el código, y verificadas a tiempo de compilación,
garantizando su cumplimiento a tiempo de ejecución.\newline 
Con \textbf{técnicas de flujo de control} y \textbf{técnicas de flujo de datos},
las políticas de seguridad son verificadas haciendo seguimiento al control de
flujo, o al flujo de datos, respectivamente. Estás técnicas suelen utilizar
grafos de Control de Flujo CFG(Control Flow Graph), Grafos de Flujo de Datos
DFG( Data Flow Graph) y Grafos de llamadas CG (Call Graphs).

Acorde a literatura científica en el ámbito de seguridad de aplicativos
Android, parte importante de las propuestas para análisis de fuga de
información(TaintDroid\cite{TaintDroid}, Flow-Droid\cite{FlowDroid-Thesis},
DidFail\cite{DidFail}, DroidForce\cite{DroidForce}), parten del bytecode para
realizar data-flow analysis, mediante técnicas de análisis tainting. Las
técnicas de análisis tainting, son un tipo especial de análisis de flujo de
datos, donde se hace seguimiento al flujo de datos entre un conjunto de fuentes
consideradas privadas y/o sensibles; y un conjunto de destinos considerados no
confiables, sources y sinks, respectivamente.\newline 
Si bien, tales propuestas permiten detectar flujos de datos indebidos en
aplicaciones Android, están enfocadas a analizar aplicaciones ya implementadas,
y no, en garantizar el cumplimiento de determinadas políticas de seguridad desde
su construcción.
% Tales propuestas se enfocan en analizar aplicativos de terceros para detectar
% flujos de datos indebidos, y no: para garantizar el cumplimiento de determinadas
% políticas de seguridad. 
% En consecuencia, es complejo que el desarrollador
% garantice la ausencia de fugas de información en la aplicación que implementa,
% partiendo de tales herramientas. 
% 
% Puesto que, al seguir únicamente a los datos
% marcados, los datos no marcados para el análisis, pueden acarrear fugas de
% información(under-tainting). Adicionalmente, si no se hace seguimiento al flujo
% de control pueden existir fugas de información a través de flujos implícitos,
% ya que, el análisis estará centrado en flujos explícitos.\newline
% No obstante, las limitaciones propias de un análisis basado en flujo de datos,
% pueden superarse enfocando el análisis de la aplicación hacia técnicas de
% análisis basadas en control de flujo de información, ya que estas analizan el
% aplicativo de forma estática para identificar todos los posibles caminos que
% podría tomar la aplicación en tiempo de ejecución. 

Precisamente, mediante análisis basado en control de flujo de información y
técnicas Security-Typed, es posible garantizar el cumplimiento de políticas de
seguridad en las aplicaciones que se implementa, puesto que,
% Así, con análisis basado en
% control de flujo de información, no sólo es posible prevenir fugas por
% under-tainting y flujos implícitos; sino que también, es posible ofrecer
% garantías del cumplimiento de determinadas políticas de seguridad.\newline 
% Así, con análisis basado en
% control de flujo de información es posible garantizar el cumplimiento
% de determinadas políticas de seguridad, desde la construcción del
% aplicativo.\newline 
% Ahora, 
las reglas para evaluar control de flujo de
información pueden definirse mediante técnicas Security-Typed, por ejemplo como
se definen con Jif \ref{JIF-Tool}, un lenguaje tipado de seguridad para realizar
control de flujo de información en aplicativos Java.
% 
% el inconveniente es que está implementada para
% aplicaciones en Java, y no para aplicativos Android.\newline 
\subsection{Security Typed Languages}
En general, herramientas basadas en técnicas de análisis
Security-Typed, involucran conceptos como flujo de información, políticas de
confidencialidad e integridad, y chequeo de tipos.

\emph{Flujo de información}: el flujo de información describe el
comportamiento de un programa, desde la entrada de los datos hasta la salida de
los mismos. 

\emph{Políticas de confidencialidad e integridad}: confidencialidad e integridad
son políticas de seguridad aplicables mediante control de flujo de información.
Mientras la confidencialidad busca prevenir que la información fluya hacia
destinos no apropiados, la integridad busca prevenir que la información provenga
de fuentes no apropiadas\cite{LanguageIFS-2013}. Una importante diferencia
entre confidencialidad e integridad, es que la integridad de la información de un programa puede ser
alterada sin la interacción con agentes externos.\newline %\textcolor{red}{(por qué es importante?)}
Ambas políticas son fundamentales para garantizar propiedades de
seguridad.\newline 
Con políticas de confidencialidad, es posible garantizar ausencia de fugas de
información. Con políticas de integridad, la finalidad es evitar
modificación de la información, de forma no consentida.\newline 
Verificar que un programa utilice la información acorde a
tales políticas, implica analizar sus flujos de información de inicio a fin.
Para tal análisis se deben definir: políticas de flujo de información y
controles de flujo de información, es decir, las políticas de seguridad a
evaluar y los mecanismos para aplicarlas. 

\emph{Chequeo de tipos}: al usar un lenguaje tipado de seguridad, las políticas
son definidas a través del lenguaje, porque son expresadas mediante anotaciones
en el código fuente del programa a verificar, y su evaluación se realiza
mediante chequeo de tipos.\newline 
El chequeo de tipos consiste en una técnica estática,
también utilizada para analizar flujo de información durante la compilación de
un programa, más específicamente en la etapa de análisis semántico, el
compilador identifica el tipo para cada expresión del programa y verifica que
corresponda al contexto de la expresión.\newline
%  El
% chequeo de tipos también es una técnica estática utilizada para analizar flujo
% de información durante la compilación de un programa, más específicamente en la
% etapa de análisis semántico, el compilador identifica el tipo para cada
% expresión del programa y verifica que corresponda al contexto de la expresión.
Bajo este principio de chequeo, lenguajes tipados de seguridad aplican
políticas de control de flujo, definiendo para cada expresión del programa un
tipo de seguridad(security type), de la forma:  tipo de dato y label de
seguridad(security label). Donde el label de seguridad regula el uso del dato,
acorde a su tipo.\newline 
El compilador realiza el chequeo de tipos, partiendo del conjunto de labels de
seguridad. Así, si el programa pasa el chequeo de tipos y compila correctamente,
se espera que cumpla con las políticas de control de flujo evaluadas.
\label{ch:problema}
\chapter{Descripción del problema}

En Android, por defecto, el desarrollador no cuenta con mecanismos para
definir políticas de confidencialidad e integridad que regulen
el flujo de información de sus aplicaciones. Siendo complejo prevenir fugas de
información del usuario, puesto que, el desarrollador carece de herramientas que
le garanticen la ausencia de flujos indeseados.\newline
Precisamente, una de las principales preocupaciones de seguridad en aplicativos
Android, es la manipulación de información del usuario.
Así lo evidencia un
estudio reciente de seguridad en dispositivos móviles, publicado por
McAfee\cite{McAfeeReport}, este señala  que una importante cantidad de
aplicaciones Android invaden la privacidad del usuario, reuniendo información
detallada de su desplazamiento, acciones en el dispositivo, y su vida personal.
De este modo, 80\% reúnen información de la ubicación, 82\%
hacen seguimiento de alguna acción en el dispositivo , 57\%
registran la forma de uso del celular (mediante Wi-Fi o
mediante la red de telefonía), y 36\% conocen información de
las cuentas de usuario.\newline
Las motivaciones para este tipo de acciones varían acorde al tipo de
información, por ejemplo: monitorear información de ubicación para mostrar
publicidad no solicitada; seguir las acciones sobre el dispositivo, para conocer
qué aplicaciones son rentables de desarrollar, o para ayudar a aplicaciones
maliciosas a evadir defensas; acceder a información de cuentas del usuario con
fines delictivos; obtener información de contactos y calendario
del usuario, buscando modificar los datos; obtener información del celular 
(número, estado, registro de MMS y SMS) para interceptar llamadas y enviar
mensajes sin consentimiento del usuario.\newline
Con o sin autorización de acceso, existen motivaciones suficientes para que un
tercero desee manipular información del usuario.\newline
Adicionalmente, el informe señala que una aplicación invasiva no necesariamente
contiene malware, y que su finalidad no siempre implica fraude; de las
aplicaciones que más vulneran la privacidad del usuario, 35\% contienen
malware.\newline 
Si bien, aplicaciones invasivas no necesariamente implican
malware y/o acciones delictivas, el cuestionamiento de fondo es la forma y
finalidad con que están accediendo la información, es decir, si información
de usuario manipulada por una determinada aplicación, realmente debería ser
accedida por otros aplicativos del dispositivo, aún cuando sean considerados no
maliciosos; y qué garantías puede ofrecer el desarrollador para que tal acceso,
efectivamente sea consentido.\newline 
La falta de control sobre los flujos de información de la aplicación puede
ocasionar fugas de información, generando problemas de seguridad tanto para
quien la implementa como para quien la usa.\newline
Como contramedida a este problema, la API de Android ofrece herramientas de
seguridad basadas en políticas de control de acceso, y el desarrollador puede
implementarlas en su aplicación. Sin embargo, estos mecanismos se centran en
regular el acceso de los usuarios del sistema a determinados recursos, y no en
verificar qué sucede con la información una vez es accedida.\newline 
Para superar tal carencia, diferentes trabajos de investigación han abordado el
problema de fuga de información en aplicaciones Android, tanto desde un enfoque
dinámico como desde un enfoque estático, la literatura existente al
respecto(TaintDroid\cite{TaintDroid}, Flow-Droid\cite{FlowDroid-Thesis},
DidFail\cite{DidFail}, DroidForce\cite{DroidForce}), indica que la mayoría de
propuestas hacen data-flow analysis mediante técnicas de análisis usando
tainting, partiendo del bytecode. Una característica sobresaliente entre estos
trabajos es el modelo de ataque, puesto que, se centran en analizar aplicaciones
de terceros asumiendo que el atacante provee bytecode malicioso. Guiar el
análisis de aplicaciones propias con el fin de verificar políticas de
confidencialidad e integridad, bajo tales propuestas, puede implicar: mayor
dificultad en el código a analizar, incompletitud en el análisis(under-tainting)
y no detección de flujos implícitos. Esto debido a que, aún cuando el
desarrollador conoce la funcionalidad de su propio código, las optimizaciones
realizadas por el compilador pueden adicionar complejidad al
mismo\cite[pag.~43]{SecureProgramming}; el seguimiento de los datos a través del
programa está centrado en datos marcados, datos no marcados quedan fuera del
análisis;   flujos de datos a través de estructuras de control, por ejemplo, las
sentencias if, permiten inferir valores de datos marcados como source, sin
necesidad de generar flujos explícitos entre sources y sinks, los cuales si
pueden ser detectados por las técnicas de análisis tainting.\newline Otra razón
fundamental para no  analizar aplicaciones propias con tales propuestas es que
están diseñadas para detectar flujos de datos indebidos, y no para garantizar el
cumplimiento de políticas de seguridad en una aplicación.\newline 
Los riesgos de seguridad tras el under-tainting de datos, y la ausencia de
garantías en el cumplimiento de determinadas políticas de seguridad, pueden
superarse mediante control de flujo de información, Information Flow
Control(IFC), puesto que, con esta técnica se analiza estáticamente la
aplicación para identificar todos los posibles caminos que podrían tomar sus
flujos de información, garantizando que a tiempo de ejecución, la
aplicación respeta políticas de seguridad.\newline

Finalmente, partiendo del contexto que se plantea, dónde se cuenta con el código
fuente Android, porque es el propio desarrollador quien requiere evaluar
políticas de confidencialidad en su aplicación, para  garantizarle
al usuario que la aplicación las cumple. Resulta apropiado proveer una
herramienta de apoyo al desarrollador, mediante la cual analice el flujo de
información de la aplicación próxima a liberar, y verifique el cumplimiento de
políticas de seguridad.
 
\section{Trabajos Relacionados}
\label{sec:trabajo}
\subsection{JIF}
\label{JIF-Tool}
JIF(Java Information Flow), es un lenguaje tipado de seguridad que
permite extender el lenguaje de programación Java,  con control de flujo de
información y control de acceso, usando anotaciones de seguridad. El compilador
usa estas anotaciones durante el chequeo de tipos, verificando el
cumplimiento de la propiedad de seguridad non-interference.

Usar JIF para el análisis estático de flujo de información de un programa,
requiere implementar la versión del mismo, especificando mediante el conjunto de
labels de JIF, las políticas de seguridad a verificar. La implementación de
programas JIF está basada en el modelo de etiquetas DLM(Decentralized Label
Model), donde un principal es una entidad con autoridad para observar y cambiar
aspectos del sistema, así, un principal puede definir y hacer cumplir los
requerimientos de seguridad del dueño de la información. Para expresar una
relación de confianza entre principals, se define la relación acts-for, a partir
de la cual, se derivan dos tipos de principals: top principal y botton
principal, un top principal puede actuar para todos los principals, mientras
que, un botton principal permite que todos los principals actúen para el. Las
políticas de seguridad se condensan en Políticas de Confidencialidad y Políticas
de Integridad, con ellas se determina el conjunto de principals readers y
writes, y el comportamiento que deberían tener.
El compilador de JIF aplica chequeo de labels para verificar  el cumplimiento
de las políticas de seguridad definidas en el programa, cuando determina que
efectivamente las cumple, da paso al compilador de Java para generar su versión
ejecutable.

Además del modelo de labels en que se centra, JIF incluye mecanismos que
aportan características adicionales en la implementación de programas para
seguimiento de Flujo de información. La opción de flexibilizar las políticas
de seguridad de la información, hace parte de estas características adicionales,
y se logra aplicando el mecanismo Downgrading. Dependiendo del tipo política al
que se realiza downgrading, políticas de confidencialidad o políticas de
integridad, el proceso se conoce como Declasificación o Endorsement,
respectivamente.

\subsection{JOANA}
\label{JOANA-Tool}
JOANA (Java Object-sensitive ANAlysis)- Information Flow Control Framework for
Java\cite{JOANA}. Verifica si una aplicación java contiene fugas de
información, mediante análisis estático de flujos de información. El análisis parte  de anotaciones en
el código fuente de la aplicación. JOANA utiliza técnicas de análisis de flujo de
datos y técnicas de análisis de control de flujo. El frontend de la herramienta
está basado en el framework de análisis de programas WALA\cite{wala}, a partir
del cual obtiene la representación intermedia del código Java en forma SSA(Static
Single Assignement), lo que permite obtener información dinámica del programa.
Por otro lado, utiliza Grafos de Dependencia, System Dependence Graphs(SDG),
para detectar dependencias entre las sentencias del programa, es decir,
si existen caminos entre sentencias etiquetadas con nivel de seguridad
alto y sentencias con nivel de seguridad bajo. Para esta etapa del análisis
recurre a técnicas de slicing y chopping, reduciendo la cantidad de caminos
posibles sólo a los válidos. Así obtiene como resultado, una mayor precisión y
reducción de falsas alarmas en el análisis.\newline

Aunque JOANA provee sencillez a la hora de anotar el código a analizar, pues
sólo es necesario anotar inputs y outputs del programa, porque la herramienta se
encarga de propagar las anotaciones en el resto del programa; carece de
características adicionales ofrecidas por sistemas de tipado de seguridad, por
ejemplo, el mecanismo downgrading facilitado por JIF.\newline 

Si bien, al igual que JOANA, la herramienta propuesta a través del presente
trabajo, aplica análisis de control de flujo de información, esta última busca
analizar aplicaciones implementadas en código Android, aprovechando las ventajas
del sistema de anotaciones de JIF. Proporcionando una herramienta de apoyo al
desarrollador de aplicaciones Android, ya que por el momento, JOANA sólo analiza
aplicaciones en JAVA.

\subsection{JoDroid}
JoDroid\cite{JoDroid-Paper} es una extención a la herramienta de análisis JOANA
para soportar analisis de aplicaciones Android.\newline 
El análisis de JOANA está basado en Program Dependence Graphs(PDG) y técnicas
slicing. Con PDGs obtiene una representación del programa que
analiza, donde los nodos representan statements y expresiones; y las aristas
modelan las dependencias sintacticas entre los statements y expresiones:
dependencias de datos y dependencias de control, por tanto el grafo está en
capacidad de modelar, tanto flujos explícitos como flujos implícitos.\newline
Con técnicas slicing provee sensibilidad al contexto, puesto que el PDG se
construye de manera tal que al hacer el backwards slice de un determinado nodo,
se obtiene cada nodo que es alcanzable por caminos del grafo que conservan
llamadas al contexto.\newline
El PDG es generado mediante el Front-end de WALA, framework que analiza bytecode
de Java. Así, los ajustes hechos a JOANA adaptan parte del Front-end de WALA
para generar el PDG de aplicaciones Android.\newline
JoDroid detecta tanto flujos explícitos como flujos implícitos.

\subsection{FlowDroid}
\label{FlowDroid-Tool}
FlowDroid es una herramienta para análisis estático de flujo de datos en
Aplicaciones Android. También permite el análisis de aplicaciones Java.\newline
Esta herramienta utiliza un tipo especial de análisis de flujo de datos:
análisis tainting, que hace seguimiento al flujo de datos entre un conjunto de
sources y un conjunto de sinks. Define tales conjuntos a partir de
SuSi[\ref{sec:susi}], un clasificador automático de sources y sinks para la Api
de Android.\newline 
FlowDroid provee un alto recall y precisión\cite{FlowDroid-Thesis} en el
análisis. El recall, mediante un fiel modelamiento del ciclo de vida de una
aplicación Android; la precisión, incluyendo elementos de análisis como:
context-, flow-, field- y object-sensitive. Para proveer sensibilidad al flujo y
al contexto, recurre a grafos de llamada; y con grafos que modelan todos los
procedimientos del programa(inter-procedural control-flow graph), analiza el
flujo de datos entre procedimientos, proporcionando field- y object-sensitive.\newline
Los autores de esta propuesta, alcanzan precisión en la construcción del grafo
de llamadas extendiendo Soot\cite{Soot}, un framework que genera código
intermedio para código Java y código ejecutable Android(dex). Adicionalmente,
con el framework Heros\cite{heros}, incluyen llamadas multihilos en el análisis
de flujo de datos entre procedimientos.\newline

Entre las limitaciones de FlowDroid está el over-tainting y la no detección
de flujos implícitos. Por tanto, la herramienta no distingue elementos marcados
ni dentro de arrays, ni dentro de collections, si se inserta un elemento marcado
dentro de alguna de estas estructuras, inmediatamente se marca el resto de
elementos. La herramienta tampoco identifica flujos implícitos,    
% causados por dependencias entre control de flujo.\newline
puesto que, según los resultados de evaluación de
DroidBench\cite{DroidBenchBenchmarks}, su benchmark; cuando Flowdroid analiza el
conjunto de aplicaciones de prueba para la identificación de flujos implícitos, no
detecta fuga de datos, generando falsos negativos en la detección de flujos
implícitos\cite[pags 32-36]{FlowDroid-Thesis}.\newline

Aún cuando el problema a atacar es el mismo: fuga de información, la propuesta
que se expone a través del presente trabajo difiere en el enfoque de análisis de
FlowDroid, mientras FlowDroid se concentra en detectar si la aplicación de un
tercero presenta fugas de información, la herramienta planteada aborda el
análisis del lado del desarrollador de la aplicación, apoyándolo en
la verificación del cumplimiento de políticas de seguridad. Así, resulta más
conveniente guiar el análisis mediante control de flujo de información, ya que
se previene fuga por datos no marcados para el análisis(under-tainting) y por
la no detección de flujos implícitos, siendo posible garantizar el cumplimiento
de políticas de seguridad.
 
\subsection{TaintDroid, Dinamic Taint Tracking, para la detección de fugas de
Información}
\label{TaintDroid-Tool}
A diferencia de las propuestas expuestas anteriormente, caracterizadas
por ejecutar el análisis de manera estática, TaintDroid es una herramienta de
análisis dinámico. Está herramienta extiende la plataforma de dispositivos
celulares Android, con el fin de verificar el uso dado por aplicaciones de
terceros a datos sensibles del usuario. El análisis aplica técnicas de análisis
tainting, marcando automáticamente como sources, datos provenientes de fuentes
consideradas privadas y/o sensibles; y como sinks, canales que permiten salir
datos de la aplicación, como por ejemplo internet.
Cada vez que un dato marcado como source sale de la aplicación, se genera un log.\newline 
Para reducir sobrecarga en el dispositivo, pues el análisis es ejecutado a nivel
de instrucciones, instrumentan la máquina virtual de Android con marcas de
propagación a nivel de: variables, métodos, mensajes y archivos. Las marcas de
variable hacen seguimiento a datos dentro de aplicaciones consideras no
confiables. Las marcas de mensaje siguen mensajes entre aplicaciones. Debido a
que TaintDroid no hace seguimiento a la ejecución de código nativo, utiliza las
marcas de métodos para hacer seguimiento a lo retornado luego de invocar métodos
de librerías nativas. Las marcas de archivo son utilizadas para verificar la
persistencia de los datos, acorde a las políticas de seguridad.\newline 
Otra medida para reducir sobrecarga en la ejecución del análisis, consiste en no
hacer seguimiento a flujos de control, generando no detección de flujos
implícitos\cite[pag 12]{TaintDroid}.\newline
Si bien, TaintDroid supera el inconveniente de sobrecarga en la ejecución del
análisis, un inconveniente característico en análisis dinámico, está limitado
para detectar fuga de datos mediante flujos implícitos, puesto que se
enfoca en hacer seguimiento a flujos de datos directos(flujos
explícitos).\newline

Al ser una herramienta de análisis dinámico, TaintDroid sólo detecta fugas de
información correspondiente a las ejecuciones presentadas por el programa, y
para la finalidad de su análisis: informar al usuario de posibles fugas de
información, se puede decir que es adecuado. No obstante, para los propósitos de
la propuesta planteada a través del presente trabajo, con la que se pretende
brindar una herramienta de análisis para que el desarrollador verifique el
cumplimiento de políticas de seguridad en la aplicación que implementa, no
resulta viable aplicar análisis dinámico, ni técnicas de análisis tainting para
hacer seguimiento a flujos de datos.
%\subsection{STAMP Análisis estático de aplicaciones}

\subsection{Comparación de técnicas}
Las técnicas utilizadas para análisis de seguridad en aplicaciones, pueden
aplicarse estática o dinámicamente, dependiendo de las propiedades del programa
en que se centre el análisis.\newline
La ejecución dinámica o estática del análisis, trae sus propias ventajas y
desventajas. En el caso de análisis estático, completitud en el análisis es una
de sus principales ventajas. Esto debido a qué, el análisis contempla todas los
caminos de ejecución en que podría incurrir el programa. Evitando que se pierdan
casos a analizar. Por otra parte, al carecer de información que sólo se puede
obtener a tiempo de ejecución, por ejemplo, las entradas que el programa
recibe, el análisis estático suele generar falsos positivos.\newline
En el análisis dinámico, una de las principales ventajas es la baja generación
de falsos positivos, puesto que, el análisis no se centra en los posibles casos
de ejecución, sino que verifica el caso de ejecución que efectivamente está
ocurriendo. No obstante, el análisis dinámico podría incurrir en incompletitud,
porque sólo verifica los casos de ejecución que se presenten, es decir, el
aplicativo podría presentar fugas de información no reportadas por el análisis,
como consecuencia de la no ejecución de los casos que permiten identificarlos.\newline 
Así, el análisis dinámico genera menor cantidad de falsos positivos que el
análisis estático, sin embargo, el análisis estático ofrece mayor completitud en
el análisis.\newline
% Ahora, partiendo del contexto de análisis planteado en el presente trabajo,
% donde el desarrollador cuenta con el código fuente de su propia aplicación y
% pretende garantizar que esta cumple con determinadas políticas de seguridad, la
% característica de completitud en el análisis estático, es cable para garantizar
% el cumplimiento de políticas de seguridad.\newline
Adicional a la forma en que son aplicadas, estática o dinámicamente, las
técnicas de análisis pueden enfocarse en hacer seguimiento al flujo de datos a
través del programa, o en verificar flujos de información. Las técnicas basadas
en tanting análisis, permiten hacer análisis de flujo de datos, marcando los
datos de interés y verificando su flujo entre sources(fuentes del programa
consideradas sensibles y/o confidenciales) y sinks(destinos considerados no
confiables). Entre las desventajas de está técnica, esta el under-tainting, es
decir, la posibilidad de fugas a través de datos no marcados para el
análisis.\newline
Las técnicas para aplicar análisis mediante control de flujo de información,
generalmente permiten definir anotaciones de seguridad en el código fuente de la
aplicación, para verificar sus flujos de información. Estas generalmente se 
basan en técnicas de seguridad de tipado(Security-Typed Analyses), o en grafos
que describen el comportamiento del programa, como Contol Dependence Graphs(PDG)
y System Dependence Graphs(SDG).
Ambas técnicas recurren a etapas de análisis de compilación(se basan en
técnicas de compilación), sin embargo, mientras las técnicas de Security-Typed
sólo requieren llegar hasta el chequeo de tipos; las basadas en grafos de
dependencia deben llegar hasta la representación de código intermedio para
generar los respectivos grafos. Si bien, con grafos de dependencia se tiene
mayor precisión en el análisis, su ejecución es costosa, ya que genera una
complejidad de orden polinomial, O(N)3\cite[page 3]{FCO-PDG}.
Las motivaciones para guiar el análisis bajo una u otra perspectiva, implica
poner a consideración tanto el nivel de precisión requerido por las propiedades
de seguridad a evaluar, como el costo de implementación y de ejecución del
análisis. \newline


 
%profundizar en las de análisis
% estático, security Typed y control flow \begin{itemize}

% 	  \item El uso de lenguajes de seguridad tipados para el análisis de flujo de
% 	  información en tiempo de ejecución, puede generar sobrecargas.\cite[pag.~1]{LanguageIFS-2013}
% 	  \item Detección de implicit information flows mediante: static enforcements
% 	  of information-flow control versus, run-time enforcement mechanisms.
% 	  \item 
% 	\end{itemize}
% 
% Dentro de las técnicas existentes para adelantar análisis de seguridad en
% aplicativos 
% Para verificar propiedades de seguridad en los aplicativos que implementa, , 
% \begin{itemize}
% 	  \item Information Flow Control
% 	  \item 
% 	  \item 
% 	\end{itemize}
	
\subsection{Clasificación de Sources y Sinks}
\label{sec:susi}
En el ámbito de análisis de flujo de información de aplicaciones,
independientemente del tipo de análisis, estático o dinámico, el punto de
partida es la definición de políticas de privacidad, los pasos sucesivos para 
detectar la perdida de información giran en torno a las políticas de privacidad
definidas.
Muchas de las propuestas para análisis de flujo de información en aplicaciones
Android, parten de un listado de sources y sinks para definir sus políticas de
privacidad. Así, en el grupo de sources se incluyen las fuentes de datos
sensibles, mientras que en el grupo de sinks, se incluyen los medios o canales
que podrían filtrar información sensible de forma no autorizada. 
La efectividad del análisis se ve limitada al listado de sources y sinks, y la
veracidad de los mismos. El inconveniente con estos sources y sinks, es que su
clasificación suele hacerse de forma manual, por tanto, existe mayor
probabilidad de error u omisión.\newline
Con el fin de precisar dicha clasificación, el trabajo de investigación SuSi
propone el uso de machine-learning para la clasificación y categorización de
sources y sinks, partiendo del código fuente de la API Android.
La propuesta de análisis se materializa en una herramienta, que recibe como
entrada métodos de Android y devuelve una lista con la respectiva
categorización de sources y sinks.\newline
La construcción del modelo de
análisis propuesto, parte definiendo los elementos necesarios para el
reconocimiento de sources y sinks; inicialmente define:
Sources y sinks, respectivamente, como las entradas y salidas de flujo de datos del
programa; un dato como un valor o una referencia a un valor; un Resource Method
como un método que lee o escribe datos en un recurso compartido. Seguidamente,
define el concepto de sources y sinks, considerando el contexto de Android:
Android Sources como llamadas a métodos tipo resources(Resources method) que
retornan valores no constantes al código de la aplicación. Android Sinks como
llamadas a methods resource, aceptando como argumento al menos un valor no
constante desde el código de la aplicación, y qué además adicionen o modifiquen
valores del recurso invocado.
El modelo de entrenamiento de SuSi usa el clasificador SMO, una implementación
del clasificador SVM(Support Vector Machines) para Weka, al que inicialmente
enseña a clasificar partiendo de ejemplos entrenados manualmente.
Adicionalmente, lo adapta utilizando la técnica de clasificación
one-againts-all, de modo que pueda representar, tanto los ejemplos de
entrenamiento, en tres clases: sources, sinks, o ninguno; como las
categorías de los sources y sinks identificados.\newline 
Los criterios de clasificación están basados en un conjunto de características,
es decir, funciones que asocian ejemplos de entrenamiento o ejemplos de prueba,
con un determinado valor.\newline
El proceso de análisis se compone de dos rondas secuenciales: clasificación y
categorización. Cada una se compone de las fases input, preparation,
classification y output. Así, la salida de la primera ronda: sources y sinks, se
convierte en entrada para la ronda de categorización, donde se definen
diferentes tipos de categorías, 12 para sources y 15 para sinks.
\section{Background}
\label{sec:back}

\subsection{Aplicaciones Android}
Explicar composición de aplicaciones Android, actividades, servicios, etc.

\subsection{Estructura de trabajo en JIF}
- estructura de los directorios del compilador Jif y estructura de trabajo en
Jif(para entender cómo funciona y cómo afecta el diseño de la
solución).

\subsection{Sintaxis de Anotación en Jif}
\label{subsec:JifSintax}
-Definición de variables: \newline 
\emph{ type\{L\} varName; }\newline 
donde type especifica el tipo de dato que
almacena la variable, \{L\} el label de seguridad  para especificar quien es el
dueño de la variable, y name, el respectivo nombre de la variable.

-Definición de arrays:\newline
en jif un array cuenta con dos labels de seguridad, Base Label(BL) y Size
Label(SL). BL indica el nivel de seguridad de los elementos que almacena el
array, controlando quien puede conocer la información del mismo. SL especifica
quienes pueden conocer la número de elementos almacenados.

-Definición de métodos.\newline
\emph{ type \{RTL\} methodName \{BL\} (arg1\{AL\},,, argn\{AL\}) :\{EL\}
}\newline 
RTL, Return Type Label, indica el label de seguridad con que
queda el tipo de dato devuelto por el método.\newline 
BL begin label, representa el máximo nivel se seguridad del pc label desde donde
se invoca el método, de este modo, el program counter label desde donde
se invoca el método debe ser menor o igual de restrictivo que el BL del
método.\newline 
AL argument label, indica el máximo nivel de seguridad  para los argumentos con
que se llama el método, así, los labels de los argumentos con que se invoca el
método deben ser menor o igual de restrictivos que los AL con que han
definido el método.\newline
EL end label, indica el pc label en el punto de terminación del método, y
representa la información que puede ser conocida.\newline
Cuando un label no es especificado, Jif define unos por defecto. En el caso de
RTL, jif hace un join entre los diferentes AL con que ha sido definido el
método.\newline
\section{Propuesta de solución}
\label{sec:propuesta-sol} 
(Descripción de la propuesta para solucionar el problema)

La propuesta para detectar fuga de información en aplicaciones Android, antes de
su publicación, consiste en proveer al desarrollador una herramienta para
análisis estático de flujos de información de la aplicación. Así, partiendo de
las anotaciones de seguridad que el desarrollador defina en el código fuente, se
verifica si la aplicación cumple con políticas de seguridad.\newline
% \textcolor{red}{Cómo sabe el desarrollador que información anota con nivel de
% seguridad bajo y con nivel de seguridad alto?}\newline
Los requerimientos iniciales para construir tal herramienta son: un lenguaje
tipado de seguridad que permita anotar código fuente Android, y el conjunto de
reglas que evaluarán las políticas de seguridad.\newline 
Al consultar literatura científica al respecto, se encuentran herramientas como
JIF \ref{JIF-Tool} y JOANA \ref{JOANA-Tool}, especializadas en anotar código
Java, pero no código Android. Es decir, las anotaciones son válidas para clases
del lenguaje java estandar, pero no para clases específicas de la API de Android.

Si bien, ambas analizan flujos de información en aplicaciones Java, y
podrían ser extendidas para anotar código Android, las técnicas utilizadas por cada una
son diferentes, por un lado, JIF es un lenguaje tipado de seguridad que basa su
análisis en el chequeo de tipos. Por el otro, JOANA es un framework basado en
análisis de grafos de dependencia. Mientras JOANA se enfoca en precisión, JIF
posee un modelo de anotaciones (DLM) encargado de definir la lattice de
seguridad adecuada para las anotaciones en el código fuente, ofreciendo un
maduro sistema que además de evaluar políticas de confidencialidad, e
integridad, permite definir características de seguridad adicionales como
declasificación y endorsement.
Acorde a los propósitos del presente trabajo, JIF ofrece los beneficios de un
lenguaje tipado de seguridad y un sistema  sólido  de anotaciones, facilitando
la definición de las propiedades de seguridad a verificar.\newline 
Partiendo de JIF como el lenguaje tipado de seguridad, los retos subsiguientes
son: implementar el setup de JIF para Android e integrar un clasificador
para sources y sinks de Android. 
El setup de JIF para Android consiste en implementar adaptaciones necesarias
a las clases de la API de Android, de modo que el compilador de Jif las
reconozca, y sea posible adicionar anotaciones de Jif dentro de aplicaciones
Android. Puesto que, aunque Jif permite extender al lenguaje Java, y en el fondo
las clases de la API Android están implementadas en Java, si no se cuenta con
una versión Jif de tales clases, el compilador de Jif no tiene como
reconocerlas, por tanto, no admite anotaciones en programas que usen esas
clases(aplicativos Android.)\newline
% en implementar las adaptaciones necesarias
% para que el lenguaje JIF reconozca código de la API de Android, y admita
% anotaciones JIF dentro de código Android, pues aunque en esencia el código
% Android es código Java, JIF no tiene como saberlo. 
Con la integración de un clasificador de sources y sinks para Android al sistema
de anotaciones de JIF, se provee información necesaria para construir las
políticas de seguridad. Porque, al conocer qué código de la API de Android,
es considerado como source o como sink, se tiene el criterio para decidir su
anotación. Permitiendo conocer el nivel de seguridad con que deben ser
anotados el código tanto de la API como el código del aplicativo a
analizar.\newline
%Si el desarrollador no sabe que es source y que es sink, cómo sabe que nivel
% de seguridad anotar durante la implementación de la versión JIf que va a
% evaluar 
La figura \ref{fig:desingInteger} expone los elementos necesarios para construir la
herramienta de análisis.
Básicamente, se requiere un módulo que extienda las clases en JIF para que el
lenguaje reconozca código de la API Android(\emph{Android Jif Setup}), más un
módulo que integre el clasificador de sources y sinks para Android(\emph{Sources
y Sinks}). 
Ambos módulos deben tener comunicación con el módulo que evalúa las
políticas de seguridad \emph{verificador de políticas}, que en esencia es el
compilador de Jif.\newline
% s decir, para que admita anotaciones dentro del código Android: Setup extended JIF classes.
% Un módulo que integre el clasificador de sources y sinks de Android al
% sistema de anotaciones en JIF:  Android Sources and Sinks. Adicionalmente, se
% requiere un modulo que evalúe las políticas de confidencialidad, Checking
% Rule Sets, que debe tener comunicación con los módulos anteriormente descritos.
Adicionalmente, la figura \ref{fig:desingInteger} ilustra que la herramienta
debe recibir como entrada el código fuente de la aplicación, debidamente
anotado por el desarrollador. De modo que el desarrollador defina las políticas
de seguridad a evaluar. A partir de tales anotaciones, la herramienta de
análisis verifica si los flujos de información del aplicativo, cumplen con la
política de seguridad expresada a través sus anotaciones, y retorna los
resultados del análisis.

Habiendo realizado las extensiones necesarias, se espera contar
con una herramienta de análisis de flujo de información, para aplicativos Android,
mediante el sistema de anotaciones de Jif.

\begin{figure}[t!]
	\begin{center}
	\includegraphics[width=10cm]{desing3-integration.jpg} 
	\end{center}
	\caption{Herramienta de análisis estático  }
	\label{fig:desingInteger}
\end{figure}

% \begin{figure}[t!]
% 	\begin{center}
% 	\includegraphics[width=9cm]{desingInOut-3.jpg}
% 	\end{center}
% 	\caption{Herramienta de análisis estático. Ilustra entradas y salidas
% 	esperadas}
% 	\label{fig:desing1}
% \end{figure} 

%color-question para Martín
\textcolor{blue}{ \textit{P: desde este parrafo hasta el final de la sección,
está bien dejar esa información así, o sencillamente la podemos omitir?}}\newline
Luego, la estrategia de evaluación, consiste en verificar si la herramienta
identifica pérdida de información mediante detección de flujos implícitos. Esto
debido a que, como se menciona en la descripción del problema, parte importante
de las propuestas para detección de fuga de información en aplicaciones Android,
hacen data-flow analysis aplicando técnicas de análisis tainting, y en contraste
con las técnicas de análisis de flujo de información, las técnicas de análisis
tainting no necesariamente consideran flujos implícitos. Por tanto, al estar
basada en JIF, cuyo enfoque de análisis es precisamente flujo de información, se
esperaría que la herramienta propuesta esté en capacidad de reconocer flujos
implícitos.

% Se esperaría que: al realizar análisis de flujo de información aplicando
% técnicas Security-Typing, la herramienta propuesta, esté en capacidad de
% reconocer flujos implícitos.\newline
% Más específicamente, se puede tomar el conjunto de aplicaciones utilizadas como
% casos de prueba para la detección de flujos implícitos en
% DroidBench\cite{DroidBenchBenchmarks}, el benchmark de Flowdroid, y analizarlas
% con la herramienta propuesta.\newline

Más específicamente, se puede partir de DroidBench\cite{DroidBenchBenchmarks},
el benchmark de Flowdroid[\ref{FlowDroid-Tool}], tomar el conjunto de
aplicaciones con que prueban la detección de flujos implícitos, y analizarlas
con la herramienta propuesta.\newline
Finalmente estos resultados serían
comparados con los obtenidos mediante otras herramientas para análisis de fuga
de información en aplicaciones Android.

En este orden de ideas, la evaluación de la herramienta propuesta está enfocada
en: medir recall frente a la detección de flujos implícitos, es decir, medir que
no genere falsos negativos ante la existencia de fugas de información,
provenientes de flujos implícitos.\newline




















 
\label{ch:desing}
\chapter{Diseño e Implementación}

\section{Limitaciones técnicas}
\label{sec:limitaciones}
Como parte del diseño de la solución, se inicia con una etapa exploratoria. En
esta se anotan manualmente varias aplicaciones de Android, y se identifican
limitaciones del lenguaje Jif para anotar código de la API de Android.
Tales limitaciones son adicionales a las características del lenguaje Java no
reconocidas por Jif, a continuación se describen tanto las encontradas, como las
especificadas en el manual de referencia de Jif.

-\textit{Características del lenguaje Java no soportadas por jif}\newline
Si bien, el sistema de anotaciones de Jif hace extensiones al lenguaje java,
permitiendo la evaluación de políticas de confidencialidad e integridad para
aplicativos implementados en dicho lenguaje, el manual de referencia de Jif
precisa las características del lenguaje Java no soportadas\cite{jifRef}. Estas
son:
\begin{itemize}
  \item nested classes: clases que son definidas dentro de otras clases.
  \item initializer blocks: bloques de código declarados dentro de la clase pero
  sin pertenecer a ningún método, dependiendo de si se trata de static
  initialization blocks, su código es el primero en ejecutarse, una vez se
  carga la clase; o si se trata de instance initialization blocks, su código se
  ejecutan cada vez que se crea una instancia de la clase.
\item threads.
\end{itemize} 
Partiendo de estas precisiones, aplicaciones Android que presenten tales
características son excluidas del grupo de aplicaciones a analizar(conjunto de
aplicaciones evaluables) mediante la herramienta propuesta.

% Adicional a las limitaciones de jif frente a características propias del
% lenguaje Java, tras experimentar la anotación manual de una serie de
% aplicaciones Android, se identifican varias limitaciones técnicas para la
% anotación de de las mismas. Entre las limitaciones identificadas están:\newline 

- \textit{Soporte para sobreescritura de métodos}\newline 
En la construcción de aplicaciones Android, según el componente que se esté
implementando(activities, content providers, receivers, services), se requiere
sobreescribir métodos de la clase que extienda el componente. Así, cuando se
define un componente tipo Activity, que debe extender de la clase Activiy.java, 
se sobreescriben métodos como Oncreate. Cada que se sobreescribe
un método se utiliza el statement @Override, con el cual se informa al
compilador de Java que el método es sobreescrito. No obstante, al implementar la
versión Jif de aplicaciones Android con dicho statement, el compilador de Jif
no lo reconoce. La dificultad que se presenta está en el reconocimiento del
statement(carácter @ y clase Override), y no en la sobreescritura de métodos,
puesto que Jif soporta tal característica. El soporte para la sobreescritura de
métodos es confirmado con una sencilla prueba, anotando la clase Activity.java
del framework Android (con un único método, el método Ocreate), e implementando
la versión Jif de una aplicación Android que extiende de tal clase, en la cual
se define una actividad y sobreescribe el método Oncreate.
Cuando se comenta la sentencia @Override, el compilador de Jif identifica la
sobreescritura del método y reporta comentarios para el flujo de información.\newline 
Al investigar el por qué Jif no reconoce tal sentencia, se encuentra que dentro
de las clases Java estándar soportadas por el compilador de Jif, no está
contenida la clase java.lang.Override.\newline
Las clases Java estándar pertenecientes a los paquetes io, lang, math, net y
sql; para las que el compilador Jif brinda soporte, vienen implementadas con
anotaciones en el directorio sig-src, directorio que forma parte de la
distribución del compilador de Jif con que se esté trabajando.\newline
Una alternativa para permitir el análisis de flujo de información entre métodos
que se sobreescriben, es comentar las líneas del programa que contengan la
sentencia @Override, puesto que, al no ser reconocida por el compilador de Jif,
es la generadora de errores de compilación.

- \textit{Casting entre tipos EditText y View}\newline
El framework de Android cuenta con diferentes clases para manejar las interfaces
gráficas que presenta al usuario, entre las cuales se encuentran EditText y
View. View es la clase principal para la creación de widgets, necesarios para la
implementación de componentes interactivos en las interfaces de usuario(UI).
EditText permite adicionar campos de texto editables en UI. El casting entre los
tipos de datos que representan ambas clases, se hace cuando la aplicación debe
procesar datos provenientes de campos en las interfaces del usuario, por ejemplo
como se observa a continuación:
\begin{lstlisting}
EditText editPassword = (EditText)findViewById(R.id.password);
String password = editPassword.getText().toString();
\end{lstlisting}
la interfaz de usuario(que es de tipo View) contiene un campo R.id.password, y
para manipular la información que almacena, debe ser de tipo EditText, siendo
necesario un casting de tipo View a tipo EditText. La dificultad que se presenta
con este tipo de casting es que para el sistema de anotaciones de jif no es
válido. Luego de probar con la anotación manual de ambas clases, tratando de
dar soporte a este tipo de casting, sin obtener resultados satisfactorios, se
opta por ``simular'' estos casos, es decir, si el tipo de dato de una variable
es de tipo EditText, se crea una variable tipo String con un valor determinado,
así se omite el casting y se puede analizar el flujo de información.

- \textit{Clase nested R}\newline
El framework de Android utiliza identificadores para hacer referencia a recursos
utilizados por la aplicación, recursos como strings, widgets y layouts, tales
identificadores son autogenerados en la clase R.java, allí cada recurso es
descrito como una clase individual. Al tratarse de una clase nested, la clase R
no puede ser anotada con jif. Esto puede solucionarse implementando una
versión Jif generalizada de la clase R, que contenga los recursos utilizados en
una aplicación, definidos como variables y no como clases.

- \textit{Sources y Sinks}\newline
En los preliminares para el diseño de la solución se propone utilizar SuSi
para clasificar los sources y sinks en las aplicaciones a analizar, sin embargo, partir del
extenso conjunto de sources y sinks, clasificados por SuSi para la API de
Android, implica una mayor complejidad en el análisis, puesto que, en un
aplicativo todo el código que le conforma puede hacer parte de sources o de
sinks. Adicional a lo complejo que se puede tornar el análisis, los sources y
sinks a considerar deben depender de la política de seguridad a evaluar. En ese
orden de ideas, resulta más viable tomar un subconjunto del listado proveído por
SuSi, partiendo de los sources y sinks que evalué la política de seguridad que
se defina.

- \textit{Enhanced for loop} \newline
Además de soportar la estructura de control for, el lenguaje Java permite el uso
de enhanced for, que es utilizado para simplificar la iteración en arrays y
colecciones, por ejemplo:
% \begin{lstlisting}
% char c[] = imei.toCharArray();
% for (int i = 0; i < c.length; i++) {
% 	obfuscated += c[i] + "_";
% }          
% \end{lstlisting}  
\begin{lstlisting}
for(char c : imei.toCharArray())
obfuscated += c + "_";
\end{lstlisting}
A diferencia de Java, Jif no soporta el enhanced for.\newline
Debido a que ambas sentencias de control son equivalentes, la solución que se
propone para poder analizar flujo de información en los aplicativos Android que
contengan dicha estructura de control, es generar el equivalente del programa
haciendo uso del for, de este modo se cambia la sintaxis sin afectar la lógica
del aplicativo a analizar.

- \textit{Otras limitaciones} \newline
Adicional a las limitaciones descritas anteriormente, para las cuales se propone
una solución, se identifica que Jif no soporta la sintaxis utilizada para
definir estructuras de datos HashMaps, LinkedList y Sets, que en Java se definen
de la siguiente manera:
\begin{lstlisting}
Map<String,String> hashMap = new HashMap<String, String>();
List<String> listData;
Set<String> phoneNumbers = new HashSet<String>();
\end{lstlisting}
Jif tampoco permite la definición de interfaces como argumentos de un método. El
siguiente fragmento de código  en una aplicación Android, muestra la definición
de una interfaz pasada como parámetro al método setOnClickListener.
\begin{lstlisting}
   Button button1= (Button) findViewById(R.id.button1);
   button1.setOnClickListener(new View.OnClickListener() {
   ...
   .....}  });
\end{lstlisting}
En estos casos, la dificultad está en encontrar una sintaxis que permita obtener
la versión equivalente del programa que las contenga. A lo que se suma, la falta
de documentación disponible para solventar los mismos. En consecuencia, se omite
del conjunto de aplicaciones evaluables, aplicaciones Android que requieran en
su implementación las estructuras de datos descritas.\newline

%- \textit{Paso de statements dentro de los argumentos de un método
%(\{\}):}\newline

\section{Diseño de la solución}

\textcolor{red}
{Pregunta: los siguientes tres parrafos explican que se cambió la propuesta
incial y por qué. Es adecuado clarificarlo aquí, o se debe actualizar la
propuesta inicial acorde al diseño real? Nota: Sandra recomienda que discutamos
si los cambios han sido en el diseño o en la implementación. De ello depende
dónde deben ir, es decir, si se cambian los preliminares para el diseño de la
solución o si se dejan aquí.}

En los preliminares para el diseño de la solución se consideró la siguiente
opción: Anotar un conjunto de clases de la API de Android mediante el sistema de
anotaciones de Jif, de modo que el compilador de Jif reconociera clases propias
de esa API, y por tanto, permitiese el análisis de flujo de información a través
de estas. Habiendo asegurado el reconocimiento a un
conjunto de clases de la API de Android, era tarea del desarrollador implementar
la versión Jif del aplicativo a evaluar.

Si bien, con dicha opción de diseño se aporta para que el desarrollador
Android evalúe flujos de información en su aplicación, mediante Jif; también se
incrementa su carga de programación, puesto que, al delegarle la anotación de la
aplicación a analizar, este debe pensar dos versiones. La versión .java, donde
utiliza los métodos proveídos por la API Android para definir las
funcionalidades de su aplicación; y la versión .jif, donde define las anotaciones pertinentes para
evaluar flujos de información; tarea para la cual, se requiere un conocimiento
previo del sistema de anotaciones de Jif y la implementación de aplicaciones
haciendo uso de las mismas.

En consecuencia, se opta por un diseño en que el análisis del aplicativo no
implique carga de programación adicional para el desarrollador, sino que por el
contrario, facilite la labor del análisis.\newline 
El diseño de solución consta de dos elementos fundamentales: anotaciones a la
API de Android, más el anotador que genere la versión Jif del aplicativo a
analizar, acorde a una política de seguridad previamente definida.\newline
Ambos elementos son complementarios, puesto que, por más que se genere la
versión Jif del aplicativo a analizar, si el compilador no reconoce clases
específicas de la API que allí se usan, .jif no puede ser compilado.

Así, (1) se parte definiendo la política de seguridad a evaluar
\ref{subsection:politica}, (2) se toman a consideración elementos influyentes
para verificar el cumplimiento de la política mediante Jif
\ref{subsec:consVerPol}, y finalmente, teniendo en cuenta (1) y (2), se
especifican anotaciones a la API de Android \ref{subsec:apiAnnotations} y se
definen los lineamientos del anotador \ref{subsec:anotador}.

\subsection{Definición de la política de seguridad}
\label{subsection:politica}
Detectar si una aplicación Android(perteneciente al conjunto evaluable) presenta
flujos de información entre, información con nivel de seguridad alto e
información con nivel de seguridad bajo.\newline
Detectando fugas de información catalogada con nivel de seguridad alto, vía:
canales creados durante el control de flujo del programa(flujos implícitos),
mensajes de texto y mensajes de Log.\newline 

\subsection{Consideraciones para verificar el cumplimiento de la política
mediante Jif} 
\label{subsec:consVerPol}
Teniendo definida la política de seguridad a verificar, se describen
elementos influyentes en el diseño de la solución.

\textit{Versión de la API de Android}\newline
Las aplicaciones utilizadas para los experimentos previos a la implementación
del prototipo, y las aplicaciones a anotar con el mismo, se realiza partiendo de
la versión Android 4.2.2(API Level 17).

\textit{Versión del compilador de Jif}\newline
se parte de la versión 3.4.2 del compilador de jif, para llevar a cabo tanto los
experimentos previos como el análisis de las aplicaciones anotadas por el prototipo.

\textit{Diferencia entre una aplicación Android y una aplicación Java
convencional}\newline 
En esencia, una aplicación Android es una aplicación Java con interfaces
descritas en XML, que para ser ejecutada necesita del framework de Android,
porque este le provee acceso al hardware del dispositivo y funcionalidades del
sistema.\newline 
Por otro lado, Jif permite hacer seguimiento al flujo de información de una
aplicación Java, extendiendo el lenguaje mediante labels de seguridad.\newline
Para analizar flujo de información de una aplicación Android mediante
Jif, es importante mencionar que mientras una aplicación Java convencional
cuenta con un único punto de entrada para iniciar su ejecución, esto es, la
clase principal donde se implementa el método main; una aplicación Android puede
tener más de un punto de entrada, generados a partir de los diferentes tipos de
componentes que le integren(Activity, Service, Content Provider y Broadcast
Receiver). La necesidad de interacción del usuario para activar tales puntos de
entrada varía acorde al tipo de componente, así, en el caso de componentes tipo
Activity su ejecución sólo inicia hasta que el usuario interactúe con la
actividad, y para manejar dicha interacción, la API Android provee el método
OnCreate. De otro modo, componentes tipo Service y Broadcast Receiver, inician
su ejecución a través de los métodos OnStartCommand y OnReceive,
respectivamente, sin necesidad de interacción del usuario.\newline 
{ \color{black} {Teniendo en cuenta lo anterior, se asume que la aplicación a
evaluar tiene un único punto de entrada, que depende del tipo de componente que
implemente.} }

\textit{Chequeo de excepciones tipo Runtime}\newline
En lenguaje Java las excepciones tipo Runtime tales como NullPointerException, no
son verificadas a tiempo de compilación, sin embargo, buscando evitar la
generación de canales encubiertos mediante las mismas, Jif si las verifica. 
En consecuencia, si un programa requiere excepciones NullPointerException,
ClassCastException y/o ArrayIndexOutOfBoundsException, el programador debe
declararlas en el programa, de lo contrario, el compilador de Jif genera error.
Para las aplicaciones a analizar, se espera que el desarrollador haya
especificado las excepciones necesarias.

\textit{Información considerada con nivel de seguridad alto}\newline
Para verificar el cumplimiento de la política de seguridad a evaluar se parte de
un conjunto de sources, caracterizados por dar a conocer información del usuario, 
considerada como privada o sensible. Los métodos que integran el conjunto de sources son: 
getDeviceId, getSimSerialNumber, findViewById, getLatitude, getLongitude y
getSubscriberId. Adicional a estos métodos, se incluye el tipo de dato EditText,
si y sólo si, el campo UI al que referencia corresponde a un campo tipo
textPassword, es decir, un campo que almacena contraseñas.

\textit{Canales que muestran información con nivel de seguridad bajo}\newline
La información enviada a través de mensajes de texto y la información conocida
tanto a través de mensajes de Log, como a través de canales generados por el
control de flujo del programa, tiene en común que debe poder ser conocida por
terceros. En consecuencia, se considera que estos canales deben dar a conocer
información con nivel de seguridad bajo.\newline
En el caso de mensajes de texto y mensajes de Log, se hace referencia
específicamente a las clases Log y SmsManager de la API de Android.

\textit{Evaluación del flujo de información}\newline
Para evaluar el flujo de información, se asume que todos los métodos definidos
en la clase serán invocados, y por tanto, todos son incluidos en el análisis.\newline 
Esta decisión de análisis busca evitar el paso inadvertido de flujos de
información, generados por omisión.

\textit{Acceso a métodos de sobreescritura.}\newline
Los métodos de las clases Activity, Service y BroadcastReceiver, son métodos
que se pueden sobreescribir, todo programa Android que extienda de tales clases
debe poder utilizarlos.

\subsection{Cómo funciona el sistema de anotaciones en Jif(Para
Justificar, las anotaciones propuestas)(Ubicación temporal)}
Jif es un lenguaje tipado de seguridad que extiende al lenguaje Java con labels
de seguridad, a través de los cuales se especifican restricciones de cómo
debería ser utilizada la información.\newline 
Jif está compuesto por un compilador y un sistema de anotaciones.\newline
% El sistema de anotaciones está basado en un modelo de etiquetas
% DLM(Decentralized Label Model), donde existen principals, políticas y
% labels.\newline
Para hacer seguimiento al control de flujo de información de un programa
implementado en Java, se debe implementar la respectiva versión Jif, es decir,
la versión del programa donde mediante el sistema de anotaciones se especifican
las políticas de seguridad a evaluar.\newline 
Luego, la versión Jif del programa se compila con el compilador de Jif.
Este aplica chequeo de labels(label checking)\cite{jifRef} para verificar que
los flujos de información que se generan al interior del programa, cumplen con
las políticas definidas.\newline

-DML(Decentralized Label Model)\newline
Jif basa su sistema de anotaciones en el modelo de etiquetas DLM(Decentralized
Label Model), donde se manejan tres elementos fundamentales: Principals,
Políticas y Labels.\newline
Principals: un principal es una entidad con autoridad para observar y cambiar
aspectos del sistema. Jif cuenta con una serie de principals ya
definidos, por ejemplo, Alice, Bob, Chunck, etc, que pueden ser
utilizados al momento de anotar.\newline 
Políticas: mediante políticas de seguridad el dueño de la política, que es el
principal que la define, determina qué otros principals pueden leer o
influenciar la información. Así, una política puede ser de confidencialidad o de
integridad, y se especifican de la forma: \{owner: reader list\} u
\{owner: writer list\}.\newline 
Labels: un label consiste en un conjunto de políticas de confidencialidad e
integridad. Los labels se escriben en las expresiones del progrma que se
anota(labels de seguridad), esto es métodos, variables, arrays, etc..\newline 
En sisntesís, las políticas de seguridad definen que principals pueden leer o
modificar la información, y esas políticas se expresan mediante labels.\newline

- Label Checking\newline
Para hacer seguimiento al flujo de información de un programa, el compilador de
Jif asocia un label al program counter de cada punto del programa,
progam-counter label(\underline{pc}). En cada punto del programa, el
(\underline{pc}) representa la información que podría conocerse tras la
ejecución de ese punto del programa.
El (\underline{pc}) es afectado por los labels con que se define cada sentencia
y expresión del programa, por tanto este es considerado como el límite
superior(máxima información que podría conocerse) de los labels que han afectado
el flujo de información para llegar a un determinado punto de ejecución.\newline
Adicionalmente, jif define labels que representan la información que podría
conocerse tras la terminación normal, o terminación por excepción de las
sentencias del programa. Y labels enviroments, que para cada punto del programa
determinan la forma en que se relacionan labels y principals.\newline
El valor dichos labels es verificado durante la compilación del programa, si se
detecta que no cumplen con las restricciones establecidas en la anotación del
mismo, el compilador genera error, indicando los puntos del programa que las
incumple.\newline

- Sintaxis de labels y sintaxis para anotación de sentencias en Jif\newline

En Jif, cada expresión del programa está conformada por dos partes: java type +
security labels un label.\newline

-Llamada a métodos\newline
-Implicit flows and program-counter labels\newline
- sobreescritura de métodos

\subsection{Lineamientos de anotación}
Para verificar el cumplimiento de la política de seguridad
establecida\ref{subsection:politica} mediante el sistema de anotaciones de Jif,
se requiere: (a) definir los elementos básicos de anotación, (b) definir
las anotaciones necesarias para la API de Android y (c) definir los
criterios de anotación para los aplicativos a analizar, tales
criterios se sintetizan en un anotador.

\textbf{(a)Elementos básicos de anotación}\newline
En \ref{subsec:consVerPol} se definió qué \textit{Canales muestran
información con nivel de seguridad bajo} y cuál es la \textit{Información
considerada con nivel de seguridad alto}. Ahora, para anotar la información
catalogada con uno u otro nivel de seguridad, de modo que, partiendo de tales
anotaciones se evalué la existencia de flujos de información entre información
con nivel de seguridad alto e información con nivel de seguridad bajo, lo primero
que se debe definir es quíen representa la autoridad principal del sistema(Top
Principal) y cuáles son los labels de seguridad.\newline 
Top principal (Alice): aprovechando que Jif ya trae una serie de principals
establecidos, se define que la autoridad máxima es el principal Alice. Este
principal tendrá todo el poder para actuar sobre los aspectos del
sistema.\newline 
Label de seguridad con nivel de seguridad alto (\{Alice:\}): en base al top
principal, se define el label de seguridad que expresa el mayor nivel de
seguridad. Las variables con nivel de seguridad alto deben ser anotadas
con tal label de seguridad, ya que esté epecífica que la información
contenida en la variable solámente podrá ser Accedida por el Top
principal.\newline 
Label de seguridad con nivel de seguridad bajo (\{\}): se define un label de
seguridad que expresa el menor nivel de seguridad, para anotar información que
es de nivel de seguridad bajo y que por tanto puede ser de conocimiento público.


\textbf{(b)Anotaciones a la API de Android}\newline

\begin{figure}[h!]
	\begin{center}
	\includegraphics[width=12cm]{annotationsMechanims.jpeg}
	\end{center}
	\caption{Mecánismos de anotación para clases de la API.}
	\label{fig:annotationsMechanims}
\end{figure}

- Anotaciones para \textit{Canales que muestran información con nivel de
seguridad bajo}: para controlar el flujo de información que se envía hacia
\textit{Canales que muestran información con nivel de seguridad bajo}, es
necesario anotar la definición de los métodos de las clases Log y SmsManager de
la API Android, de manera tal, que se controle el nivel de seguridad de los
argumentos con que se invocan. Por consiguiente, en la definición del método el
AL del parametro que recibe la información a mostrar(logs) o la información a
enviar(sms), se anota con el label de seguridad bajo(label público\{\}). Con
esto se garantiza que la información se muestra o envía, si y sólo si el nivel
de seguridad del argumento con que se invoca el método es público. Por ejemplo,
si el método se llama con información anotada con label de seguridad alto, se
genera error en la compilación del programa que le llama.\newline 
El resto de labels del método BL y AL se anotan con el top principal. Al anotar
el BL con el top principal, se permite que el método sea invocado desde
cualquier punto de un programa. 
Para el resto de labels de los argumentos XXX\newline
% cualquier punto de un programa que sea igual o menor de restrictivo que el
% principal Alice, esto se traduce en que podrá ser invocado desde cualquier punto
% de un programa, lo cual es correcto porque lo que se busca el evitar que se
% envíe información considerada con nivel se deguridad alto y no, evitar que el
% método sea utilizado. Poner ejemplo XXXX?\newlie
- Anotaciones para metódos de sobreescritura: en \ref{subsec:consVerPol},
también se definieron las clases de la API para las que se debe soportar el
\textit{Acceso a métodos de sobreescritura}(Activity, Service y
BroadcastReceiver). La anotación para la definicíon de tales métodos se basa en
los siguientes criterios: (1)reglas de Jif para la sobreescritura de métodos y
(2)desde dónde pueden ser invocados. (1) Jif exige que el nivel de seguridad del
BL del método desde donde se invoca el método a sobreescribir, no debe ser menos
restrictivo que el BL de la definición de tal método(Cómo se traduce qué es más
restrictivo y que es más restrictivo? Definiciones previas).
(2) la sobreescritura de métodos se debe poder utilizar desde cualquier aplicativo que
extienda de las clases Activity, Service y BroadcastReceiver.\newline
Para cumplir con (1) y (2), los métodos que requieren ser sobreescritos se
definen con BL público(\{\}). De este modo los aplicativos desde dónde se
invocan los métodos a sobreescribir deben tener igual BL.

- Adicional a las clases de la API (Log, SmsManager, Activity, Service y
BroadcastReceiver), es necesario brindar soporte a un conjunto de clases que
representan librerias importadas por los aplicativos a analizar.(Estas
son)Brindar soporte significa que deben ser reconocidas por el compilador Jif.

% Y para poder integrarlas al conjunto de clases reconocidas por el compilador
% de Jif, basta con recurrir a signaturas nativas. Mediante el uso de signaturas
% nativas es posible incluir clases Java ya existentes. Tal mecanismo consiste en
% implementa una versión Jif de las clase fuentes, en la qué sólo es necesario
% declarar constructores y cuerpo de los métodos a utilizar.
% 
% Ahora, para incluir clases Java ya existentes es posible recurrir a signaturas
% nativas, con las cuales se implementa la versión Jif de las clases fuentes, esto
% es, declarando constructores y cuerpo de los métodos a utilizar.
- Integración de clases de la API de Android a las clases reconocidas por el
compilador de Jif:\newline
Definidos los criterios de anotación para las clases de la API, a las cuales se
debe implementar su respectiva versión Jif. Se definen los mécanismos a utilizar
para implementar tal versión. Además del mecanismo de anotación completa en que
se basa la implementación de aplicativos en Jif(mecanismo de anotación), el
compilador provee un mecanismo que permite reutilizar código de clases Java ya
existentes, para esto, se recurre a signaturas nativas. Así, se implementa una
versión Jif de una clase Java ya existente, en la que se declaran signaturas
nativas proveídas por Jif, constructor y métodos necesarios de la
clase(mecanismo de signaturas nativas).
Dependiendo de las implicaciones que pueda tener para el análisis de flujo de información del
sistema, a la versión simplificada puede específicar u omitir labels de
seguridad(mecanismo de signaturas nativas más labels de seguridad).\newline 
Para el presente caso y de acuerdo a los criterios de anotación previamente
establecidos las clases a implementar mediante uno u otro mecanismo, son
ilustrados en la siguiente figura \ref{fig:annotationsMechanims}.\newline
Nota: Falta decir qué las anotaciones a la API se implementan manualmente.

\textbf{(c)Criterios de anotación de los aplicativos a analizar}\newline
- Aquí se definen los criterios de anotación del aplicativo a analizar, y
partiendo de estos se define un algoritmo de anotación.\newline
- Al final describir exactamente las entradas y salidas del anotador, que es
donde se condensa el algoritmo de anotación.\newline

XXXPara evaluar los flujos de información de una aplicación Android, de modo que
se verifique si cumplen con la política de seguridad definida
\ref{subsection:politica}; es necesario implementar su respectiva versión Jif,
esto implica que variables y métodos de la clase sean anotados haciendo uso del
sistema de anotaciones de Jif.\newline 
Ante esto, se propone un esquema de anotación enfocado a detectar flujos de
información desde: información considerada con nivel de seguridad alto,
hacia: canales considerados con nivel de seguridad bajoXXX.\newline 

Los elementos en que se fundamenta el esquema de anotación para el aplicativo a
analizar, son los siguientes:\newline 
Primero, como los \textit{Canales que muestran
información con nivel de seguridad bajo} están anotados desde la definición en
sus respectivas clases de la API, de modo que la información a mostrar(logs) o
enviar(msn) debe tener nivel de seguridad bajo(debe ser pública), en el esquema
de anotación del aplicativo no es necesario identificar tales canales, pues
estos ya están definidos desde la API. Lo que si se debe hacer desde el esquema
de anotación del aplicativo, es proveer los labels de seguridad adecuados a la
información con que se invocan tales canales.\newline
Segundo, la anotación para la sobreescritura de métodos de la API, está definida
en las respectivas clases de la API, el esquema de anotación para la invocación
de los métodos a sobreescribir debe ser consecuentes con tales definiciones de
anotación.\newline
Segundo, el esquema de anotación para el aplicativo a analizar parte de
los sources que se identifiquen en el mismo. Tales sources pertenecen al conjunto
definido en (\textit{Información considerada con nivel de seguridad alto}),
conjunto que está integrado por el tipo de dato EditText\footnote{Este tipo de
dato es considerado como source si y sólo si, el campo UI al que referencia
corresponde a un campo tipo textPassword, es decir, un campo que almacena
contraseñas.} y los métodos: getDeviceId, getSimSerialNumber, findViewById,
getLatitude, getLongitude y getSubscriberId.\newline 
Cuarto, se parte de los \textit{(a)Elementos básicos de anotación} prevíamente
definidos. Así, una clase Android tendrá como autoridad máxima(top principal) al
principal Alice, y los labels de seguridad para anotar información con nivel de
seguridad alto y nivel de seguridad bajo son: (\{Alice:\}) y (\{\}),
respectivamente.\newline
Quinto, enfoque de la anotación.\newline
Conceptos:\newline
Variable source: una variable source es una variable que almacena información
proveída por algún elemento del conjunto de sources. Una variable source, tiene
nivel de seguridad alto.\newline
Array source: un array source es un array que alamcena información de variables
source.\newline
Método source: un método source es un método(definido dentro de la clase)
que cuando es invocado, contiene dentro de los parámetros de invocación, al menos
una varible source. Como el método influencia la información de la variable
source, cuyo nivel de seguridad es alto, la información a conocer tras el
llamado del método debe tener un nivel de seguridad alto.\newline 
Método no source: un método no source es un método(definido dentro de la
clase) que cuando es invocado, no contiene dentro de los parámetros de
invocación, varibles source. Como el método no influencia la información
de la variable source, el nivel de seguridad de la información que se obtenga
con el método depende del nivel de seguridad de información de sus argumentos.\newline 
El esquema de anotación se centra en identificar si una clase contiene variables
source, y verificar si los métodos de la clase influencian la información de
esas variables, de modo que, cuando la información sea envíada por canales con
nivel de seguridad bajo, tenga el nivel de seguridad adecuado.\newline
Las variables y métodos que no son anotadas mediante este esquema, quedan con
los labels de seguridad que Jif define por defecto para cada caso.(estos son).


Partiendo del enfoque del esquema de anotación se precisan las definiciones de
anotación y se definen los pasos de anotación:\newline
\textit{Definición A:}\newline
Anotación de variable source: como las variables source tienen nivel de
seguridad alto, se anotan con el label se guridad \{Alice:\}, ya que este
indica que la información sólo puede ser conocida por la autoridad máxima del
sistema(se trata de información privada).\newline
\textit{Definición B:}\newline
Anotación de array source\newline
Los labels de seguridad para definir un array source son:\newline
modifier type\{Alice:\}[ ]\{Alice:\}\newline
\textit{Definición C:}\newline
Anotación de método que se sobreescribe: el BL de los métodos a sobreescribir(de
la API) deben ser anotados con label de seguridad bajo \{\}, puesto que en su respectiva
definición en las clases de la API, han sido definidas con ese BL.\newline
\textit{Definición D:}\newline
Anotación de método source: un método source se define con los siguientes
labels de seguridad:\newline
modifier type nameMethod\textit{\{Alice:\}} type\textit{\{Alice:\}}\newline
\textit{Definición E:}\newline
Anotación de método no source: \newline
nameMethod\textit{\{\}}(type\textit{\{Alice:\}}arg1,.....type\textit{\{Alice:\}}argn ) \{\}\newline

Pasos para la anotación\newline
(1) Identificar variables source de la clase. Si se encuentran variables sources
continuar con los pasos (2) a (11), sino, continuar con pasos: (3), (6) y (9).
definiciones B y D.\newline
(2) Identificar arrays sources de la clase.
(3) Identificar el total de métodos de la clase.\newline
(4) Del total de métodos listar los métodos source.\newline
(5) Del total de métodos listar los métodos no source.\newline
(6) Del total de métodos listar los métodos a sobreescribir\newline
(7) Aplicar definición D a listado del paso(4).\newline
(8) Aplicar definición E a listado del paso (5).\newline
(9) Aplicar definición C a listado del paso (6).\newline
(10) Aplicar definición B a listado del paso (2).\newline
(11) Aplicar definición A a listado del paso (1).\newline

% \subsection{Anotaciones a la API de Android}
% \label{subsec:apiAnnotations}
% 
% \begin{figure}[h!]
% 	\begin{center}
% 	\includegraphics[width=5cm]{desingSol-steps1.jpg}
% 	\end{center}
% 	\caption{Tipos de anotación necesarias para implementar la solución.}
% 	\label{fig:desingSol-steps1}
% \end{figure}
% 
% \begin{figure}[t!]
% 	\begin{center}
% 	\includegraphics[width=12cm]{desingSol-step1-details.jpg}
% 	\end{center}
% 	\caption{Clases de la API a anotar.}
% 	\label{fig:desingSol-details}
% \end{figure}
% 
% El compilador de Jif tiene implementadas mediante el sistema de anotaciones Jif,
% las clases estándar del lenguaje Java para las que brinda soporte.\newline
% Ahora, para incluir clases Java ya existentes es posible recurrir a signaturas
% nativas, con las cuales se implementa la versión Jif de las clases fuentes, esto
% es, declarando constructores y cuerpo de los métodos a utilizar.
% 
% Para dar soporte a clases específicas de la API de Android se recurre a ambas
% opciones de anotación. Tal como se ilustra en la figura
% \ref{fig:desingSol-steps1}.\newline 
% El criterio para decidir que se anota de una u otra forma, depende de lo que
% represente la clase Android para verificar la política de seguridad
% establecida.\newline
% Las clases Log y SmsManager, que representan canales para conocer información,
% son anotadas de forma no nativa.\newline
% La opción de anotación nativa se utiliza
% para librerías Android, por ejemplo la clase TelephonyManager necesaria para utilizar
% el método getDeviceId.\newline
% En la figura \ref{fig:desingSol-details} se
% especifican clases de la API Android a anotar.\newline

\subsection{Diseño del anotador}
\label{subsec:anotador}
\label{subsec:pasosSol}
\begin{figure}[H]
	\begin{center}
	\includegraphics[width=7cm]{desingSolution.jpg}
	\end{center}
	\caption{Entradas y salidas para el generador de anotaciones.}
	\label{fig:desingSolution} 
\end{figure}

El anotador debe recibir como entrada el código fuente de la aplicación
Android(perteneciente al conjunto evaluable), para retornar su respectiva
implementación en Jif, versión que contiene las anotaciones para evaluar la
política de seguridad definida. Tal como se ilustra en la figura
\ref{fig:desingSolution}\newline 
Luego la versión obtenida se evalúa con el compilador de Jif.

A continuación, se definen los lineamientos de anotación:\newline
% \textit{Paso dos: Definir Autoridades y forma de anotación del aplicativo
% Android a analizar.}\newline
Una clase Android tendrá una Autoridad máxima(un principal), en este caso Alice,
así que, información con nivel de seguridad alto deberá pertenecer a dicha
autoridad.\newline
Jif hace seguimiento al flujo de información del programa, asociando un label
de seguridad al program counter de cada sentencia y expresión del programa,
program counter(pc) label. Este pc es afectado por el label de seguridad que se
especifique en la declaración de variables y métodos\ref{subsec:JifSintax}. 
Partiendo de que Jif se fundamenta en labels de seguridad para hacer seguimiento
al flujo de información del programa, es necesario definir los labels a
anotar para métodos y variables del programa.\newline
En el caso de variables con nivel de seguridad  alto, la anotación debe
ser:\newline
\emph{ type\{Alice:\} varName; }\newline
Para el resto de variables, entran a jugar las anotaciones definidas por Jif
acorde al contexto donde están definidas.\newline
Ahora en el caso de los métodos, la anotación varía según si el método debe
influenciar(acceder, modificar) o no, información anotada con nivel de seguridad
alto.

En base a lo anterior, se define un algoritmo de anotación que se condensa en un
generador de anotaciones; y está fundamentado en las siguientes
definiciones:\newline \textit{Definición A:} anotación de variables con nivel de
seguridad alto:\newline modifier type\{Alice:\} varName;\newline 
\textit{Definición B:} métodos que se sobreescriben. El sistema de anotaciones
de Jif exige que el nivel de seguridad del método desde donde se invoca la
sobreescritura de un método, no debe ser menos restrictivo que el método a
sobreescribir, y los métodos a sobreescribir deben poder ser invocados desde
todo programa Android, siguiendo con este principio, y buscando que jif no
limite el flujo de información, estos métodos deben ser anotados con BL
público(\{\}).\newline 
\textit{Definición C:} anotación de métodos con sources\newline
\textcolor{red}{La definición de los terminos etá en: Sintaxis de anotación en
Jif \ref{subsec:JifSintax}, debo recordar lo que significan, o remito el lector
a dicha sección?}\newline
Los labels para la definición del método\textcolor{red}{(BL, EL,
AL )}se anotan de la siguiente manera:\newline 
modifier type nameMethod\textit{\{Alice:\}} type\textit{\{Alice:\}}
( arg1,.....type\textit{\{Alice:\}}argn ) \{\}\newline Si dentro del método se
definen arrays, sus respectivos BL y SL, deben ser anotados así: modifier type\{Alice:\}[ ]\{Alice:\}\newline
\textit{Definición D:} anotación de métodos que no reciben información del
source. 
nameMethod\textit{\{\}}(
type\textit{\{Alice:\}}arg1,.....type\textit{\{Alice:\}}argn ) \{\}\newline Teniendo claras las anteriores definiciones, los pasos para el algoritmo son los
siguiente:\newline
(1) Identificar sources de la clase. Si se encuentran sources continuar con
los pasos (2) a (4), sino, continuar con paso (2) y aplicar definiciones B y
D.\newline
(*) Identificar arrays sources de la clase 
(2) Identificar el total de métodos de la clase.\newline
(3) Del total de métodos listar los que son invocados con el source.\newline
(4) Del total de métodos listar los que no son invocados con el source.\newline
(5) Aplicar definición C a listado del paso(3).\newline
(6) Aplicar definición D a listado del paso (4).\newline
%(7) Aplicar definición B. \newline
(**) Aplicar definición 
(8) Aplicar definición A a listado del paso (1).
(7) Aplicar definición B. \newline
\subsection{Descripción implementación prototipo}
-Diagrama de clases o descripción\newline
En la sección \ref{sec:diagramaClass} de los anexos, se incluye el diagrama de
clases para la implementación del anotador.\newline

El anotador consta de cinco clases 




\chapter{Descripción implementación}

\section{Limitaciones técnicas}
\label{sec:limitaciones}
Antes de iniciar con la implementación del esquema de anotación propuesto, tanto
para las clases de la API \ref{subsec:api}, como para los aplicativos a analizar
\ref{subsec:apps}, se identifican una serie de limitaciones del lenguaje
Jif para anotar código de la API de Android. 
Tales limitaciones son adicionales a las características del lenguaje Java no
soportadas por Jif, a continuación se describen tanto las encontradas, como las
especificadas en el manual de referencia de Jif.

\subsection{Características del lenguaje Java no soportadas por jif}
% -\textit{Características del lenguaje Java no soportadas por jif}\newline
Si bien, el sistema de anotaciones de Jif hace extensiones al lenguaje java,
permitiendo la evaluación de políticas de confidencialidad e integridad para
aplicativos implementados en dicho lenguaje, el manual de referencia de Jif
precisa las características del lenguaje Java no soportadas\cite{jifRef}. Estas
son:
\begin{itemize}
  \item nested classes: clases que son definidas dentro de otras clases.
  \item initializer blocks: bloques de código declarados dentro de la clase pero
  sin pertenecer a ningún método, dependiendo de si se trata de static
  initialization blocks, su código es el primero en ejecutarse, una vez se
  carga la clase; o si se trata de instance initialization blocks, su código se
  ejecutan cada vez que se crea una instancia de la clase.
\item threads.
\end{itemize} 
Partiendo de estas precisiones, aplicaciones Android que presenten tales
características son excluidas del grupo de aplicaciones a analizar(conjunto de
aplicaciones evaluables) mediante la herramienta propuesta.

% Adicional a las limitaciones de jif frente a características propias del
% lenguaje Java, tras experimentar la anotación manual de una serie de
% aplicaciones Android, se identifican varias limitaciones técnicas para la
% anotación de de las mismas. Entre las limitaciones identificadas están:\newline 
\subsection{Soporte para la clase java.lang.Override}
\label{sub:override}
% - \textit{Soporte para sobreescritura de métodos}\newline 
En la construcción de aplicaciones Android, según el componente que se esté
implementando(activities, content providers, receivers, services), se requiere
sobreescribir métodos de la clase que extienda el componente. Así, cuando se
define un componente tipo Activity, que debe extender de la clase Activiy.java, 
se sobreescriben métodos como Oncreate. Cada que se sobreescribe
un método se utiliza el statement @Override, con el cual se informa al
compilador de Java que el método es sobreescrito. No obstante, al implementar la
versión Jif de aplicaciones Android con dicho statement, el compilador de Jif
no lo reconoce. La dificultad que se presenta está en el reconocimiento del
statement(carácter @ y clase Override), y no en la sobreescritura de métodos,
puesto que Jif soporta tal característica.\newline 
El soporte para la sobreescritura de
métodos es confirmado con una sencilla prueba, anotando la clase Activity.java
del framework Android (con un único método, el método Ocreate), e implementando
la versión Jif de una aplicación Android que extiende de tal clase, en la cual
se define una actividad y sobreescribe el método Oncreate.
Cuando se comenta la sentencia @Override, el compilador de Jif identifica la
sobreescritura del método y reporta comentarios para el flujo de información.

Al investigar el por qué Jif no reconoce tal sentencia, se encuentra que dentro
de las clases Java estándar soportadas por el compilador de Jif, no se incluye
la clase java.lang.Override.\newline 
Las clases Java estándar pertenecientes a los paquetes io, lang, math, net y
sql; soportadas po el compilador Jif, vienen implementadas con anotaciones en el
directorio sig-src, directorio que forma parte de la distribución del compilador
de Jif con que se esté trabajando.

Un mecanismo para permitir el análisis de flujo de información entre métodos
que se sobreescriben, es comentar las líneas del programa que contengan la
sentencia @Override, puesto que, al no ser reconocida por el compilador de Jif,
es la generadora de errores de compilación.

\subsection{Casting entre tipos EditText y View}
\label{sec:casting}
El framework de Android cuenta con diferentes clases para manejar las interfaces
gráficas que presenta al usuario, entre las cuales se encuentran EditText y
View.\newline
View es la clase principal para la creación de widgets, necesarios para la
implementación de componentes interactivos en las interfaces de
usuario(UI).\newline 
EditText permite adicionar campos de texto editables en UI.\newline
El casting entre los tipos de datos que representan ambas clases, se hace cuando la aplicación debe
procesar datos provenientes de campos en las interfaces del usuario, por ejemplo
como se observa a continuación:
\begin{lstlisting}
EditText editPassword = (EditText)findViewById(R.id.password);
String password = editPassword.getText().toString();
\end{lstlisting}
la interfaz de usuario(que es de tipo View) contiene un campo R.id.password, y
para manipular la información que almacena, debe ser de tipo EditText, siendo
necesario un casting de tipo View a tipo EditText. La dificultad que se presenta
con este tipo de casting es que para el sistema de anotaciones de jif no es
válido. Luego de probar con la anotación manual de ambas clases, tratando de
dar soporte a este tipo de casting, sin obtener resultados satisfactorios, se
opta por ``simular'' estos casos, es decir, si el tipo de dato de una variable
es de tipo EditText, se crea una variable tipo String con un valor determinado,
así se omite el casting y se puede analizar el flujo de información.

\subsection{Clase nested R}
\label{sec:nested}
El framework de Android utiliza identificadores para hacer referencia a recursos
utilizados por la aplicación, recursos como strings, stilos, widgets, layouts, e
interfaces xml, tales identificadores son autogenerados en la clase R.java, allí
cada recurso es descrito como una clase individual. Al tratarse de una clase
nested, la clase R no puede ser anotada con jif. Esto puede solucionarse
implementando una versión Jif generalizada de la clase R, que contenga los
recursos utilizados en una aplicación, definidos como variables y no como clases.

% - \textit{Sources y Sinks}\newline
% En los preliminares para el diseño de la solución se propone utilizar SuSi
% para clasificar los sources y sinks en las aplicaciones a analizar, sin embargo, partir del
% extenso conjunto de sources y sinks, clasificados por SuSi para la API de
% Android, implica una mayor complejidad en el análisis, puesto que, en un
% aplicativo todo el código que le conforma puede hacer parte de sources o de
% sinks. Adicional a lo complejo que se puede tornar el análisis, los sources y
% sinks a considerar deben depender de la política de seguridad a evaluar. En ese
% orden de ideas, resulta más viable tomar un subconjunto del listado proveído por
% SuSi, partiendo de los sources y sinks que evalué la política de seguridad que
% se defina.

\subsection{Enhanced for loop}
\label{seubsec:enh}
Además de soportar la estructura de control for, el lenguaje Java permite el uso
de enhanced for, que es utilizado para simplificar la iteración en arrays y
colecciones, por ejemplo:
% \begin{lstlisting}
% char c[] = imei.toCharArray();
% for (int i = 0; i < c.length; i++) {
% 	obfuscated += c[i] + "_";
% }          
% \end{lstlisting}  
\begin{lstlisting}
for(char c : imei.toCharArray())
obfuscated += c + "_";
\end{lstlisting}
A diferencia de Java, Jif no soporta el enhanced for.\newline
Debido a que ambas sentencias de control son equivalentes, la solución que se
propone para poder analizar flujo de información en los aplicativos Android que
contengan dicha estructura de control, es generar el equivalente del programa
haciendo uso del for, de este modo se cambia la sintaxis sin afectar la lógica
del aplicativo a analizar.

\subsection{Otras limitaciones}
Adicional a las limitaciones descritas anteriormente, para las cuales se propone
una solución, se identifica que Jif no soporta la sintaxis utilizada para
definir estructuras de datos HashMaps, LinkedList y Sets, que en Java se definen
de la siguiente manera:
\begin{lstlisting}
Map<String,String> hashMap = new HashMap<String, String>();
List<String> listData;
Set<String> phoneNumbers = new HashSet<String>();
\end{lstlisting}
Jif tampoco permite la definición de interfaces como argumentos de un método. El
siguiente fragmento de código  en una aplicación Android, muestra la definición
de una interfaz pasada como parámetro al método setOnClickListener.
\begin{lstlisting}
   Button button1= (Button) findViewById(R.id.button1);
   button1.setOnClickListener(new View.OnClickListener() {
   ...
   .....}  });
\end{lstlisting}
En estos casos, la dificultad está en encontrar una sintaxis que permita obtener
la versión equivalente del programa que las contenga. A lo que se suma, la falta
de documentación disponible para solventar los mismos. En consecuencia, se omite
del conjunto de aplicaciones evaluables, aplicaciones Android que requieran en
su implementación las estructuras de datos descritas.\newline

%- \textit{Paso de statements dentro de los argumentos de un método
%(\{\}):}\newline
\section{Detalles de implementación}
Como se describió anteriormente, la implementación del diseño requiere
anotaciones a la API de Android y anotaciones en los aplicativos a
analizar.\newline 
En el caso de las anotaciones requeridas para la API \ref{subsec:api}, su
implementación se hace manualmente.\newline
En el caso de las anotaciones en el aplicativo a analizar \ref{subsec:apps},
para las que se propuso un esquema de anotación automatizable, se implementa un
prototipo de anotación en lenguaje Java. En la sección \ref{sec:diagramaClass}
de los anexos, se incluye su respectivo diagrama de clases.\newline 
El anotador está implementado en lenguaje Java y consta de cinco clases:
\emph{Main}, \emph{Source}, \emph{XmlExtract}, \emph{Annotation} y
\emph{BufferWriter}.\newline 
Todos los pasos de anotación descritos en
\ref{subsec:pasosSol}, son coordinados desde la clase \emph{Main}.\newline
La clase \emph{Source} identifica qué variables source contiene la aplicación que se
está analizando, esta clase se apoya en la clase \emph{XmlExtract} para identificar
sources definidos en interfaces UI. En este caso particular, para la
verificación de la política establecida, la clase \emph{XmlExtract} permite identificar
el source EditText, cuando se trata de un campo UI que almacena
contraseñas.\newline 
Como las interfaces UI se describen mediante XML, es
necesario extraer los elementos de la interfaz utilizando las librerías javax.xml.parsers y
org.w3c.dom. Con javax.xml.parsers se obtiene el respectivo DOM del XML a
analizar. Con org.w3c.dom se recorren los elementos del DOM, elementos como los
campos EditText.\newline 
Con la clase \emph{Annotation}  se identifican y anotan los diferentes tipos de
métodos(métodos source, no source y métodos a sobreescribir).\\
En la clase \emph{BufferWriter} se anotan las variables sources, y finalmente
se retorna la versión .jif del aplicativo que se está analizando.\newline 
Para la anotación del código se utiliza la librería Java Parser 1.8  que permite
generar y visitar el árbol de sintaxis para la estructura del código de cada
programa, haciendo posible la adición de los labels de seguridad respectivos.\\
% \textcolor{blue}{(Incluir Figura)}\newline 
\label{subsec:anotador}
\begin{figure}[t!]
	\begin{center}
	\includegraphics[width=12cm]{JifFilesystem.jpg}
	\end{center}
	\caption{Estructura de directorios en Jif.
	El código con anotaciones que se requiere para la evaluación de flujo de
	información en Jif, es alojado en los directorios sig-src y jif-src.}
	\label{fig:jifFilesystem} 
\end{figure}

\section{Ambiente y herramientas}
\textit{Versión de la API de Android}\newline
Las anotaciones a las clases de la API Android y las aplicaciones a anotar
mediante el prototipo, corresponden a la versión Android 4.2.2(API Level 17).

\textit{Versión del compilador de Jif}\newline
El componente que verifica el cumplimiento de las políticas de seguridad,
corresponde a la versión 3.4.2 del compilador de Jif.

\textit{Estructura de trabajo en JIF}\newline
El compilador de Jif maneja una estructura de directorios en la que se incluye
el código fuente .jif de las aplicaciones a analizar. Como ilustra la figura
\ref{fig:jifFilesystem} se tienen los directorios principales sig-src y jif-src.
El directorio sig-src está destinado para incluir librerías adicionales.\newline 
El directorio jif-src contiene todo el código jif correspondiente al programa
como tal que se va evaluar mediante Jif. Para los propósitos del presente
trabajo, se tienen los subdirectorios principals y project. En el subdirectorio
principals se incluyen las autoridades requeridas, y en el subdirectorio
project se incluyen, tanto las clases específicas de los aplicativos Android a
analizar, como las clases de que estos heredan como: Activity.jif, Service.jif,
etc.

\section{Compilación, integración y finalmente ejecución }
En la sección \ref{sec:ejecutarPrototipo} de los anexos, se describen las
instrucciones paso a paso de cómo ejecutar el anotador, y cómo compilar los
aplicativos a analizar mediante Jif.

% %\chapter{Descripción implementación}

\section{Limitaciones técnicas}
\label{sec:limitaciones}
Antes de iniciar con la implementación del esquema de anotación propuesto, tanto
para las clases de la API \ref{subsec:api}, como para los aplicativos a analizar
\ref{subsec:apps}, se identifican una serie de limitaciones del lenguaje
Jif para anotar código de la API de Android. 
Tales limitaciones son adicionales a las características del lenguaje Java no
soportadas por Jif, a continuación se describen tanto las encontradas, como las
especificadas en el manual de referencia de Jif.

\subsection{Características del lenguaje Java no soportadas por jif}
% -\textit{Características del lenguaje Java no soportadas por jif}\newline
Si bien, el sistema de anotaciones de Jif hace extensiones al lenguaje java,
permitiendo la evaluación de políticas de confidencialidad e integridad para
aplicativos implementados en dicho lenguaje, el manual de referencia de Jif
precisa las características del lenguaje Java no soportadas\cite{jifRef}. Estas
son:
\begin{itemize}
  \item nested classes: clases que son definidas dentro de otras clases.
  \item initializer blocks: bloques de código declarados dentro de la clase pero
  sin pertenecer a ningún método, dependiendo de si se trata de static
  initialization blocks, su código es el primero en ejecutarse, una vez se
  carga la clase; o si se trata de instance initialization blocks, su código se
  ejecutan cada vez que se crea una instancia de la clase.
\item threads.
\end{itemize} 
Partiendo de estas precisiones, aplicaciones Android que presenten tales
características son excluidas del grupo de aplicaciones a analizar(conjunto de
aplicaciones evaluables) mediante la herramienta propuesta.

% Adicional a las limitaciones de jif frente a características propias del
% lenguaje Java, tras experimentar la anotación manual de una serie de
% aplicaciones Android, se identifican varias limitaciones técnicas para la
% anotación de de las mismas. Entre las limitaciones identificadas están:\newline 
\subsection{Soporte para la clase java.lang.Override}
\label{sub:override}
% - \textit{Soporte para sobreescritura de métodos}\newline 
En la construcción de aplicaciones Android, según el componente que se esté
implementando(activities, content providers, receivers, services), se requiere
sobreescribir métodos de la clase que extienda el componente. Así, cuando se
define un componente tipo Activity, que debe extender de la clase Activiy.java, 
se sobreescriben métodos como Oncreate. Cada que se sobreescribe
un método se utiliza el statement @Override, con el cual se informa al
compilador de Java que el método es sobreescrito. No obstante, al implementar la
versión Jif de aplicaciones Android con dicho statement, el compilador de Jif
no lo reconoce. La dificultad que se presenta está en el reconocimiento del
statement(carácter @ y clase Override), y no en la sobreescritura de métodos,
puesto que Jif soporta tal característica.\newline 
El soporte para la sobreescritura de
métodos es confirmado con una sencilla prueba, anotando la clase Activity.java
del framework Android (con un único método, el método Ocreate), e implementando
la versión Jif de una aplicación Android que extiende de tal clase, en la cual
se define una actividad y sobreescribe el método Oncreate.
Cuando se comenta la sentencia @Override, el compilador de Jif identifica la
sobreescritura del método y reporta comentarios para el flujo de información.

Al investigar el por qué Jif no reconoce tal sentencia, se encuentra que dentro
de las clases Java estándar soportadas por el compilador de Jif, no se incluye
la clase java.lang.Override.\newline 
Las clases Java estándar pertenecientes a los paquetes io, lang, math, net y
sql; soportadas po el compilador Jif, vienen implementadas con anotaciones en el
directorio sig-src, directorio que forma parte de la distribución del compilador
de Jif con que se esté trabajando.

Un mecanismo para permitir el análisis de flujo de información entre métodos
que se sobreescriben, es comentar las líneas del programa que contengan la
sentencia @Override, puesto que, al no ser reconocida por el compilador de Jif,
es la generadora de errores de compilación.

\subsection{Casting entre tipos EditText y View}
\label{sec:casting}
El framework de Android cuenta con diferentes clases para manejar las interfaces
gráficas que presenta al usuario, entre las cuales se encuentran EditText y
View.\newline
View es la clase principal para la creación de widgets, necesarios para la
implementación de componentes interactivos en las interfaces de
usuario(UI).\newline 
EditText permite adicionar campos de texto editables en UI.\newline
El casting entre los tipos de datos que representan ambas clases, se hace cuando la aplicación debe
procesar datos provenientes de campos en las interfaces del usuario, por ejemplo
como se observa a continuación:
\begin{lstlisting}
EditText editPassword = (EditText)findViewById(R.id.password);
String password = editPassword.getText().toString();
\end{lstlisting}
la interfaz de usuario(que es de tipo View) contiene un campo R.id.password, y
para manipular la información que almacena, debe ser de tipo EditText, siendo
necesario un casting de tipo View a tipo EditText. La dificultad que se presenta
con este tipo de casting es que para el sistema de anotaciones de jif no es
válido. Luego de probar con la anotación manual de ambas clases, tratando de
dar soporte a este tipo de casting, sin obtener resultados satisfactorios, se
opta por ``simular'' estos casos, es decir, si el tipo de dato de una variable
es de tipo EditText, se crea una variable tipo String con un valor determinado,
así se omite el casting y se puede analizar el flujo de información.

\subsection{Clase nested R}
\label{sec:nested}
El framework de Android utiliza identificadores para hacer referencia a recursos
utilizados por la aplicación, recursos como strings, stilos, widgets, layouts, e
interfaces xml, tales identificadores son autogenerados en la clase R.java, allí
cada recurso es descrito como una clase individual. Al tratarse de una clase
nested, la clase R no puede ser anotada con jif. Esto puede solucionarse
implementando una versión Jif generalizada de la clase R, que contenga los
recursos utilizados en una aplicación, definidos como variables y no como clases.

% - \textit{Sources y Sinks}\newline
% En los preliminares para el diseño de la solución se propone utilizar SuSi
% para clasificar los sources y sinks en las aplicaciones a analizar, sin embargo, partir del
% extenso conjunto de sources y sinks, clasificados por SuSi para la API de
% Android, implica una mayor complejidad en el análisis, puesto que, en un
% aplicativo todo el código que le conforma puede hacer parte de sources o de
% sinks. Adicional a lo complejo que se puede tornar el análisis, los sources y
% sinks a considerar deben depender de la política de seguridad a evaluar. En ese
% orden de ideas, resulta más viable tomar un subconjunto del listado proveído por
% SuSi, partiendo de los sources y sinks que evalué la política de seguridad que
% se defina.

\subsection{Enhanced for loop}
\label{seubsec:enh}
Además de soportar la estructura de control for, el lenguaje Java permite el uso
de enhanced for, que es utilizado para simplificar la iteración en arrays y
colecciones, por ejemplo:
% \begin{lstlisting}
% char c[] = imei.toCharArray();
% for (int i = 0; i < c.length; i++) {
% 	obfuscated += c[i] + "_";
% }          
% \end{lstlisting}  
\begin{lstlisting}
for(char c : imei.toCharArray())
obfuscated += c + "_";
\end{lstlisting}
A diferencia de Java, Jif no soporta el enhanced for.\newline
Debido a que ambas sentencias de control son equivalentes, la solución que se
propone para poder analizar flujo de información en los aplicativos Android que
contengan dicha estructura de control, es generar el equivalente del programa
haciendo uso del for, de este modo se cambia la sintaxis sin afectar la lógica
del aplicativo a analizar.

\subsection{Otras limitaciones}
Adicional a las limitaciones descritas anteriormente, para las cuales se propone
una solución, se identifica que Jif no soporta la sintaxis utilizada para
definir estructuras de datos HashMaps, LinkedList y Sets, que en Java se definen
de la siguiente manera:
\begin{lstlisting}
Map<String,String> hashMap = new HashMap<String, String>();
List<String> listData;
Set<String> phoneNumbers = new HashSet<String>();
\end{lstlisting}
Jif tampoco permite la definición de interfaces como argumentos de un método. El
siguiente fragmento de código  en una aplicación Android, muestra la definición
de una interfaz pasada como parámetro al método setOnClickListener.
\begin{lstlisting}
   Button button1= (Button) findViewById(R.id.button1);
   button1.setOnClickListener(new View.OnClickListener() {
   ...
   .....}  });
\end{lstlisting}
En estos casos, la dificultad está en encontrar una sintaxis que permita obtener
la versión equivalente del programa que las contenga. A lo que se suma, la falta
de documentación disponible para solventar los mismos. En consecuencia, se omite
del conjunto de aplicaciones evaluables, aplicaciones Android que requieran en
su implementación las estructuras de datos descritas.\newline

%- \textit{Paso de statements dentro de los argumentos de un método
%(\{\}):}\newline
\section{Detalles de implementación}
Como se describió anteriormente, la implementación del diseño requiere
anotaciones a la API de Android y anotaciones en los aplicativos a
analizar.\newline 
En el caso de las anotaciones requeridas para la API \ref{subsec:api}, su
implementación se hace manualmente.\newline
En el caso de las anotaciones en el aplicativo a analizar \ref{subsec:apps},
para las que se propuso un esquema de anotación automatizable, se implementa un
prototipo de anotación en lenguaje Java. En la sección \ref{sec:diagramaClass}
de los anexos, se incluye su respectivo diagrama de clases.\newline 
El anotador está implementado en lenguaje Java y consta de cinco clases:
\emph{Main}, \emph{Source}, \emph{XmlExtract}, \emph{Annotation} y
\emph{BufferWriter}.\newline 
Todos los pasos de anotación descritos en
\ref{subsec:pasosSol}, son coordinados desde la clase \emph{Main}.\newline
La clase \emph{Source} identifica qué variables source contiene la aplicación que se
está analizando, esta clase se apoya en la clase \emph{XmlExtract} para identificar
sources definidos en interfaces UI. En este caso particular, para la
verificación de la política establecida, la clase \emph{XmlExtract} permite identificar
el source EditText, cuando se trata de un campo UI que almacena
contraseñas.\newline 
Como las interfaces UI se describen mediante XML, es
necesario extraer los elementos de la interfaz utilizando las librerías javax.xml.parsers y
org.w3c.dom. Con javax.xml.parsers se obtiene el respectivo DOM del XML a
analizar. Con org.w3c.dom se recorren los elementos del DOM, elementos como los
campos EditText.\newline 
Con la clase \emph{Annotation}  se identifican y anotan los diferentes tipos de
métodos(métodos source, no source y métodos a sobreescribir).\\
En la clase \emph{BufferWriter} se anotan las variables sources, y finalmente
se retorna la versión .jif del aplicativo que se está analizando.\newline 
Para la anotación del código se utiliza la librería Java Parser 1.8  que permite
generar y visitar el árbol de sintaxis para la estructura del código de cada
programa, haciendo posible la adición de los labels de seguridad respectivos.\\
% \textcolor{blue}{(Incluir Figura)}\newline 
\label{subsec:anotador}
\begin{figure}[t!]
	\begin{center}
	\includegraphics[width=12cm]{JifFilesystem.jpg}
	\end{center}
	\caption{Estructura de directorios en Jif.
	El código con anotaciones que se requiere para la evaluación de flujo de
	información en Jif, es alojado en los directorios sig-src y jif-src.}
	\label{fig:jifFilesystem} 
\end{figure}

\section{Ambiente y herramientas}
\textit{Versión de la API de Android}\newline
Las anotaciones a las clases de la API Android y las aplicaciones a anotar
mediante el prototipo, corresponden a la versión Android 4.2.2(API Level 17).

\textit{Versión del compilador de Jif}\newline
El componente que verifica el cumplimiento de las políticas de seguridad,
corresponde a la versión 3.4.2 del compilador de Jif.

\textit{Estructura de trabajo en JIF}\newline
El compilador de Jif maneja una estructura de directorios en la que se incluye
el código fuente .jif de las aplicaciones a analizar. Como ilustra la figura
\ref{fig:jifFilesystem} se tienen los directorios principales sig-src y jif-src.
El directorio sig-src está destinado para incluir librerías adicionales.\newline 
El directorio jif-src contiene todo el código jif correspondiente al programa
como tal que se va evaluar mediante Jif. Para los propósitos del presente
trabajo, se tienen los subdirectorios principals y project. En el subdirectorio
principals se incluyen las autoridades requeridas, y en el subdirectorio
project se incluyen, tanto las clases específicas de los aplicativos Android a
analizar, como las clases de que estos heredan como: Activity.jif, Service.jif,
etc.

\section{Compilación, integración y finalmente ejecución }
En la sección \ref{sec:ejecutarPrototipo} de los anexos, se describen las
instrucciones paso a paso de cómo ejecutar el anotador, y cómo compilar los
aplicativos a analizar mediante Jif.
    
\label{ch:evaluacion}
\chapter{Evaluación}
% Ventajas y limitaciones de la solución.\newline 
% Si aplica, evaluación de desempeño.  \newline 
% Si aplica, evaluación de usabilidad.  
% Hay otras soluciones similares? \newline 
% Cuáles son las diferencias y las ventajas y desventajas con respecto a esas soluciones.

\section{Consideraciones de evaluación}
El flujo de información es analizado al interior de una sóla aplicación, no se
consideran flujos de información vía interApp, es decir, varias aplicaciones que
se comunican entre sí.

\section{Conjunto de evaluación}
\label{sec:evalSet}
Para la evaluación se parte de DroidBech versión 1.0\cite{DroidBenchBenchmarks},
benchmark integrado por 39 casos de prueba para aplicaciones Android, cuyos
autores son los mismos creadores de FlowDroid. Se opta por utilizar DroidBench
puesto que, en la literatura científica consultada al respecto, no se encuentran
otros benchmarks diseñados específicamente para evaluar aplicaciones Android.\newline 
De DroidBech se toma un grupo de testcases evaluables frente a la política de
seguridad establecida\ref{subsection:politica}, este grupo está integrado por 20
testcases. De los cuales, 14 presentan fugas de información.

La tabla \ref{tab:descripApps0} describe parte del grupo de testcases a
evaluar. En los casos en que se requiere, se precisan observaciones entre los
resultados de evaluación esperados para la técnica de análisis utilizada por
FlowDroid(análisis de flujo de datos) y la técnica de análisis propuesta en el
presente trabajo(análisis de flujo de información).
En la sección\ref{sec:testcases} de los anexos, se encuentra la
descripción del grupo de prueba completo.

El conjunto de prueba es analizado con FlowDroid, JoDroid y con el Prototipo. Los
resultados del análisis que devuelve cada herramienta son calificados como:
Falso Positivo(FP) cuando se detecta un leak que no existe; Falso Negativo(FN)
cuando no se detecta un leak existente; Verdadero Positivo(TP) cuando se detecta
un leak existente; Verdadero Negativo(TN) cuando no existe leak que detectar.

En base a estos resultados se calcula la Precisión y el Recall, para cada una de
las herramientas evaluadas. La Precisión mide la cantidad de verdaderos
Positivos(TP) frente a la cantidad de Falsos positivos(FP), reportados por la
herramienta. A mayor Precisión, la herramienta detecta menos falsos
positivos(FP).\newline 
El Recall(\textit{r}) mide la cantidad de verdaderos Positivos(TP) frente a la
cantidad de Falsos negativos(FN), reportados por la herramienta. A mayor Recall,
menos leaks deja pasar la herramienta, es decir menor reporte de falsos
negativos. Las formulas \ref{p} y \ref{r}, permiten el calculo de ambas métricas
de seguridad.\newline
Adicionalmente, para medir el desempeño de cada herramienta se utiliza el
comando \textit{time}\cite{time-man} de unix. 

\begin{table}[H]
\small\addtolength{\tabcolsep}{-3pt}
\begin{tabular}{|p{13cm}|p{1cm}|}
	\hline
	\multicolumn{2}{|>{\columncolor[gray]{0.8}}c|}{\textbf{AndroidSpecific\_DirectLeak1}}\\
	\hline
	\textbf{Descripción} & \textbf{Leaks}\\
	\hline
	La variable \textit{mrg} tiene un nivel de seguridad alto,
	almacena información retornada por el método source \textit{getDeviceId}. Se
	genera flujo de información directo entre información con nivel de seguridad alto e
	información con nivel de seguridad bajo, al enviar como parámetro del método
	\textit{sendTextMessage}, información de la variable \textit{mrg}. & 1 \\
	\hline
	\multicolumn{2}{|>{\columncolor[gray]{0.8}}c|}{\textbf{AndroidSpecific\_LogNoLeak}}\\
	\hline
	\textbf{Descripción} & \textbf{Leaks}\\
	\hline
	El caso de prueba no presenta información con niveles de seguridad alto. Se
	presentan flujos de información entre información con el mismo nivel de
	seguridad, en este caso bajo, lo cual es permitido. & 0 \\
	\hline
	\multicolumn{2}{|>{\columncolor[gray]{0.8}}c|}{\textbf{ArraysAndLists\_ArrayAccess1}}\\
	\hline
	\textbf{Descripción} & \textbf{Leaks}\\
	\hline
	Se tiene un array en que se almacena información tanto proveniente como no
	proveniente de sources, parte de la información que almacena es enviada como
	parámetro del método \textit{sendTextMessage}. \textit{Observación:}
	Para la técnica de análisis de FlowDroid(taint analysis), se marca únicamente el
	índice del array donde se almacena el dato considerado como source, así,
	cuando se envía como parámetro del método \textit{sendTextMessage},
	el dato de un índice no marcado, no se genera leak. Para nuestra técnica
	de análisis(flujo de información mediante JIF), para que un array almacene
	información con nivel de seguridad alto, primero debe ser catalogo(anotado)
	con nivel de seguridad alto, lo que implica que el array podrá almacenar
	información tanto de nivel de seguridad alto como bajo, pero toda la
	información quedará con nivel de seguridad alto. En consecuencia, al enviar
	cualquier índice del array como parámetro del método 
	\textit{sendTextMessage} se presenta un flujo de información no
	permitido. & 0
	\\
	\hline
	\multicolumn{2}{|>{\columncolor[gray]{0.8}}c|}{\textbf{GeneralJava\_Exceptions2}}\\
	\hline
	\textbf{Descripción} & \textbf{Leaks}\\
	\hline
	La variable \textit{imei} es de nivel de seguridad alto, almacena información
	devuelta por el método \textit{getDeviceId}. El control de flujo del
	programa conduce de manera implícita a la captura de una excepción tipo
	RuntimeException, desde allí se utiliza información proveída por la variable
	\textit{imei}, como parámetro para invocar el método \textit{sendTextMessage}.
	Generando un flujo de información indebido. & 1
	\\
	\hline
	\multicolumn{2}{|>{\columncolor[gray]{0.8}}c|}{\textbf{ImplicitFlows\_ImplicitFlow2}}\\
	\hline
	\textbf{Descripción} & \textbf{Leaks}\\
	\hline
	 La variable \textit{userInputPassword} con nivel de seguridad alto, almacena
	 información de un campo EditText tipo textPassword(password suministrado por
	 el usuario). Se generan flujos de información indebidos: al tratar de asignar
	 información a la variable passwordCorrect con nivel de seguridad bajo, a
	 partir de la comparación de información con nivel de seguridad alto(variable
	 textPassword), después, al tratar de mostrar en el \textit{log} información
	 que depende de tal comparación. & 1\\
	\hline
\end{tabular}
\caption{Descripción aplicaciones de prueba.\newline
Se considera con nivel de seguridad alto, variables y métodos que almacenan y
modifican(respectivamente), información catalogada como privada(Sources).\newline 
Se considera con nivel de seguridad bajo, canales para envío de mensajes,
muestra de logs y canales creados durante el control de flujo del programa.\newline }
\label{tab:descripApps0}
\end{table}

\subsection{Evaluación FlowDroid y Prototipo } 
\label{subsec:fvsp}
Del total de testcases(20), 14 presentan fugas de información mientras que 6 de
ellos no. Los resultados de evaluación para FlowDroid y el Prototipo, son
presentados en la tabla \ref{tb:resultados}. En esta, por cada caso de prueba
se indica la cantidad de leaks que presenta, el resultado y el tiempo que tarda el
análisis.

\begin{table}[H]
\begin{center}
\small\addtolength{\tabcolsep}{-3pt}
\begin{tabular}{|p{6cm}|p{1cm}|p{1cm}|p{1cm}|p{1cm}|p{1cm}|}
	\hline
	\textbf{Testcase} & \textbf{Leaks} & \textbf{F} &
	\textbf{P} & \textbf{ tF} & 
	\textbf{tP}\\
	\hline
	AndroidSpecific\_DirectLeak1 & 1 & TP & TP &5.371s &2.063s\\
	\hline
	AndroidSpecific\_InactiveActivity & 0 & TN & FP  &3.255s &2.469s\\
	\hline
	AndroidSpecific\_LogNoLeak & 0 & TN & TN &5.505s &2.946s\\
	\hline
	AndroidSpecific\_Obfuscation1 & 1 & TP & TP &6.734s &2.706s\\
	\hline
	 AndroidSpecific\_PrivateDataLeak2 & 1 & TP & TP & 6.144s &2.644s\\
	\hline
	 ArraysAndLists\_ArrayAccess1 & 0 & FP & FP & 4.708s & 1.278s\\
	\hline
	 ArraysAndLists\_ArrayAccess2 & 0 & FP & FP & 4.4s &1.361s\\
	 \hline
	 GeneralJava\_Exceptions1 & 1 & TP & TP &6.397s &2.755s\\
	\hline
	 GeneralJava\_Exceptions2 & 1 & TP & TP &5.887s &1.980s\\
	\hline
	GeneralJava\_Exceptions3 & 0 & FP & FP &6.008s &2.032s\\
	\hline
	GeneralJava\_Exceptions4 & 1 & TP & TP &5.731s &2.313s\\
	\hline
	GeneralJava\_Loop1 & 1 & TP & TP &5.605s &2.800s\\
	\hline
	GeneralJava\_Loop2 & 1 & TP & TP &4.719s &1.361s\\
	\hline
	GeneralJava\_UnreachableCode & 0 & TN & FP &3.792s &1.197s\\
	\hline
	ImplicitFlows\_ImplicitFlow1 & 1 & FN & TP &4.853s &1.331s\\
	\hline
	ImplicitFlows\_ImplicitFlow2 & 1 & FN & TP &4.496s &1.212s\\
	\hline
	ImplicitFlows\_ImplicitFlow4 & 1 & FN & TP &4.375s &1.224s\\
	\hline
	Lifecycle\_ActivityLifecycle3 & 1 & TP & TP &4.792s &1.222s\\
	\hline
	Lifecycle\_BroadcastReceiverLifecycle1 & 1 & TP & TP &4.456s &1.061s\\
	\hline
	Lifecycle\_ServiceLifecycle1 & 1 & TP & TP &5.225s &1.180s\\
	\hline
\end{tabular}
\end{center}
\caption{Resultados de evaluación para FlowDroid y Prototipo. Donde
\textit{Testcase} especifica el nombre de la aplicación que se está evaluando;
\textit{Leaks} indica si el testcase presenta fugas de información;
\textit{F} y  \textit{P} muestran los resultados devueltos por FlowDroid y por
el Prototipo; \textit{tF} y \textit{tP}, señalan el tiempo(en segundos) que toma
el análisis para Flowdroid y para el Prototipo, respectivamente.}
\label{tb:resultados}
\end{table}
\subsubsection{Análisis de evaluación entre FlowDroid y Prototipo}
-\textit{Resultados de desempeño}\newline
Acorde a los datos señalados en los campos tF y tP de la tabla
\ref{tb:resultados}, en promedio, FlowDroid tarda 3,3 segundos más que el
Prototipo para ejecutar el análisis.

% PULIR IDEAAAAA***\newline
% Tal diferencia de tiempo va en concordancia con los costos de ejecución que
% implica una u otra técnica de análisis, mientras el análisis de flujo de
% información basado en lenguajes tipados de seguridad(que es lo que se evalua
% mediante el prototipo) sólo requiere llegar hasta el chequeo de tipos, la
% técnica de analisis de flujo de datos en que se basa FlowDroid utiliza
% algoritmos de orden O(ED)3 , cuya complejidad es de orden polinomial,
% O(ED)3\cite[page 3]{FCO-PDG}.\newline
% Se podría destacar como positivo que el análisis de flujo de información
% mediante técnicas de tipado de seguridad, requiere menos tiempo que la técnica
% de marcado de datos utilizada por FlowDroid.
-\textit{Resultados de precisión}\newline
En lo que respecta a los resultados del Prototipo, los FP correspondientes a
AndroidSpecific\_InactiveActivity y GeneralJava\_UnreachableCode, surgen como
consecuencia de un análisis pesimista, donde se asume que el desarrollador
utiliza lo que implementa, es decir, que los métodos serán invocados.\newline 
Por otro lado, en el caso de ArraysAndLists: ArrayAccess1 y ArrayAccess2, no es
sencillo calificar los resultados como FP, puesto que, para lo que está
analizando FlowDroid(verificar que su técnica de análisis diferencie entre los
elementos marcados y no marcados de un array), efectivamente se presentan FP,
sin embargo, para la forma en que se deben implementar los programas en Jif,
donde se suele definir un nivel de seguridad para todo el array antes de
almacenar los elementos en el mismo, podría decirse que no se trata de un FP,
porque se revelo información que había sido catalogada con nivel de seguridad alto.

A diferencia de FlowDroid, el Prototipo detecta fugas de información través de
flujos implícitos. La no detección de Flujos implícitos por parte de FlowDroid,
responde al tipo de análisis y las técnicas en que se fundamenta la herramienta:
análisis de flujo de datos mediante técnicas tainting. Basada en 
%Puesto que, 
% 
% \newline
% --el Prototipo, la anotación generada por el prototipo, o la técnica de análisis
% en que se basa el diseño de la solución: lenguajes tipados de seguridad???
% 
% -\textit{Acerca de por qué FlowDroid no detecta flujos implícitos}\newline
%el análisis de FlowDroid utiliza  técnicas DataFlow, específicamente, utiliza
% tainting análisis.
% Para hacer seguimiento al flujo de información de un
% programa, la técnica de análisis tainting se basa en: 
asociar una o más marcas
con el valor de los datos en el programa, y en propagarlas. Dependiendo de los
criterios definidos para el análisis, la marca puede ser propagada a causa de
flujos explícitos o de flujos implícitos, o a causa de ambos. En flujos
explícitos la propagación ocurre cuando el valor de una variable marcada está
implicada en el calculo de otra variable. En flujos implícitos la propagación
tiene lugar a través de dependencias en el control de flujo del programa, por
ejemplo, cuando el valor de un dato marcado afecta indirectamente otra variable.\newline 
En el caso de FlowDroid, los criterios que fundamentan el análisis de la
herramienta, hacen que el marcado de datos se propague para flujos explícitos y
y no para flujos implícitos. Por consiguiente, FlowDroid no detecta flujos
implícitos.\newline
Con esto presente, se definen dos escenarios de análisis: Precisión y Recall,
incluyendo flujos implícitos; y  Precisión y Recall, exlcuyendo flujos
implícitos, donde se incluyen u omiten los testcases para flujos
implícitos(ImplicitFlows\_ImplicitFlow 1, 2 y 4).
La tabla \ref{tb:FlowDroidPrototipoFI} muestra los resultados de evaluación para
ambos casos.\newline
%En el caso de TaintDroid, tampoco detecta flujos implícitos, porque entre las
% desiciones de diseño de la herramienta está enfocarse en el seguimeinto al
% flujo de datos y no al flujo de control, puesto que si incluyen seguimiento al
% flujo de control, se adiciona sobrecarga a la herramienta, la cual es de tipo
% dinámico.

\begin{table}[t!]
\begin{center}
\begin{tabular}{c|c|c|c|c|}
\cline{2-5}
& \multicolumn{2}{>{\columncolor{gray!30}} c  }{\multirow{1}{*}{Con FI}}
& \multicolumn{2}{|c|}{\multirow{1}{*}{\cellcolor{gray!50} {Sin FI}}}\\
\cline{2-5}
& \cellcolor{gray!30}FlowDroid & \cellcolor{gray!30}Prototipo &
\cellcolor{gray!50}FlowDroid & \cellcolor{gray!50}Prototipo \\
\cline{1-5}
\multicolumn{0}{ |c|  }{\multirow{0}{*}{FP} }  & 3 & 5 & 3 & 5\\ \cline{0-4}
\multicolumn{0}{ |c|  }{\multirow{0}{*}{FN} }  & 3 & 0 & 0 & 0\\ \cline{0-4}
\multicolumn{0}{ |c|  }{\multirow{0}{*}{TP} }  & 11 & 14 & 11 & 11 \\
\cline{0-4}
\multicolumn{0}{ |c|  }{\multirow{0}{*}{TN} }  & 3 & 1 &  3 & 1\\ \cline{0-4}
\end{tabular}
\end{center}
\caption{Resultados de precisión para FlowDroid y Prototipo, de acuerdo al
escenario, incluyendo o excluyendo flujos implicitos(FI). Resume el total de
falsos positivos(FP), verdaderos positivos(TP), verdaderos negativos(TN) y falsos
negativos(FN); obtenidos tanto con FlowDroid como con el Prototipo.}
\label{tb:FlowDroidPrototipoFI}
\end{table}
\textit{Precisión y Recall, incluyendo flujos implícitos}\newline
Los resultados obtenidos en la tabla \ref{tb:FlowDroidPrototipoFI} señalan que
de las 14 fugas existentes, el Prototipo las detecta todas, presenta 14
TP(verdaderos positivos); mientras que, FlowDroid deja pasar 3.\newline
Por otro lado, el Prototipo presenta más falsos positivos que FlowDroid, de los
6 testcases que no presentan leaks, el prototipo reporta 5 como si fuesen
fugas, mientras que FlowDroid reporta 3.\newline
Así, en lo que respecta a Precisión, FlowDroid presenta un porcentaje del 78\%,
siendo más preciso frente al Prototipo, que presenta un porcentaje del
73\%.\newline 
Por otro lado, el Prototipo presenta un porcentaje en Recall del 100\%,
mientras que FlowDroid presenta un porcentaje del 78\%.\newline

\textit{Precisión y Recall, exlcuyendo flujos implícitos}\newline
Excluyendo los testcases para flujos implícitos, el conjunto de prueba se reduce
a 17 casos. De los cuales, 11 presentan leaks.\newline 
En este contexto, el porcentaje de Precisión para FlowDroid es de 78,57\%,
mientras que para el Prototipo es del 68,75\%. El porcentaje de Recall es igual
para FlowDroid y el Prototipo: 100\%.


\subsection{Evaluación JoDroid y Prototipo}
% Del conjunto de casos de prueba JoDroid ignora las excepciones, ya que su actual
% implementación no soporta análisis del flujo de información a través de
% tales sentencias[pag 93 \cite{JoDroid-Thesis}]. Por tanto, los testcases
% GeneralJava\_Exceptions1 a GeneralJava\_Exceptions4, no son evaluados. 
% No obstante, para calcular la precisión y el recall se consideran dos
% escenarios, uno en que se incluyen y otra en que no.
A continuación se ilustran los resultados de evalución para JoDroid y el
Prototipo, en base a los cuales, se calculan las métricas de precisión y recall. 
\label{subsec:jvsp}
\begin{table}[H]
\begin{center}
\small\addtolength{\tabcolsep}{-3pt}
\begin{tabular}{|p{6cm}|p{1cm}|p{1cm}|p{1cm}|p{1,8cm}|p{1cm}|}
	\hline
	\textbf{Testcase} & \textbf{Leaks} & \textbf{J} &
	\textbf{P} & \textbf{ tJ} & 
	\textbf{tP}\\
	\hline
	AndroidSpecific\_DirectLeak1 & 1 & TP & TP & 22m11.991s &2.063s\\
	\hline
	AndroidSpecific\_InactiveActivity & 0 & FP & FP  & 22m25.617s &2.469s\\
	\hline
	AndroidSpecific\_LogNoLeak & 0 & TN & TN & 21m6.548s &2.946s\\
	\hline
	AndroidSpecific\_Obfuscation1 & 1 & TP & TP &22m46.541s&2.706s\\
	\hline
	 AndroidSpecific\_PrivateDataLeak2 & 1 & TP & TP &21m32.447s&2.644s\\
	\hline
	 ArraysAndLists\_ArrayAccess1 & 0 & FP & FP &22m01.926s& 1.278s\\
	\hline
	 ArraysAndLists\_ArrayAccess2 & 0 & FP & FP &22m11.023s&1.361s\\
	 \hline
	 GeneralJava\_Exceptions1 & 1 & FN & TP & 22m52.134s &2.755s\\
	\hline
	 GeneralJava\_Exceptions2 & 1 & FN & TP & 21m4.434s&1.980s\\
	\hline
	GeneralJava\_Exceptions3 & 0 & TN\tablefootnote{Al igual que en
	los testcases GeneralJava\_Exceptions(1, 2 y 4), la herramienta no detecta
	leaks, la diferencia para el presente caso, es que efectivamente no existe
	leak. Por tanto se califica como TN.} & FP & 21m37.040s &2.032s\\
	\hline
% 	\hline
% 	GeneralJava\_Exceptions3 & 0 & TN & FP & 21m37.040s &2.032s\\
% 	\hline
	GeneralJava\_Exceptions4 & 1 & FN  & TP & 21m10.240s &2.313s\\
	\hline
	GeneralJava\_Loop1 & 1 & TP & TP &21m15.30s&2.800s\\
	\hline
	GeneralJava\_Loop2 & 1 & TP & TP &21m41.224s&1.361s\\
	\hline
	GeneralJava\_UnreachableCode & 0 & TN & FP &22m84.138s&1.197s\\
	\hline
	ImplicitFlows\_ImplicitFlow1 & 1 & TP & TP &22m55.645s&1.331s\\
	\hline
	ImplicitFlows\_ImplicitFlow2 & 1 & TP & TP &22m32.231s&1.212s\\
	\hline
	ImplicitFlows\_ImplicitFlow4 & 1 & TP & TP &22m43.110s&1.224s\\
	\hline
	Lifecycle\_ActivityLifecycle3 & 1 & TP & TP &22m54.651s&1.222s\\
	\hline
	Lifecycle\_BroadcastReceiverLifecycle1 & 1 & TP & TP &22m42.347s&1.061s\\
	\hline
	Lifecycle\_ServiceLifecycle1 & 1 & TP & TP &22m92.722s&1.180s\\
	\hline
\end{tabular}
\end{center}
\caption{Resultados de evaluación para JoDroid y Prototipo. Donde
\textit{Testcase} especifica el nombre de la aplicación que se está evaluando;
\textit{Leaks} indica si el testcase presenta fugas de información; \textit{J} y
\textit{P} muestran los resultados devueltos por JoDroid y por el Prototipo;
\textit{tJ} y \textit{tP}, señalan el tiempo que toma el análisis para JoDroid
y para el Prototipo, respectivamente.}
\label{tab:JoDroid-Prototipo}
\end{table}

\subsubsection{Analisis de evaluación entre JoDroid y Prototipo}
-\textit{Resultados de desempeño}\newline
Para el analisis mediante JoDroid se deben seguir una seríe de pasos, tal como
se describen en el manual de referencia\cite{joDroidManual}, de estos,
únicamente se contabiliza el tiempo correspondiente al paso para generar el
grafo de dependencia del programa(PDG), del cual parte el análisis. En general,
el tiempo que tarda la generación del PDG para cada aplicación analizada, oscila
entre 21 y 22 minutos. Cabe anotar que estos valores podrían cambiar con otras
características de hardware, sin embargo, asignando 1048 MB de Ram a la máquina
virtual de Java, para la generación del PDG, es ese el rango de tiempo obtenido.
En consecuencia, para los valores de tiempo que señala la presente evaluación,
es posible decir que la herramienta es costosa en desempeño.

-\textit{Resultados de Precisión}\newline
La tabla \ref{tab:JoDroid-Prototipo} muestra que al igual que el Prototipo,
JoDroid detecta fugas de información presentes en flujos implícitos.\newline
Por otro lado, es importante resaltar que del conjunto de casos de prueba,
JoDroid ignora el control de flujo de información para
excepciones(GeneralJava\_Exceptions1 a GeneralJava\_Exceptions4), puesto que, su
actual implementación no soporta análisis del flujo de información a través de
tales sentencias[pag 93 \cite{JoDroid-Thesis}]. En la
tabla\ref{tb:jodroidP2exce} se muestran dos excenarios para los resultados de
evaluación: Con  excepciones y Sin excepciones. Con base en tales resultados se
calcula la Precisión y Recall para cada uno de los escenarios.

\begin{table}[t!]
\begin{center}
\begin{tabular}{c|c|c|c|c|}
\cline{2-5}
& \multicolumn{2}{>{\columncolor{gray!30}} c  }{\multirow{1}{*}{Con
excepciones}} & \multicolumn{2}{|c|}{\multirow{1}{*}{\cellcolor{gray!50} {Sin
excepciones}}}\\
\cline{2-5}
& \cellcolor{gray!30}JoDroid & \cellcolor{gray!30}Prototipo &
\cellcolor{gray!50}JoDroid & \cellcolor{gray!50}Prototipo \\
\cline{1-5}
\multicolumn{0}{ |c|  }{\multirow{0}{*}{FP} }  & 3 & 5 & 3 & 4\\ \cline{0-4}
\multicolumn{0}{ |c|  }{\multirow{0}{*}{FN} }  & 3 & 0 & 0 & 0\\ \cline{0-4}
\multicolumn{0}{ |c|  }{\multirow{0}{*}{TP} }  & 11 & 14 & 11 & 11 \\
\cline{0-4}
\multicolumn{0}{ |c|  }{\multirow{0}{*}{TN} }  & 3 & 1 &  2 & 1\\ \cline{0-4}
\end{tabular}
\end{center}
\caption{Resultados de precisión para JoDroid y Prototipo. Muestra los
escenarios en que mide. Resume el total de falsos positivos(FP), verdaderos
positivos(TP), verdaderos negativos(TN) y falsos negativos(FN); obtenidos tanto
con JoDroid como con el Prototipo.}
\label{tb:jodroidP2exce}
\end{table}

\textit{Precisión y Recall incluyendo excepciones}\newline
Del total de testcases(20), 14 presentan fugas de información. 
De los casos con fuga de información, 3 corresponden a las excepciones
incluidas(GeneralJava\_Exceptions), y se califican como falsos negativos(FN)
puesto que JoDroid no los detecta. La tabla\ref{tb:jodroidP2exce} ilustra los
resultados de evaluación.\newline
En cuanto a la Precisión(p), JoDroid presenta un porcentaje del 78,57\%,
mientras que el Prototipo presenta un porcentaje del  73,68\%.\newline
En cuanto a Recall(r), el Prototipo presenta un porcentaje del 100\%, frente a
un porcentaje del 78,57\% presentado por JoDroid.


\textit{Precisión y Recall excluyendo exceptions}\newline
Omitiendo los casos de prueba de GeneralJava\_Exceptions1 a
GeneralJava\_Exceptions4, el total de testcases(20) queda reducido a 16. De
estos, 11 presentan fugas de información.\newline
En lo que respecta a la métrica de Precisión, JoDroid
presenta un porcentaje del 78,57\%; frente al Prototipo que presenta un
porcentaje del 73,33\%.\newline 
Para la métrica de Recall, tanto JoiDroid como el Prototipo, presentan el mismo
porcentaje esto es 100\%.\newline



\subsection{Análisis de evaluación FlowDroid, JoDroid, Prototipo}
\label{subsec:fjp}
%las tablas confirman lo que ya se sabia segun la literatura existe.

En base a los resultados de evaluación para el conjunto de
evaluación(compuesto por 20 testcases, de los cuales 14 presentan leaks),
obtenidos en los anteriores apartados, 
% donde se presentó un análisis detallado para la evaluación entre FlowDroid vs
% Prototipo, y, JoDroid vs Prototipo; 
se comparan las tres herramientas(FlowDroid, JoDroid y Prototipo) frente a
Precisión, Recall, y la detección de fugas de información mediante flujos
implícitos. La tabla \ref{tb:porcentajes} ilustra todos los resultados y la
tabla \ref{tb:comparacion} ilustra los respectivos porcentajes.

\begin{table}[t!]
\begin{center}
\begin{tabular}{c|c|c|c|}
\cline{2-4}
& \cellcolor{gray!30}FlowDroid & \cellcolor{gray!30}JoDroid &
\cellcolor{gray!30}Prototipo \\
\cline{1-4}
\multicolumn{0}{ |c|  }{\multirow{0}{*}{FP} }  & 3 & 5 & 3\\ \cline{0-3}
\multicolumn{0}{ |c|  }{\multirow{0}{*}{FN} }  & 3 & 0 & 0\\ \cline{0-3}
\multicolumn{0}{ |c|  }{\multirow{0}{*}{TP} }  & 11 & 14 & 11\\\cline{0-3}
\multicolumn{0}{ |c|  }{\multirow{0}{*}{TN} }  & 3 & 1 &  3\\ \cline{0-3}
\end{tabular}
\end{center}
\caption{Resultados de precisión para FlowDroid y Prototipo. Resume el total de
falsos positivos(FP), verdaderos positivos(TP), verdaderos negativos(TN) y
falsos negativos(FN).}
\label{tb:porcentajes}
\end{table}

\begin{table}[t!]
\begin{center}
\begin{tabular}{cc|c|c|c}
\cline{2-4}
& \multicolumn{0}{ |c|  }{\multirow{1}{*}{\cellcolor{gray!30} FlowDroid} } &
\cellcolor{gray!30}JoDroid & \cellcolor{gray!30}Prototipo \\
\cline{1-4}
\multicolumn{0}{ |c|  }{\multirow{0}{*}{Precisión} }  & 78\% & 78,57\% & 73,68\%
\\
\cline{0-3}
\multicolumn{0}{ |c|  }{\multirow{0}{*}{Recall} }  & 78\% & 78,57\% &  100\%\\
\cline{0-3}
\multicolumn{0}{ |c|  }{\multirow{0}{*}{Detección Flujos Implícitos} }  & No &
Si & Si\\
\cline{0-3}
\end{tabular}
\end{center}
\caption{Comparación entre FlowDroid, JoDroid y Prototipo. Ilustra los
porcentajes para Precisión, Recall, y la detección de leaks mediante
flujos implícitos.\newline}
\label{tb:comparacion}
\end{table}

-\textit{Desempeño}\newline 
Como muestran las tablas \ref{tb:resultados} y \ref{tab:JoDroid-Prototipo}, el
Prototipo presenta mejor desempeño frente FlowDroid y JoDroid. En el caso de
FlowDroid, en promedio tarda 3,3 segundos más que el Prototipo para ejecutar el
análisis. En el caso de JoDroid, el tiempo de análisis es costoso en comparación
a las otras herramientas, puesto que su tiempo de ejecución oscila entre 21 y 22
minutos.\newline
% EXPLICAR POR QUE, DE ACUERDO A LA TECNICA DE ANALISIS?\newline
% Como muestran las tablas \ref{tb:resultados} y \ref{tab:JoDroid-Prototipo}, el
% análisis de flujo de información basado en lenguajes tipados de seguridad(en que
% se fundamenta la propuesta de análisis) presenta un mejor desempeño, frente a
% las técnicas en que se basan FlowDroid y JoDroid, análisis de flujo de datos
% mediante técnicas tainting y IFC(Information Flow Control) mediante PDG y
% slicing, respectivamente.

-\textit{Precisión y Recall}\newline
Tanto FlowDroid como JoDroid presentan mejor Precisión que el Prototipo, es
decir que el Prototipo presenta más falsos positivos(FP).\newline 
Por otro lado, el Prototipo presenta mayor Recall frente a FlowDroid y JoDroid,
por tanto, el Prototipo detecta mayor cantidad de fugas existentes (reporta
menos FN).
Para este caso particular, el Prototipo detecta todos los TP.\newline 
En consecuencia, es posible decir que aunque el Prototipo presenta mayor
cantidad de FP frente a FlowDroid y JoDroid, deja pasar menos fugas de
información.\newline
En lo que respecta a flujos implícitos, a diferencia de FlowDroid, tanto JoDroid
como el Prototipo detectan fugas de información a través de Flujos
implícitos.\newline

% EXPLICAR POR QUE DE ACUERDO A LAS TECNICAS DE ANALISIS?\newline
% El análisis de flujo de información mediante lenguajes tipados de
% seguridad(en que se basa el Prototipo), ofrece un mejor Recall que
% FlowDroid, sin embargo FlowDroid es más preciso. Esto se traduce en que la
% propuesta de análisis evaluada a través del Prototipo, presenta más falsos
% positivos que FlowDroid, pero no deja pasar fugas de información.\newline
% Por otro lado, el Prototipo detecta fugas de información presentes en Flujos
% implicitos, FlowDroid No.\newline
% El análisis de flujo de información mediante lenguajes tipados de seguridad,
% ofrece igual Recall que la técnica de PDG utilizada por JoDroid, sin
% embargo, JoDroid presenta mejor Precisión.\newline
Analizando los resultados para las métricas de desempeño, precisión y recall;
descritas anteriormente, acorde al tipo de análisis y técnicas en que se basa
cada herramienta, es posible anotar:\newline 
- Desempeño: la técnica de lenguajes tipados se seguridad en que se basa el
prototipo aprovecha las ventajas de optimización con que se construyen los
compiladores, permitiendo que el tiempo para ejecución del análisis sea
despreciable.\newline 
-Precisión, Recall y detección de Flujos implícitos: el análisis pesimista en
que se basa el Prototipo, donde se asume que todos los métodos implementados en
la aplicación seran invocados, hace que los resultados del análisis sean menos
precisos.
%(EL ANÁLISIS PESIMISTA O LA TECNICA DE LENGUAJES TIPADOS?)
Al basar su técnica de análisis en flujo de información, el Prototipo y JoDroid
ofrecen mayor Recall que FlowDroid, que se basa en flujo de datos.\newline
Una ventaja de las técnicas basadas en control de flujo de información es la
detección de fugas de información a través de flujos implícitos.

  

\subsection{Comparación técnicas de análisis evaluadas}
% En las subsecciones anteriores(4.2.1 a 4.2.3), se analizaron los resultados de
% evaluación con respecto a un conjunto de aplicaciones específico. En la presente
% sección, el análisis se basa en las herramientas previamente evaluadas, pero
% haciendo enfásis en las técnicas utilizadas por las mismas.
%tipo de analisis y técnica
FlowDroid se fundamenta en análisis de flujo de datos, mediante técnicas
tainting.\\
El código .dex a ser analizadado es transformado a una representación
intermedia(Jimple representation).\\
El análisis parte de la construcción de un super-grafo del programa que se
analiza, el super-grafo es una colección de grafos dirijidos, mediante los
cuales se representa el programa, donde los nodos asocian las sentencias del
programa y las aristas, la forma en que estas se conectan. Para recorrer el
super-grafo utiliza un algoritmo basado en el problema de
graph-reachability\cite{Graph-reachability}; cuyo costo computacional es de
orden polinomial O(ED3), donde E representa funciones de flujo de datos(dataflow
functions) y D conjunto de elementos para guiar el seguimiento de los
datos marcados(set of data flow facts).\newline
Para propagar la marca en los datos que análiza omite el control de flujo de
información, sólo se centra en el flujo de datos marcados como sources y
sinks.\newline
La herramienta recibe como entrada el apk del aplicativo, detecta
automáticamente los sources y sinks del programa mediante el uso de SuSi y
genera un reporte del análisis.

JoDroid se fundamenta en analísis de control de flujo de información, aplicando
técnicas de grafos de dependencia(PDG) y técnicas slicing.\newline 
El código .dex es transformado a código de representación intermedio(SSA-form).
Construye un grafo PDG, donde los nodos representan statements y expresiones, y
las aristas modelan las dependencias sintacticas entre los statements y
expresiones. Este PDG permite modelar flujos explícitos e implícitos.\newline
El costo computacional un análisis basado en PDG es de orden polinomial
O(N)3\cite[page 3]{FCO-PDG}.\newline 
Para hacer seguimiento al control de flujo de información, utiliza labels de
seguridad, estos califican con nivel de seguridad alto o bajo información de
variables y statements.\newline
Los procedimientos para usar la herramienta comprenden: generar el punto de
entrada del análisis, generar el PDG, ejecutar el respectivo análisis. Primero,
recibe como entrada el apk y manifest del aplicativo para generar un archivo con
el punto de entrada del análisis; luego, a partir del archivo devuelto
anteriormente genera el PDG, finalmente, recibe como entrada el PDG, lista los
statements y variables del aplicativo para que se indique manualmente los
sources y sinks, y genera el respectivo análisis.\newline

La propuesta está basada en análisis de flujo de información mediante lenguajes
tipados de seguridad, más específicamente mediante Jif.\newline
Para cada programa a analizar se debe implementar la versión Jif, es decir
el programa debe estar implementado acorde al sistema de anotaciones de Jif. A
partir de tales anotaciones el compilador verifica la generación de flujos de
información que incumplan la política de seguridad establecida, para reportalos
como flujos de información indebidos. 
Al ser evaluado directamente por un
compilador, obtiene los beneficios de bajo costo computacional del mismo.\newline
La generación del análisis para verificar la política de seguridad
definida, requiere dos pasos. Primero, se genera la versión Jif del aplicativo a
analizar usando el prototipo de anotación, este recibe como entrada el
código fuente del aplicativo a analizar. No requiere la espcificación de sources
y sinks.\\
Segundo, se compila el .jif, para obtener el reporte de análisis.






En el cuadro \ref{tab:comparacion} se resumen los puntos comparados
anteriormente.
\begin{table}[H]
\begin{center}
\small\addtolength{\tabcolsep}{-3pt}
\begin{tabular}{|p{2,2cm}|p{1,3cm}|p{5cm}|p{2cm}|p{2cm}|}
	\hline
	\textbf{Herramienta} & \textbf{Tipo} & \textbf{Técnicas} & \textbf{Costo
	computacional} & \textbf{ Entradas} \\
	\hline
	FlowDroid & Flujo de datos & 
	Tainting; super-grafo integrado por grafos dirigidos; Representación intermedia
	Jimple; algoritmo graph-reachability & Polinomial
	O(ED3)\cite{Graph-reachability} & apk\\
	\hline
	JoDroid & Flujo de información & PDG; slicing; Representación intermedia(SSA-
	form) & polinomial O(N)3\cite{FCO-PDG} & apk; Manifest; sources y sinks
	\\
	\hline
	Prototipo & Flujo de información  & Lenguajes tipados de seguridad; Type
	checking & Tiempo de compilación(Tiempo realmente bajo) & código fuente
	\\
	\hline
\end{tabular}
\end{center}
\caption{Generalidades técnicas de análisis evaluadas}
\label{tab:comparacion}
\end{table}	


\begin{table}[H]
\begin{center}
\small\addtolength{\tabcolsep}{-3pt}
\begin{tabular}{|p{2,3cm}|p{2cm}|p{2,5cm}|p{3cm}|p{3cm}|}
	\hline
	\textbf{Herramienta} & \textbf{ventajas } & \textbf{desventajas} 
	& \textbf{similitudes} & \textbf{diferencias}\\
	\hline
	Prototipo vs FlowDroid &  Mejor Recall.  &  Menor Precisión. &
	Detección automática de sources y sinks & Tipo de análisis(flujo de
	informacion; flujo de datos)\\
	 & Detección de flujos implícitos.  & No soporte para análisis interApp. & Bajo
	 costo en desempeño. & \\
	\hline
	Prototipo vs JoDroid & Mejor Recall. &  Menor Precisión. &  & \\ 
	 & Detección automática de sources y sinks. &  & No soporte para análisis
	 interApp &
	\\
	& Menor costo en desempeño & & Tipo de análisis IFC & Técnica de análisis: PDG,
	slicing\\
	\hline
\end{tabular}
\end{center}
\caption{Comparación técnicas de FlowDroid y JoDroid, frente a técnicas
del Prototipo.\newline Ventajas y desventajas comparando el prototipo(Propuesta
de solución)}
\label{tab:comparacion}
\end{table}	

\begin{table}[H]
\begin{center}
\small\addtolength{\tabcolsep}{-3pt}
%\begin{tabular}{|p{4cm}|p{1cm}|p{1cm}|}
\begin{tabular}{|p{4cm}|p{1cm}|p{0,5cm}|p{0,5cm}|p{0,5cm}|p{0,5cm}|p{0,5cm}|p{0,5cm}|p{0,5cm}|p{0,5cm}|p{0,5cm}|}
	\hline
	\textbf{Item} & \textbf{P vs F} & \textbf{P vs J}\\
	%\multirow{4}{*}\textbf{Item} & \textbf{P vs F} & \textbf{P vs J}\\
	\hline
	 & & \\
	\hline
	Mayor Precisión & & & & & & & & & &\\
	\hline
	Menor Precisión & & \\
	\hline
	Mayor Recall & & \\
	\hline
	Mayor Recall & & \\
	\hline
	Mayor costo en desempeño & & \\
	\hline
	Bajo costo en desempeño & & \\
	\hline
	Detección automática de sources y sinks & & \\
	\hline
	No soporte para Análisis interApp & & \\
	\hline
	Tipo de análisis(flujo de informacion; flujo de datos) & & \\
	\hline
	Tipo de análisis IFC & & \\
	\hline
	Técnica de análisis: PDG, slicing & & \\
	\hline
\end{tabular}
\end{center}
\caption{CoOMPARACION técnicas de FlowDroid y JoDroid, frente a técnicas
del Prototipo.\newline Ventajas y desventajas comparando el prototipo(Propuesta
de solución)}
\label{tab:comparacion}
\end{table}

\subsection{Qué tanto cambia la anotación del código original}
(\textbf{Martín: esta parte debería ir en esta sección, o en la
discusión?})\newline 
En el diseño de la solución se describieron los retos
técnicos\ref{sec:limitaciones} que implica anotar código Android, y cómo superar
algunos de estos. Específicamente, las limitaciones para las que se propone un
mecanismo que permita soportarlas son: Statement @Override; Casting entre tipos
EditText y View; Clase Nested R y Enhanced for loop.

% 
% Dentro de las limitaciones técnicas \ref{sec:limitaciones} descritas en el
% diseño de la solución, se describieron una serie de limitaciones para generar la
% versión Jif del programa Android a analizar. Para algunas de tales limitaciones
% se propuso un mécanismo que permita soportarlas, estas son:
% \begin{itemize}
%   \item Statement @Override
%   \item Casting entre tipos EditText y View
%   \item Clase Nested R 
%   \item Enhanced for loop
% \end{itemize}
Adicionalmente, se describió que el compilador de Jif exige la declaración del
chequeo de excepciones tipo runtime, en programas que lo
requieran\ref{subsec:consVerPol}. Ahora, cumplir con los requisitos del
compilador de Jif y aplicar mecanismos que soporten tales limitaciones, implica
una serie de transformaciones en el código fuente del programa Android a
analizar, tanto en la etapa previa a la anotación como en la anotación
misma.\newline
% para soportar tales limitaciones y
% cumplir con los requisitos del compilador de Jif, el código fuente del programa
% Android a analizar, pasa por una serie de transformaciones tanto en la etapa
% previa a la anotación como en la anotación misma.\newline
Prevía a la anotación el desarrollador debe garantizar dos cosas, primero debe
adicionar las runtime exceptions que su programa necesite, segundo cuando
requiera el uso de bucles for, debe usar la versión sencilla y no la versión
mejorada del mismo(Enhanced for loop).\newline 
Durante la anotación, además de aplicar los criterios de anotación definidos en
el diseño de la solución \ref{subsec:pasosSol}, se aplican los mecanismos
propuestos para soportar el Statement @Override, Casting entre tipos EditText y
View; y Clase Nested R. Así, en el caso de  la sentencia Statement @Override, se
comenta la línea que lo contenga; para el casting entre tipos EditText y View,
la información implicada en este tipo de casting es abstraída mediante un tipo
de dato String; finalmente para la clase nested R se crea una clase que define
los define los recursos utilizados por la aplicación a través de
variables.\newline 
Si bien, la idea que fundamenta el diseño de la solución consiste en generar la
versión Jif del aplicativo Android a analizar, lo cual se traduce en adicionar
las anotaciones correspondientes para evaluar determinada política de seguridad,
de modo que el compilador de Jif entienda el programa y permita analizar el
flujo de información en el mismo; no es suficiente con sólo anotar el código, en
otras palabras, sin los ajustes previamente mencionados, el compilador genera
error. Por otro lado, lo positivo es que tales ajustes no alteran la lógica del
programa.

\label{ch:trabajoFuturo}
\chapter{Trabajo Futuro y Conclusiones}
\section{Discusión}
Límites de la solución propuesta 

\section{Trabajo Futuro}
Cómo puede ser extendido el trabajo y qué beneficios tendría esa extensión 

\section{Conclusiones}
Qué aprendimos con este trabajo.\newline

- tipos de análisis y técnicas

- dificultades implicitas de usar Jif:\newline
-poca documentación
- a parte del trabajo que implica escribir la versión Jif para clases de la Api
de Android, hace falta soporte para una serie de clases estandar del propio
lenguane Java. \newline

- la técnica de lenguaje tipado de seguridad es muy util ..



\label{ch:anexos}
\chapter{Anexos}

\section{Diagrama de clases para el anotador}
\label{sec:diagramaClass}
\begin{figure}[H]
	\includegraphics[width=13cm]{consolidado.png}
	\caption{Clases necesarias para la implementación del anotador}
	\label{fig:classDiagram} 
\end{figure}

\section{Descripción de testcases para evaluación}
\label{tb:muestra-descripApps}
\label{sec:testcases}
En las tablas \ref{tab:descripApps1}, \ref{tab:descripApps2} y
\ref{tab:descripApps3}, se describe el comportamiento de los casos de prueba
evaluados, donde:\newline 
Se considera con nivel de seguridad alto, variables y métodos que almacenan y
modifican(respectivamente), información catalogada como privada(Sources).\newline 
Se considera con nivel de seguridad bajo, métodos para envío de mensajes y para
muestra de logs.

En los casos en que se requiere, se precisan observaciones entre los
resultados de evaluación esperados para la técnica de análisis utilizada por
FlowDroid y la técnica de análisis propuesta en el presente trabajo.

\begin{table}[H]
\small\addtolength{\tabcolsep}{-3pt}
\caption{Descripción aplicaciones de prueba}
\label{tab:descripApps1}
\begin{tabular}{|p{13cm}|p{1cm}|}
	\hline
	\multicolumn{2}{|>{\columncolor[gray]{0.8}}c|}{\textbf{AndroidSpecific\_DirectLeak1}}\\
	\hline
	\textbf{Descripción} & \textbf{Leaks}\\
	\hline
	La variable \textit{mrg} tiene un nivel de seguridad alto,
	almacena información retornada por el método source \textit{getDeviceId}. Se
	genera flujo de información directo entre información con nivel de seguridad alto e
	información con nivel de seguridad bajo, al enviar como parámetro del método
	\textit{sendTextMessage}, información de la variable \textit{mrg}. & 1 \\
	\hline
	\multicolumn{2}{|>{\columncolor[gray]{0.8}}c|}{\textbf{AndroidSpecific\_InactiveActivity}}\\
	\hline
	\textbf{Descripción} & \textbf{Leaks}\\
	\hline 
	La variable \textit{imei} tiene un nivel de seguridad alto, almacena
	información retornada por el source getDeviceId. La variable es enviada como
	parámetro a \textit{Log}, canal que muestra información con nivel de
	seguridad bajo. \textit{Observación:} debido a que la actividad en que se
	presenta este flujo de información no está activada en el Manifest de la
	aplicación, para la técnica de análisis de FlowDroid no existen leaks. Para
	nuestra propuesta de análisis si existe leak, porque se asume que los métodos y
	sus aplicaciones podrán ser ejecutados. & 0
	\\
	\hline
	\multicolumn{2}{|>{\columncolor[gray]{0.8}}c|}{\textbf{AndroidSpecific\_LogNoLeak}}\\
	\hline
	\textbf{Descripción} & \textbf{Leaks}\\
	\hline
	El caso de prueba no presenta información con niveles de seguridad alto. Se
	presentan flujos de información entre información con el mismo nivel de
	seguridad, en este caso bajo, lo cual es permitido. & 0 \\
	\hline
	\multicolumn{2}{|>{\columncolor[gray]{0.8}}c|}{\textbf{AndroidSpecific\_Obfuscation1}}\\
	\hline
	\textbf{Descripción} & \textbf{Leaks}\\
	\hline 
	La variable \textit{\textbf{mrg}} tiene un nivel de seguridad alto,
	almacena información retornada por el método source getDeviceId().
	Se genera flujo de información entre información con nivel de seguridad alto e
	información con nivel de seguridad bajo, al enviar como parámetro del método
	\textit{sendTextMessage}, información de la variable
	\textit{mrg}. \textit{Observación:} el elemento adicional para este
	testcase es proveer una suplantación de la clase
	android.telephony.TelephonyManager, en el apk de la aplicación. Para la
	evaluación que proponemos, se verifica acorde a la versión que se tiene anotada
	para esta clase, es decir, independientemente de la ofuscación de la clase,
	nuestro análisis debe detectar que existe un flujo de información indebido. &
	1\\
	\hline
	\multicolumn{2}{|>{\columncolor[gray]{0.8}}c|}{\textbf{AndroidSpecific\_PrivateDataLeak2}}\\
	\hline
	\textbf{Descripción} & \textbf{Leaks}\\
	\hline
	La variable \textit{info} tiene un nivel de seguridad alto, almacena
	información suministrada por el campo EditText de tipo textPassword. Se genera
	flujo de información entre información con nivel de seguridad alto e
	información con nivel de seguridad bajo, al pasar la variable
	\textit{info} como parámetro de \textit{Log}, que muestra
	información con nivel de seguridad bajo. & 1 
	\\
	\hline
	\multicolumn{2}{|>{\columncolor[gray]{0.8}}c|}{\textbf{ArraysAndLists\_ArrayAccess1}}\\
	\hline
	\textbf{Descripción} & \textbf{Leaks}\\
	\hline
	Se tiene un array en que se almacena información tanto proveniente como no
	proveniente de sources, parte de la información que almacena es enviada como
	parámetro del método \textit{sendTextMessage}. \textit{Observación:}
	Para la técnica de análisis de FlowDroid(taint analysis), se marca únicamente el
	índice del array donde se almacena el dato considerado como source, así,
	cuando se envía como parámetro del método \textit{sendTextMessage},
	el dato de un índice no marcado, no se genera leak. Para nuestra técnica
	de análisis(flujo de información mediante JIF), para que un array almacene
	información con nivel de seguridad alto, primero debe ser catalogo(anotado)
	con nivel de seguridad alto, lo que implica que el array podrá almacenar
	información tanto de nivel de seguridad alto como bajo, pero toda la
	información quedará con nivel de seguridad alto. En consecuencia, al enviar
	cualquier índice del array como parámetro del método 
	\textit{sendTextMessage} se presenta un flujo de información no
	permitido. & 0
	\\
	\hline
\end{tabular}
\end{table}

\begin{table}[H]
\small\addtolength{\tabcolsep}{-3pt}
\caption{Descripción aplicaciones de prueba}
\label{tab:descripApps2}
\begin{tabular}{|p{13cm}|p{1cm}|}
	\hline
	\multicolumn{2}{|>{\columncolor[gray]{0.8}}c|}{\textbf{ArraysAndLists\_ArrayAccess2}}\\
	\hline
	\textbf{Descripción} & \textbf{Leaks}\\
	\hline
	Se presenta el contexto descrito en ArraysAndLists\_ArrayAccess1, con un
	elemento adicional, se implementa el método calculateIndex(), que calcula el
	índice del array a ser enviado como parámetro del método
	\textit{sendTextMessage}. & 0 \\
	\hline
	\multicolumn{2}{|>{\columncolor[gray]{0.8}}c|}{\textbf{GeneralJava\_Exceptions1}}\\
	\hline
	\textbf{Descripción} & \textbf{Leaks}\\
	\hline
	La variable \textit{imei} es de nivel de seguridad alto, almacena información
	devuelta por el método \textit{getDeviceId}. Se genera flujo de información
	entre información de nivel de seguridad alto e información con nivel de
	seguridad bajo, al enviar como parámetro del método \textit{sendTextMessage}
	información de la variable \textit{imei}. Este flujo de información se presenta
	dentro de la captura de una excepción RuntimeException(no es verificada
	en tiempo de compilación).
	& 1
	\\
	\hline
	\multicolumn{2}{|>{\columncolor[gray]{0.8}}c|}{\textbf{GeneralJava\_Exceptions2}}\\
	\hline
	\textbf{Descripción} & \textbf{Leaks}\\
	\hline
	La variable \textit{imei} es de nivel de seguridad alto, almacena información
	devuelta por el método \textit{getDeviceId}. El control de flujo del
	programa conduce de manera implícita a la captura de una excepción tipo
	RuntimeException, desde allí se utiliza información proveída por la variable
	\textit{imei}, como parámetro para invocar el método \textit{sendTextMessage}.
	Generando un flujo de información indebido. & 1
	\\
	\hline
	\multicolumn{2}{|>{\columncolor[gray]{0.8}}c|}{\textbf{GeneralJava\_Exceptions3}}\\
	\hline
	\textbf{Descripción} & \textbf{Leaks}\\
	\hline
	La variable \textit{imei} es de nivel de seguridad alto, almacena información
	devuelta por el método \textit{getDeviceId}. La información proveída por
	\textit{imei} es utilizada como parámetro para invocar el método
	\textit{sendTextMessage} dentro de la captura de una excepción tipo
	RuntimeException, sin embargo, el programa no genera un caso que haga ejecutar
	la captura de la excepción. & 0
	\\
	\hline
	\multicolumn{2}{|>{\columncolor[gray]{0.8}}c|}{\textbf{GeneralJava\_Exceptions4}}\\
	\hline
	\textbf{Descripción} & \textbf{Leaks}\\
	\hline
	La variable \textit{imei} es de nivel de seguridad alto, almacena información
	devuelta por el método \textit{getDeviceId}. información proveída por esta
	variable es enviada como parámetro para la captura de una excepción en tiempo
	de ejecución, donde es utilizado como parámetro para invocar el método
	\textit{sendTextMessage}, generando un flujo de información indebido. & 1\\
	\hline
	\multicolumn{2}{|>{\columncolor[gray]{0.8}}c|}{\textbf{GeneralJava\_Loop1}}\\
	\hline
	\textbf{Descripción} & \textbf{Leaks}\\
	\hline
	La variable \textit{imei} es de nivel de seguridad alto, almacena información
	devuelta por el método \textit{getDeviceId}. Se generan flujos de información
	indebidos, primero al tratar de asignar la información de la variable a un
	array con nivel de seguridad bajo(donde se intenta ofuscar la información),
	luego al tratar de enviar la información ofuscada como parámetro del método
	\textit{sendTextMessage}, con nivel de seguridad bajo. & 1 \\
	\hline
	\multicolumn{2}{|>{\columncolor[gray]{0.8}}c|}{\textbf{GeneralJava\_Loop2}}\\
	\hline
	\textbf{Descripción} & \textbf{Leaks}\\
	\hline
	La variable \textit{imei} es de nivel de seguridad alto, almacena información
	devuelta por el método \textit{getDeviceId}. Se busca ofuscar la información de
	\textit{imei} mediante ciclos for anidados, allí se asigna la información de la
	variable a un array con nivel de seguridad bajo. Luego se envía la información
	ofuscada, como parámetro del método \textit{sendTextMessage}, con nivel de
	seguridad bajo, generando otro flujo de información indebido. & 1\\
	\hline
\end{tabular}
\end{table}

\begin{table}[H]
\small\addtolength{\tabcolsep}{-3pt}
\caption{Descripción aplicaciones de prueba}
\label{tab:descripApps3}
\begin{tabular}{|p{13cm}|p{1cm}|}
	\multicolumn{2}{|>{\columncolor[gray]{0.8}}c|}{\textbf{GeneralJava\_UnreachableCode}}\\
	\hline
	\textbf{Descripción} & \textbf{Leaks}\\
	\hline
	La variable \textit{deviceid} con nivel de seguridad alto, está contenida en un
	método que no es llamado, dentro del mismo, \textit{deviceid} es pasada como
	parámetro para invocar el método \textit{sendTextMessage}, cuyo nivel de
	seguridad es bajo. \textit{Observaciones:} para el análisis de FlowDroid el
	programa no presenta leaks, ya que el método nunca es llamado.
	Para nuestro análisis, el programa presenta leak porque se asume que todos los
	métodos son llamados. & 0\\
	\hline
	\multicolumn{2}{|>{\columncolor[gray]{0.8}}c|}{\textbf{ImplicitFlows\_ImplicitFlow1}}\\
	\hline
	\textbf{Descripción} & \textbf{Leaks}\\
	\hline
	 La variable \textit{imei} con nivel de seguridad alto, almacena información
	 devuelta por el método \textit{getDeviceId}, \textit{imei} se pasa como
	 parámetro al método obfuscateIMEI que devuelve la información ofuscada.
	 Después se invoca el método WriteToLog, con la información ofuscada como
	 parámetro para ser mostrada en el log. Al invocar el método WriteToLog con la
	 información ofuscada, se genera un flujo de información indebido. & 1 \\
	\hline
	\multicolumn{2}{|>{\columncolor[gray]{0.8}}c|}{\textbf{ImplicitFlows\_ImplicitFlow2}}\\
	\hline
	\textbf{Descripción} & \textbf{Leaks}\\
	\hline
	 La variable \textit{userInputPassword} con nivel de seguridad alto, almacena
	 información de un campo EditText tipo textPassword(password suministrado por
	 el usuario). Se generan flujos de información indebidos: al tratar de asignar
	 información a la variable passwordCorrect con nivel de seguridad bajo, a
	 partir de la comparación de información con nivel de seguridad alto(variable
	 textPassword), después, al tratar de mostrar en el \textit{log} información
	 que depende de tal comparación. & 1\\
	\hline
	\multicolumn{2}{|>{\columncolor[gray]{0.8}}c|}{\textbf{ImplicitFlows\_ImplicitFlow4}}\\
	\hline
	\textbf{Descripción} & \textbf{Leaks}\\
	\hline
	La variable \textit{password} con nivel de seguridad alto, almacena información
	de un campo EditText tipo textPassword, \textit{password} es utilizada como
	parte de los parámetros para invocar el método \textit{lookup} que busca
	identificar el password suministrado por el usuario. Se genera un flujo de
	información indebido, cuando se compara lo retornado por el método para mostrar
	en el \textit{log} información del password. & 1 \\
	\hline
	\multicolumn{2}{|>{\columncolor[gray]{0.8}}c|}{\textbf{Lifecycle\_ActivityLifecycle3}}\\
	\hline
	\textbf{Descripción} & \textbf{Leaks}\\
	\hline
	 El flujo de información entre información con nivel de seguridad alto e
	 información con nivel de seguridad bajo, tiene lugar a través de dos
	 métodos del ciclo de vida de la actividad: onSaveInstanceState y
	 onRestoreInstanceState. En onSaveInstanceState, se asigna información con
	 nivel de seguridad alto a la variable \textit{s}, la información que almacene
	 este método es utilizada durante la reanudación de la actividad, a través del
	 método onRestoreInstanceState, donde se muestra en el \textit{log} información
	 de la variable \textit{s}. & 1\\
	\hline
	\multicolumn{2}{|>{\columncolor[gray]{0.8}}c|}{\textbf{Lifecycle\_BroadcastReceiverLifecycle1}}\\
	\hline
	\textbf{Descripción} & \textbf{Leaks}\\
	\hline
	 Se tiene un broadcast receiver  que muestra información con nivel de
	 seguridad alto, contenida en la variable \textit{imei}(almacena información retornada por el
	 método \textit{getDeviceId}) a través del \textit{log}. & 1 \\
	\hline
	\multicolumn{2}{|>{\columncolor[gray]{0.8}}c|}{\textbf{Lifecycle\_ServiceLifecycle1}}\\
	\hline
	\textbf{Descripción} & \textbf{Leaks}\\
	\hline
	 Se tiene un servicio que presenta flujo de información indebida mediante dos
	 métodos de su ciclo de vida. En el método que inicia el servicio
	 onStartCommand, la variable con nivel de seguridad alto, almacena
	 información devuelta por el método \textit{getDeviceId}. Luego el método
	 onLowMemory, se envía información de la variable \textit{secret} a través de
	 un mensaje msm. & 1\\
	\hline
\end{tabular}
\end{table}

% \section{Formulas}
% Las formulas \ref{p} y \ref{r} son aplicadas para calcular los porcentajes
% de Precisión(p) y Recall(r).
% \begin{equation}
% \label{p}
% 	p = TP/(TP +FP) 
% \end{equation}
% \begin{equation}
% \label{r}
% 	r = TP/(TP+FN)
% \end{equation}
% 
% Donde: \newline
% TP representa el total de verdaderos positivos; FP corresponde al total de
% falsos positivos y  FN representa el total de falsos
% negativos; reportados por la herramienta.\newline

\section{Instrucciones para probar el prototipo}
\label{sec:ejecutarPrototipo}
En el directorio \small{\ttfamily{/home/testing/eule}} están los elementos
necesarios para evaluar los casos de prueba, de allí interesan los
subdirectorios \small{\ttfamily{androidFlows}},
\small{\ttfamily{InputLabelGenerator}} y el jar
\small{\ttfamily{LabelGenerator.jar}}.

El subdirectorio \small{\ttfamily{androidFlows}} contiene la estructura de
archivos necesaria para ejecutar un programa jif, así:
\small{\ttfamily{sig-src}} aloja clases java y clases de la API de Android, con
signaturas para que jif las reconozca de forma nativa.
\small{\ttfamily{jif-src/test}} tiene clases de la API de Android con
anotaciones jif(Activity.jif, BroadcastReceiver.jif, Log.jif, R.jif,
Service.jif, SmsManager.jif). Allí se deben alojar los programas jif a
ejecutar.\newline 
En {\small{\ttfamily{InputLabelGenerator}}} están los fuentes java a pasar como
entrada para el generador de labels(LabelGenerator.jar), que devuelve la
versión jif de los mismos. Se recomienda utilizar estos, ya que contienen las
adaptaciones necesarias para poder ser analizadas con JIF, la adición de
excepciones NullPointerException, ClassCastException y
ArrayIndexOutOfBoundsException, son algunos ejemplos de elementos adicionados.

 \textbf{Instrucciones de
ejecución:}\newline \textbf{(1)} Ejecutar el jar para la generación de los labels:
\begin{lstlisting}
testing@debianJessie:~/eule$ java -jar LabelGenerator.jar
\end{lstlisting}
Una vez se ejecuta el .jar, se solicita el directorio de entrada(que contiene
las aplicaciones a anotar) y el directorio de salida(para alojar las
aplicaciones anotadas). Separados por el simbolo @
\begin{lstlisting}
Ingrese la ruta completa para el directorio de entrada, y para el 
directorio de salida:
Ejemplo: dir-entrada@dir-salida 
\end{lstlisting}
Se deben pasar los directorios:
\begin{lstlisting}
InputLabelGenerator@androidFlows/jif-src/test/
\end{lstlisting}

\textbf{(2)} ejecutar el script setup.sh(basta con ejecutarlo una sola vez)
\begin{lstlisting}
testing@debianJessie:~/eule/androidFlows$ ./setup.sh
\end{lstlisting}

\textbf{(3)} 
Ejecutar el .jif generado:\newline
En la ruta pasada como directorio de salida en el punto anterior
{\small{\ttfamily{androidFlows/jif-src/test}}}, se genera un subdirectorio por
aplicación, con un .java y un *-out.jif. Se debe ejecutar el *-out.jif. Por
ejemplo, para evaluar el testcase ArraysAndLists\_ArrayAccess1:
\begin{lstlisting}
testing@debianJessie:~/eule/androidFlows$ ./jifc-java-libraries.sh \
jif-src/test/ArraysAndLists_ArrayAccess1/ArrayAccess1-out.jif
\end{lstlisting}
Cuando se presentan flujos indebidos, el compilador genera una salida señalando
los problemas de seguridad.
\lstset{
    language=bash,
    basicstyle=\tiny,
  }
\begin{lstlisting}
estudiante@debianJessie:~/eule/androidFlows$ ./jifc-java-libraries.sh \
jif-src/test/ArraysAndLists_ArrayAccess1/ArrayAccess1-out.jif 
/home/testing/eule/androidFlows/jif-src/test/ArraysAndLists_ArrayAccess1/ArrayAccess1-out.jif:51:
Unsatisfiable constraint
    	general constraint:
    		actual_arg_3 <= formal_arg_3
    	in this context:
    		{Alice->; _<-_ ⊔ caller_pc} <= {}
    	cannot satisfy equation:
    		{Alice->} ⊑ {}
    	in environment:
    		{this} ⊑ {caller_pc}
    		[]

    Label Descriptions
    ------------------
     - actual_arg_3 = the label of the 3rd actual argument
     - actual_arg_3 = {Alice->; _<-_ ⊔ caller_pc}
     - formal_arg_3 = the upper bound of the formal argument text
     - formal_arg_3 = {}
     - caller_pc = The pc at the call site of this method (bounded above by 
    {})
     - this = label of the special variable "this" in test.ArrayAccess1

    The label of the actual argument, actual_arg_3, is more restrictive than 
    the label of the formal argument, formal_arg_3.
            sms.sendTextMessage("+49 1234", null, arrayData[2], null, null);
                                                  ^-------^

1 error.
testing@debianJessie:~/eule/androidFlows$
\end{lstlisting}

Cuando el caso de prueba no presenta flujos indebidos, el compilador no genera
salidas, por ejemplo, al evaluar el testcase AndroidSpecific\_LogNoLeak, el
compilador retorna el prompt de la shell, sin ningún comentario.

\section{Instrucciones para uso de FlowDroid}
En el directorio \small{\ttfamily{/home/estudiante/eule}} también se encuentran
los subdirectorios \small{\ttfamily{/FlowDroid}} y
\small{\ttfamily{/DroidBench-master}} que contienen los elementos necesarios
para probar los testcases con FlowDroid.\newline 
Para ello se requiere ejecutar el jar de FlowDroid, indicando el apk a analizar,
los apk están en el subdirectorio \tetxtit{DroidBench-master}.
Por ejemplo, para analizar el testace ImplicitFlow4:
\begin{lstlisting}
testing@debianJessie:~/eule/FlowDroid$ java -jar FlowDroid.jar \
../DroidBench-master/apk/ImplicitFlows/ImplicitFlow4.apk
/home/estudiante/android-sdks/platforms/
\end{lstlisting}
El archivo \small{\ttfamily{howRunIt}} contenido en el directorio
\small{\ttfamily{/FlowDroid}}, indica como se debe ejecutar.
 \bibliographystyle{IEEEtran} 
 \bibliography{referencias}

\renewcommand{\listtablename}{ÍNDICE DE TABLAS}\listoftables
%Modificar el nombre a mayúsculas
\renewcommand{\listfigurename}{ÍNDICE DE GRÁFICAS}\listoffigures
%hacer que el título del capitulo aparezca en mayuscula
\end{document}
