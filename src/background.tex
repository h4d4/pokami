\section{Background}
\label{sec:back}

\subsection{Aplicaciones Android}
código con ejemplo de componentes(Clases
Activity, ..) 

\subsection{Estructura de trabajo en JIF}
- estructura de los directorios del compilador Jif y estructura de trabajo en
Jif(para entender cómo funciona y cómo afecta el diseño de la
solución).

\subsection{Sintaxis de Anotación en Jif}
\label{subsec:JifSintax}
-Definición de variables: \newline 
\emph{ type\{L\} varName; }\newline 
donde type especifica el tipo de dato que
almacena la variable, \{L\} el label de seguridad  para especificar quien es el
dueño de la variable, y name, el respectivo nombre de la variable.

-Definición de arrays:\newline
en jif un array cuenta con dos labels de seguridad, Base Label(BL) y Size
Label(SL). BL indica el nivel de seguridad de los elementos que almacena el
array, controlando quien puede conocer la información del mismo. SL especifica
quienes pueden conocer la número de elementos almacenados.

-Definición de métodos.\newline
\emph{ type \{RTL\} methodName \{BL\} (arg1\{AL\},,, argn\{AL\}) :\{EL\}
}\newline 
RTL, Return Type Label, indica el label de seguridad con que
queda el tipo de dato devuelto por el método.\newline 
BL begin label, representa el máximo nivel se seguridad del pc label desde donde
se invoca el método, de este modo, el program counter label desde donde
se invoca el método debe ser menor o igual de restrictivo que el BL del
método.\newline 
AL argument label, indica el máximo nivel de seguridad  para los argumentos con
que se llama el método, así, los labels de los argumentos con que se invoca el
método deben ser menor o igual de restrictivos que los AL con que han
definido el método.\newline
EL end label, indica el pc label en el punto de terminación del método, y
representa la información que puede ser conocida.\newline
Cuando un label no es especificado, Jif define unos por defecto. En el caso de
RTL, jif hace un join entre los diferentes AL con que ha sido definido el
método.\newline