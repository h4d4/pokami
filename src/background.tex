\section{Background}
\label{sec:back}

\subsection{Aplicaciones Android}
Explicar composición de aplicaciones Android, actividades, servicios, etc.

\subsection{Estructura de trabajo en JIF}
- estructura de los directorios del compilador Jif y estructura de trabajo en
Jif(para entender cómo funciona y cómo afecta el diseño de la
solución).

\subsection{Sistema de anotaciones en Jif}
Jif es un lenguaje tipado de seguridad que extiende al lenguaje Java con labels
de seguridad, a través de los cuales se especifican restricciones de cómo
debería ser utilizada la información.\newline 
Jif está compuesto por un compilador y un sistema de anotaciones.\newline
% El sistema de anotaciones está basado en un modelo de etiquetas
% DLM(Decentralized Label Model), donde existen principals, políticas y
% labels.\newline
Para hacer seguimiento al control de flujo de información de un programa
implementado en Java, se debe implementar la respectiva versión Jif, es decir,
la versión del programa donde mediante el sistema de anotaciones se especifican
las políticas de seguridad a evaluar.\newline
Básicamente, la implementación de programas en Jif, consiste en adicionan labels
de seguridad a la definición de métodos, variables, arrays, etc. Los labels de
seguridad que no se especifiquen son generados automáticamente con labels por
defecto.
%http://www.cs.cornell.edu/jif/doc/jif-3.3.0/language.html#inference.

Luego, la versión Jif del programa se compila con el compilador de Jif.
Este aplica chequeo de labels(label checking)\cite{jifRef} para verificar que
los flujos de información que se generan al interior del programa, cumplen con
las políticas definidas.

\subsubsection{DML(Decentralized Label Model)}
Jif basa su sistema de anotaciones en el modelo de etiquetas DLM(Decentralized
Label Model), donde se manejan tres elementos fundamentales: Principals,
Políticas y Labels.\newline
Principals: un principal es una entidad con autoridad para observar y cambiar
aspectos del sistema. Un programa pertenece a un principal, quien determina el
comportamiento que este debería tener. Jif cuenta con una serie de principals ya
definidos, por ejemplo, Alice, Bob, Chunck, etc, que pueden ser
utilizados al momento de anotar.\newline 
Políticas: mediante políticas de seguridad el dueño de la política, que es el
principal que la define, determina qué otros principals pueden leer o
influenciar la información. Así, una política puede ser de confidencialidad o de
integridad, y se especifican de la forma: \{owner: reader list\} u
\{owner: writer list\}.\newline 
Labels: un label consiste en un conjunto de políticas de confidencialidad e
integridad. Los labels se escriben en las expresiones del progrma que se
anota(labels de seguridad), esto es métodos, variables, arrays, etc..\newline 
En sisntesís, las políticas de seguridad definen que principals pueden leer o
modificar la información, y esas políticas se expresan mediante labels.

\subsubsection{Label Checking}
Para hacer seguimiento al flujo de información de un programa, el compilador de
Jif asocia un label al program counter de cada punto del programa,
progam-counter label(\underline{pc}). En cada punto del programa, el
(\underline{pc}) representa la información que podría conocerse tras la
ejecución de ese punto del programa.
El (\underline{pc}) es afectado por los labels con que se define cada sentencia
y expresión del programa, por tanto este es considerado como el límite
superior(máxima información que podría conocerse) de los labels que han afectado
el flujo de información para llegar a un determinado punto de ejecución.\newline
Adicionalmente, jif define labels que representan la información que podría
conocerse tras la terminación normal, o terminación por excepción de las
sentencias del programa. Y labels enviroments, que para cada punto del programa
determinan la forma en que se relacionan labels y principals.\newline
El valor dichos labels es verificado durante la compilación del programa, si se
detecta que no cumplen con las restricciones establecidas en la anotación del
mismo, el compilador genera error, indicando los puntos del programa que las
incumple.\newline

-Implicit flows and program-counter labels\newline
- sobreescritura de métodos

\subsubsection{Sintaxis de labels}
Como se representa leer y escribir

\subsubsection{Sintaxis de Anotación en Jif}
\label{sssec:JifSintax} 

-Definición de variables: \newline 
En Java la sintaxis para definir una variable es: \emph{ modifier java-type
varName}\newline Extendiendo la sintaxis Java, en Jif las variables se definen de la forma:\\
\emph{ modifier java-type\{L\} varName}\newline 
donde java-type especifica el tipo de dato Java que almacena la variable, \{L\}
el label de seguridad  para especificar quien es el dueño de la variable, y
varName, el respectivo nombre de la variable.

-Definición de arrays:\newline
En Java un array se define de la forma:\\
\emph{modifier java-type [ ] nameArray}\\
En Jif, además del tipo de dato Java(java-type) de los elementos almacenados en
el array, se deben especificar dos labels de seguridad: Base Label(BL) y Size
Label(SL). BL indica el nivel de seguridad de los elementos que almacena el
array, controlando quien puede conocer la información del mismo. SL especifica
quienes pueden conocer la número de elementos almacenados. Así, la sintaxis para
anotar el array es:\\
\emph{java-type\{BL\} [ ]\{SL\} nameArray}

-Definición de métodos.\newline
En Java la definición de un método tiene la siguiente sintaxis:\newline
\emph{modifier java-type methodName(java-type arg1,,, java-type
argn)\{body method\}}
En Jif se debe asociar un label de seguridad al tipo de dato retornando, los
argumentos que recibe y las excepciones declaradas.
Adicionalmente, se declara un begin-label(BL) y un end-label(EL). La sintaxis es
la siguiente:\newline 
\emph{ modifier java-type\{RTL\} methodName \{BL\}(java-type arg1\{AL\},,,
java-type argn\{AL\}) :\{EL\} }

Donde: \emph{java-type}, es el tipo de dato Java retornado por el
método.\newline 
\emph{RTL}, Return Type Label, indica el label de seguridad para el valor
devuelto por el método.\newline 
\emph{BL}, Begin Label, representa el máximo nivel se seguridad del
\underline{pc} desde donde se invoca el método, de este modo,
el program counter label desde donde se invoca el método debe ser menor o igual
de restrictivo que el BL con que se define el método. El BL también asegura que
el método sólo podrá actualizar partes del programa que tengan igual BL. Con
tales restricciones se evita la generación de flujos implícitos, vía invocación
del método.\newline
\emph{AL}, Argument Label, indica el máximo nivel de seguridad para los
argumentos con que se llama el método, así, los labels de los argumentos con que se
invoca el método deben ser menor o igual de restrictivos que el AL con que ha
definido el método.\newline 
\emph{EL}, End Label, indica el \underline{pc} en el punto de terminación del
método, y representa el máximo nivel de información que puede conocerse tras la
finalización del método.

\subsubsection{Labels de anotación que Jif asume por defecto}
\label{sssec:default-labels}
Cuando un label no es especificado, Jif define unos por defecto. 
- Labels por defecto para la declaración de métodos\newline
En el caso de
RTL, jif hace un join entre los diferentes AL con que ha sido definido el
método.

- Labels por defecto para \newline

\subsubsection{Llamada a métodos}