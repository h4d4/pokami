\subsection{Técnicas de análisis de código}
\label{subsec:contexto}
\subsubsection{Análisis estático y dinámico}

Las soluciones propuestas para detectar fuga de información en aplicaciones
Android, se enmarcan en el análisis estático o dinámico de la aplicación, en
algunos casos se combinan ambos tipos.\newline 
En \textbf{análisis estático}\cite{Static-dynamic}, se estudia el código del
programa para inferir todos los posibles caminos de ejecución. Esto se logra
construyendo modelos de estado del programa, y determinando los estados posibles
a alcanzar por el programa.
No obstante, debido a que existen múltiples posibilidades de ejecución, se opta
por construir un modelo abstracto de los estados del programa. La consecuencia
de tener un modelo aproximado es pérdida de información y posibilidad de menor
precisión en el análisis.\newline 
Por otro lado, en \textbf{análisis dinámico} se ejecuta el programa y se analiza
su comportamiento, verificando el camino de ejecución que ha tomado el programa.
Esa exactitud en la ejecución que se verifica da precisión al análisis, porque
no es necesario construir un modelo aproximado de todos los posibles caminos de
ejecución.

Aunque los resultados del análisis estático pueden perder precisión, la ventaja
es que son generalizables, porque el modelo construido representa una
descripción del comportamiento del programa, independientemente de las entradas
y el contexto en que este se ejecute. Ahora, con el análisis dinámico, no es
posible generalizar sus resultados para futuras ejecuciones, porque no
existen garantías de que las entradas con que fue ejecutado el programa,
contengan características para todos los posibles caminos de ejecución.\newline 
Además de las ventajas y desventajas de ambas clases de análisis, cada uno
implica su propio reto. Mientras en el análisis estático la dificultad está
en construir el modelo de abstracción adecuado, en el análisis dinámico, es
complejo encontrar un conjunto de casos de prueba representativo.\newline
%, a analizar durante la ejecución del programa.\newline
Por otra parte, dependiendo de la finalidad con que se detecte la fuga de
información, un tipo de análisis puede ser más apropiado que otro. Si se busca
contener la fuga de información a tiempo de ejecución, análisis dinámico es el
camino apropiado. 
De lo contrario, si se busca garantizar que a tiempo de
ejecución la aplicación no incurre en fugas de información, resulta más
conveniente aplicar análisis estático, porque cumplir con tales garantías
implica definir políticas de confidencialidad y/o integridad desde la
implementación de la
aplicación\cite{DD2460}\cite{information-flow-control}.\newline 
% Adicionalmente, debido a que una propiedad de seguridad no se prueba con
% monitoreo, es decir, no se prueba durante la ejecución del programa, resulta
% pertinente la aplicación de análisis estático, porque a través de este tipo
% de análisis que se busca probar el cumplimiento de la propiedad
% \cite{volpano}[page 4].


Precisamente, el propósito fundamental del presente trabajo es ofrecer al
desarrollador de aplicaciones Android una herramienta para aplicar políticas de
confidencialidad en la aplicación que implementa, así, la aplicación se
ejecutará exitosamente, si y sólo si, cumple con las políticas definidas, de lo
contrario, el desarrollador puede revisar y corregir su código.\newline
 
\subsubsection{Técnicas utilizadas en análisis estático } 
Generalmente, para verificar el cumplimiento de políticas de seguridad mediante
análisis estático, se aplican técnicas de seguridad de tipado
(Typed-Inference/Security-Typed Analysis) y técnicas de flujo de
datos(Data/Control Flow Analysis)\cite{Information-Flow-Java}.\newline 
Con \textbf{técnicas Security-Typed} las propiedades de confidencialidad e
integridad son anotadas en el código, y verificadas a tiempo de compilación,
garantizando su cumplimiento a tiempo de ejecución.\newline 
Con \textbf{técnicas de flujo de control} y \textbf{técnicas de flujo de datos},
las políticas de seguridad son verificadas haciendo seguimiento al control de
flujo, o al flujo de datos, respectivamente. Estás técnicas suelen utilizar
grafos de Control de Flujo CFG(Control Flow Graph), Grafos de Flujo de Datos
DFG( Data Flow Graph) y Grafos de llamadas CG (Call Graphs).

Acorde a literatura científica en el ámbito de seguridad de aplicativos
Android, parte importante de las propuestas para análisis de fuga de
información(TaintDroid\cite{TaintDroid}, Flow-Droid\cite{FlowDroid-Thesis},
DidFail\cite{DidFail}, DroidForce\cite{DroidForce}), parten del bytecode para
realizar data-flow analysis, mediante técnicas de análisis tainting. Las
técnicas de análisis tainting, son un tipo especial de análisis de flujo de
datos, donde se hace seguimiento al flujo de datos entre un conjunto de fuentes
consideradas privadas y/o sensibles; y un conjunto de destinos considerados no
confiables, sources y sinks, respectivamente.\newline 
Si bien, tales propuestas permiten detectar flujos de datos indebidos en
aplicaciones Android, están enfocadas en analizar aplicaciones ya implementadas,
y no, en garantizar el cumplimiento de determinadas políticas de seguridad desde
su construcción.
% Tales propuestas se enfocan en analizar aplicativos de terceros para detectar
% flujos de datos indebidos, y no: para garantizar el cumplimiento de determinadas
% políticas de seguridad. 
% En consecuencia, es complejo que el desarrollador
% garantice la ausencia de fugas de información en la aplicación que implementa,
% partiendo de tales herramientas. 
% 
% Puesto que, al seguir únicamente a los datos
% marcados, los datos no marcados para el análisis, pueden acarrear fugas de
% información(under-tainting). Adicionalmente, si no se hace seguimiento al flujo
% de control pueden existir fugas de información a través de flujos implícitos,
% ya que, el análisis estará centrado en flujos explícitos.\newline
% No obstante, las limitaciones propias de un análisis basado en flujo de datos,
% pueden superarse enfocando el análisis de la aplicación hacia técnicas de
% análisis basadas en control de flujo de información, ya que estas analizan el
% aplicativo de forma estática para identificar todos los posibles caminos que
% podría tomar la aplicación en tiempo de ejecución. 

Precisamente, mediante análisis basado en control de flujo de información y
técnicas Security-Typed, es posible garantizar el cumplimiento de políticas de
seguridad en las aplicaciones que se implementa, puesto que,
% Así, con análisis basado en
% control de flujo de información, no sólo es posible prevenir fugas por
% under-tainting y flujos implícitos; sino que también, es posible ofrecer
% garantías del cumplimiento de determinadas políticas de seguridad.\newline 
% Así, con análisis basado en
% control de flujo de información es posible garantizar el cumplimiento
% de determinadas políticas de seguridad, desde la construcción del
% aplicativo.\newline 
% Ahora, 
las reglas para evaluar control de flujo de
información pueden definirse mediante técnicas Security-Typed, por ejemplo como
se definen con Jif \ref{JIF-Tool}, un lenguaje tipado de seguridad para realizar
control de flujo de información en aplicativos Java.
% 
% el inconveniente es que está implementada para
% aplicaciones en Java, y no para aplicativos Android.\newline 
\subsubsection{Security Typed Languages}
Las herramientas basadas en técnicas de análisis Security-Typed, involucran
conceptos como flujo de información, políticas de confidencialidad e integridad,
y chequeo de tipos.

\emph{Flujo de información}: el flujo de información describe el
comportamiento de un programa, desde la entrada de los datos hasta la salida de
los mismos. 

\emph{Políticas de confidencialidad e integridad}: confidencialidad e integridad
son políticas de seguridad aplicables mediante control de flujo de información.
Mientras la confidencialidad busca prevenir que la información fluya hacia
destinos no apropiados, la integridad busca prevenir que la información provenga
de fuentes no apropiadas\cite{LanguageIFS-2013}.\newline
% Una importante diferencia
% entre confidencialidad e integridad, es que la integridad de la información de
% un programa puede ser alterada sin la interacción con agentes externos.\newline %\textcolor{red}{(por qué es importante?)}
Ambas políticas son fundamentales para garantizar propiedades de
seguridad.\newline 
Con políticas de confidencialidad, es posible garantizar ausencia de fugas de
información. Con políticas de integridad, la finalidad es evitar
modificación de la información, de forma no consentida.\newline 
Verificar que un programa utilice la información acorde a
tales políticas, implica analizar sus flujos de información de inicio a fin.
Para tal análisis se deben definir: políticas de flujo de información y
controles de flujo de información, es decir, las políticas de seguridad a
evaluar y los mecanismos para aplicarlas. 

\emph{Chequeo de tipos}: al usar un lenguaje tipado de seguridad, las políticas
son definidas a través del lenguaje, porque son expresadas mediante anotaciones
en el código fuente del programa a verificar, y su evaluación se realiza
mediante chequeo de tipos.\newline 
El chequeo de tipos consiste en una técnica estática,
también utilizada para analizar flujo de información durante la compilación de
un programa, más específicamente en la etapa de análisis semántico, el
compilador identifica el tipo para cada expresión del programa y verifica que
corresponda al contexto de la expresión.\newline
%  El
% chequeo de tipos también es una técnica estática utilizada para analizar flujo
% de información durante la compilación de un programa, más específicamente en la
% etapa de análisis semántico, el compilador identifica el tipo para cada
% expresión del programa y verifica que corresponda al contexto de la expresión.
Bajo este principio de chequeo, lenguajes tipados de seguridad aplican
políticas de control de flujo, definiendo para cada expresión del programa un
tipo de seguridad(security type), de la forma:  tipo de dato y label de
seguridad(security label). Donde el label de seguridad regula el uso del dato,
acorde a su tipo.\newline 
El compilador realiza el chequeo de tipos, partiendo del conjunto de labels de
seguridad. Así, si el programa pasa el chequeo de tipos y compila correctamente,
se espera que cumpla con las políticas de control de flujo evaluadas.