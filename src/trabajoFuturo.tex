\label{ch:trabajoFuturo}
\chapter{Trabajo Futuro y Conclusiones}
\section{Discusión}
\subsection{Límites de la solución propuesta}
Las limitaciones del diseño de solución en que se enfoca el presente trabajo, se
enmarcan en políticas de seguridad evaluables, tipos de aplicaciones evaluables 
y características del lenguaje Jif.

\emph{Políticas de seguridad evaluables}\\
Se propone un esquema de anotación con niveles de seguridad alto y
bajo, que permite definir y evaluar políticas de confidencialidad en aplicativos
Android mediante el sistema de anotaciones de Jif.
Sin embargo, el esquema de anotación propuesto no permite evaluar políticas de
integridad ni aplicar mecanismos Downgrading, características ofrecidas por el
sistema de anotaciones de Jif.

\emph{Tipos de aplicaciones evaluables}\\
El diseño de solución en que se hace énfasis \ref{sec:sol-desig}, para la
herramienta de análisis propuesta, no permite hacer análisis de flujo de
información vía interApp, es decir, no se hace seguimiento al flujo de
información durante el envío de mensajes intents para activar componentes de
aplicaciones externas.
%\textcolor{red}{POR QUE???}

\emph{Características del lenguaje Jif}\\
Con limitaciones propias del lenguaje Jif se hace referencia a las
características del lenguaje Java estándar que aún no están soportadas por el
compilador de Jif, y que por tanto, impiden analizar el flujo de información de
aplicativos Android que requieran de tales características del lenguaje Java, a
menos que se encuentre sintaxis equivalente para soportar tales
características.\newline 
Más específicamente, se hace referencia a las limitaciones descritas en
\ref{sec:limitaciones-tec}: nested clases, initializer blocks, threads, etc.

\subsection{Retos para el análisis}
El análisis de flujo de información en aplicativos Android mediante el sistema
de anotaciones de Jif, involucra una serie de retos como consecuencia de: las
diferencias entre una aplicación Android y una aplicación Java convencional; y
las limitaciones propias del lenguaje Jif.\newline 
Por un lado, Jif permite anotar código Java pero no código Android, es decir,
las anotaciones Jif son válidas para clases del lenguaje Java estándar, no para
clases específicas de la API del framework Android.\newline 
Ahora bien, aunque en esencia una aplicación Android es una aplicación Java con
interfaces descritas en sintaxis XML, una aplicación Android requiere de las
clases proveídas por el framework para la implementación de
funcionalidades.\newline 
Así, para analizar aplicativos Android con Jif, se
necesita anotar clases de la API Android mediante el sistema de anotaciones de Jif, de modo que sean
reconocidas para el análisis de flujo de información. 

Por otro lado, la anotación de las clases de la API Android(que en el fondo
están implementadas en lenguaje Java), están limitadas a las características del
lenguaje Java estándar reconocidas por el compilador de Jif.

Entre los retos que surgen están:\newline 
\emph{Clases de la API del framework y su soporte para Jif}\newline 
Anotar las clases de la API Android para el sistema de anotaciones de Jif no
es una tarea trivial debido a la cantidad(1) y a características específicas
del framework(2).

(1) Cantidad de clases a anotar: para analizar flujo de información en
aplicativos Android, lo ideal sería que todas las clases que conforman la API de
Android estuviesen anotadas a través del sistema de anotaciones de Jif, no
obstante, como la cantidad de clases que conforman la API es bastante amplía, en
la presente tesis se anotan únicamente las requeridas para verificar una
política de seguridad específica.
 
(2) Características específicas del framework: entre las funcionalidades del
framework está proveer las clases necesarias para interactuar con las interfaces
XML. No obstante, las clases involucradas para tales propósitos incluyen
operaciones específicas del framework no soportadas mediante el sistema de
anotaciones de Jif, puesto que, implican características del lenguaje Java
estándar que no son soportadas. Para las cuales, se adoptan mecanismos que
permitan analizar la información, tal como se ilustra en las limitaciones
técnicas \ref{sec:limitaciones-tec}. Tales características incluyen:\newline 
- Casting entre tipos EditText y View: como se describe en
\ref{sec:casting}, para procesar información proveniente de campos de interfaces XML, se requiere
hacer casting entre los tipos de datos representados por las clases EditText y
View. Sin embargo, este tipo de casting no es reconocido por el compilador
Jif.\newline 
- Clase Nested R: como se describe en \ref{sec:nested}, para referenciar los
recursos(strings, estilos, widgets, layouts, e interfaces xml) necesarios para
una aplicación, el framework genera la clase R.java. Ahora bien, el
inconveniente para anotar ese tipo de clases con Jif, es que los identificadores
son descritos mediante clases nested, y esa característica del lenguaje Java
está dentro de las limitaciones del lenguaje Jif.\newline 
- Sobreescritura de componentes: en la sección \ref{sub:override} se expone
cómo para la implementación de aplicativos en Android es necesaria la
sobreescritura de componentes específicos de la API Android, componentes como
(activities, content providers, receivers, services), y cómo para indicar la
sobreescritura de los respectivos métodos de un componente se requiere el
Statement @Override cuya clase no es soportada por el sistema de anotaciones de
Jif.

\newpage
\emph{Características específicas del lenguaje Java estándar}\newline
Otro reto importante para analizar aplicativos Android a través del sistema de
anotación de Jif, es que la sintaxis del lenguaje java en la implementación del
aplicativo se restringe a la sintaxis soportada por Jif, por ejemplo: Jif no
soporta el enhanced for loop \ref{seubsec:enh}. En consecuencia, habría que
extender el compilador de Jif para que efectivamente reconozca características
adicionales del lenguaje Java.

En síntesis, los retos emergentes implican extensiones al sistema de
anotaciones de Jif, tanto para soportar clases específicas del framework de
Android, como para soportar características del lenguaje Java estándar no
soportadas por Jif.\newline
Adicionalmente, para las características del framework Android que
definitivamente no se puedan soportar mediante el sistema de anotaciones de Jif,
es necesario adoptar mecanismos que permitan analizar el flujo de información en
las aplicaciones que implementen tales características.\newline

\subsection{Anotación por parte del desarrollador}
\label{subsec:cambios}
Para implementar aplicaciones en el sistema de anotaciones de Jif, el
desarrollador debe definir un elemento adicional: la declaración de
excepciones tipo Runtime \ref{subsec:consVerPol}, que en aplicaciones Java no
son verificadas a tiempo de compilación, haciendo que su definición no sea requerida. 
Por el contrario, en Jif si un programa requiere excepciones
NullPointerException, ClassCastException y/o ArrayIndexOutOfBoundsException, el
desarrollador debe declararlas, de lo contrario, el compilador Jif genera
error.
En consecuencia, las aplicaciones Android a analizar deben tener definidas
excepciones tipo runtime.

\section{Trabajo Futuro}
% \textcolor{red}{Cómo puede ser extendido el trabajo y qué beneficios tendría esa
% extensión}

Exceptuando las características del lenguaje Java que no son soportadas por el
compilador de Jif(nested clases, initializer blocks, threads, como se ilustra en
 \ref{sec:limitaciones-tec}), se podría ampliar el setup de Jif para clases de
 la API de Android, de modo que se brinde soporte mediante el sistema de anotaciones de
Jif a la mayor cantidad de clases posibles de la API.
Esto permitiría hacer análisis de flujo de información a aplicaciones
Android más robustas. Por ejemplo, si se anotan todas las clases
correspondientes para el manejo de intents(mensajes utilizados principalmente
para comunicar diferentes aplicaciones), sería posible incluir en el análisis de
flujo de información, el análisis vía interApp. 

El esquema de anotación propuesto podría ser
extendido para definir y analizar políticas de integridad y mecanismos
adicionales como declasificación y endorsement, verificables mediante el modelo
de anotaciones de Jif. De este modo, el desarrollador también podría garantizar
el cumplimiento de políticas de integridad, y contaría con mecanismos que le
permitan flexibilizar la definición de las políticas tanto de confidencialidad
como de integridad.

\section{Conclusiones}
El presente trabajo de investigación ha abordado la problemática enfrentada por
el desarrollador de aplicaciones Android, a la hora de definir políticas de
seguridad que regulen el flujo de información de sus aplicaciones. Puesto que,
aún cuando la API Android ofrece mecanismos de control de acceso y el
desarrollador puede implementarlos en sus aplicaciones, estos se centran en
regular el acceso de los usuarios a determinados recursos del sistema, y no en
verificar qué sucede con la información una vez es accedida.

Buscando contribuir con la solución de tal problemática, se propone una
herramienta de análisis estático basada en el sistema de anotaciones de Jif, que
permita analizar flujo de información en aplicativos Android. El diseño ideal
para la propuesta de solución, implica extender el setup de Jif para la API de
Android e incluir un clasificador de sources y sinks. Sin embargo, para efectos
de la presente tesis se limita el setup y el conjunto de sources y sinks, acorde
a una política de seguridad específica.

El diseño de solución en que se hace énfasis para la herramienta de análisis
estático, es evaluado y los resultados obtenidos son comparados frente a otras
herramientas de análisis estático: FlowDroid y Jodroid. Partiendo de los tipos
de análisis y técnicas evaluadas, de sus ventajas y desventajas, se
puede concluir:\newline 
- Con el sistema de anotaciones de Jif es posible proveer una
herramienta de apoyo al desarrollador de aplicaciones Android, de tal manera que evalúe el
cumplimiento de políticas de seguridad desde la construcción de sus aplicativos.\\
No obstante, el desarrollador debe adquirir un conocimiento previo de la
implementación de aplicativos en Jif.

- Al estar basado en análisis de flujo de información, Jif analiza tanto flujos
explícitos como flujos implícitos, ofreciendo la ventaja de detección de fugas
de información a través de flujos implícitos, sin requerirse trabajo adicional
para que el análisis incluya tales flujos. Contrario a lo que sucede con las
técnicas de análisis tainting, pues para incluir flujos implícitos en el
análisis, se requiere especificar casos que propaguen el marcado de datos a
través de dichos flujos.

- Al tratarse de análisis de flujo de información mediante lenguajes tipados de
seguridad, se obtienen las ventajas de desempeño y completitud en el análisis,
pero al mismo tiempo se obtiene como desventaja una baja precisión.\newline 
Las ventajas de desempeño obedecen a que el análisis se centra en técnicas de
chequeo de tipos(label checking), que al corresponder a etapas de compilación
dan como resultado bajo costo en desempeño.\newline
La completitud en el análisis se obtiene  haciendo seguimiento al flujo de
información de inicio a fin del aplicativo\cite{LanguageIFS-2013}, incluyendo
tanto flujos implícitos como flujos explícitos, generando así, un menor 
reporte de falsos negativos.\newline 
La baja precisión en el análisis obedece a
un enfoque de análisis pesimista, en el que, al incluir todas las posibles ramas de ejecución, se generan más
falsos positivos.

- Además de las ventajas y desventajas, el análisis de flujo de información en
aplicativos Android por medio del sistema de anotaciones de Jif,
comprende varios retos que implican extensiones al sistema de anotaciones de
Jif, tanto para soportar clases específicas del framework de Android, como para
soportar características del lenguaje Java estándar no soportadas por Jif.\newline 
Adicionalmente, para las características del framework Android que
definitivamente no se puedan soportar mediante el sistema de anotaciones de Jif,
es necesario adoptar mecanismos que permitan analizar el flujo de información en
las aplicaciones que las requieran.

- En el presente trabajo de investigación se dan los primeros pasos para el
análisis de flujo de información de aplicativos Android mediante el sistema de
anotaciones de Jif, porque Jif permite anotar código Java pero no código
Android, es decir, las anotaciones Jif son válidas para clases del lenguaje Java
estándar, no para clases específicas de la API del framework Android, las cuales
son indispensables para implementar las funcionalidades de aplicativos Android.

- Finalmente, la presente tesis hace un aporte importante al brindar una
herramienta para análisis de flujo de información de aplicativos Android
mediante el sistema de anotaciones de Jif, la cual permite que el desarrollador
analice flujos de información de la aplicación que implementa, para verificar el
cumplimiento de políticas de confidencialidad.

% la sirve como apoyo en las labores de verificación de las
% 
% cual permite que el desarrollador verifique el
% cumplimiento de políticas de confidencialidad en los aplicativos que implementa.


% en En este orden de ideas, se hace un aporte importante, al brindar una herramienta de apoyo al desarrollador, que
% posibilita el análisis de flujo de información en Aplicativos Android mediante
% el sistema de anotaciones de Jif.

% Finalmente, partiendo del hecho de que Jif permite anotar código Java pero no
% código Android, es decir, las anotaciones Jif son válidas para clases del
% lenguaje Java estándar, no para clases específicas de la API del framework
% Android, las cuales son indispensables para implementar las funcionalidades de
% aplicativos Android. En el trabajo de investigación se dan los primeros pasos
% para el analisis de flujo de información de aplicativos Android mediante el
% sistema de anotaciones de Jif, y aunque quedan bastantes retos por superar, se
% hace un aporte importante al brindar una herramienta de análisis de flujo
% de información en Aplicativos Android mediante el sistema de anotaciones de
% Jif.

% apoyo al dessarrollador que le permite anotar y verificar políticas de
% confidencialidad en los aplicativos que implementa, mediante el sistema de anotaciones de Jif.
% 
% al posibilitar el
% análisis de flujo de información de aplicativos Android mediante el sistema de
% anotaciones de Jif, puesto que, se brinda una herramienta a través de la cual,
% el desarrollador define con anotaciones Jif la política de seguridad y verifica
% el cumplimiento de la misma.

% 
% de apoyo al desarrollador mediante la cual
% puede verificar el cumplimiento de determinadas políticas de seguridad, desde
% la construcción del aplicativo.


% Mientras una aplicación Java convencional cuenta con un único punto de entrada
% para iniciar su ejecución, esto es, la clase principal donde se implementa el
% método main; una aplicación Android puede tener más de un punto de entrada,
% generados a partir de los diferentes tipos de componentes que le
% integren(Activity, Service, Content Provider y Broadcast Receiver).
% Cuando se hace análisis de flujo de información a una aplicación Java mediante
% Jif, se tiene la certeza de que el punto de entrada a la aplicación es la clase
% que contenga el método Main, sin embargo, en una aplicación Android no se tiene
% dicha certeza porque la aplicación puede tener varios puntos de entrada. En
% consecuencia, cuando el aplicativo Android a analizar se compone de varias
% clases(correspondientes a diferentes tipos de componentes, actividades,
% servicios, Receivers) identificar el punto de entrada requiere modelar el cliclo
% de vida de la aplicación, sin embargo, mediante Jif no es posible obtener tal
% información, porque Jif hace un análisis plano del aplicativo.

