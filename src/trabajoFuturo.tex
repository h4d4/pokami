\label{ch:trabajoFuturo}
\chapter{Trabajo Futuro y Conclusiones}
\section{DiscusiónXXXXXXXXX}
Límites de la solución propuesta\newline

Las limitaciones de la solución propuesta se enmarcan en: políticas evaluables y
características del lenguaje Jif.\\
Por un lado, se propone un esquema de anotación con niveles de seguridad alto y
bajo, que permite definir y evaluar políticas de confidencialidad en aplicativos
Android mediante el sistema de anotaciones de Jif.
Sin embargo, el esquema de anotación propuesto no permite evaluar políticas de
integridad ni aplicar mecánismos Downgrading, características ofrecidas por el
sistema de anotaciones de Jif.

Por otro lado, están las limitaciones propias del lenguaje Jif, es decir, las
características para el lenguaje Java estandar que aún no están soportadas por
el compilador de Jif, y que por tanto, impiden analizar el flujo de información
de aplicativos Android que requieran de tales caracteríticas del lenguaje Java.
Más específicamente, se hace referencia a las limitaciones descritas en
\ref{subsec:limitaciones}: nested clases, initializer blocks, threads, etc.

\subsection{Analizar flujo de información de aplicaciones Android mediante Jif:}
- Si bien, con el prototipo de anotación se evita la anotación manual del
aplicativo Android, mediante el sistema de anotaciones de Jif, el trabajo
subyacente para generar tal anotación implica grandes retos, entre ellos:\newline
-la versión java del programa debe ser pensada e implementada mediante el
sistema de anotaciones de Jif, es decir, se debe aprender a implementar
aplicativos en jif, y para esto, la documentación existente es poca.\newline
- Para generar la versión Jif del programa, se deben anotar una serie de clases
de la Api de Android.\newline
- No todas las características del lenguaje Java son soportadas.

\subsection{Qué tanto cambia el código original con las anotaciones}
En el diseño de la solución se describieron los retos
técnicos\ref{sec:limitaciones} que implica anotar código Android, y cómo superar
algunos de estos. Específicamente, las limitaciones para las que se propone un
mecanismo que permita soportarlas son: Statement @Override; Casting entre tipos
EditText y View; Clase Nested R y Enhanced for loop.

% 
% Dentro de las limitaciones técnicas \ref{sec:limitaciones} descritas en el
% diseño de la solución, se describieron una serie de limitaciones para generar la
% versión Jif del programa Android a analizar. Para algunas de tales limitaciones
% se propuso un mécanismo que permita soportarlas, estas son:
% \begin{itemize}
%   \item Statement @Override
%   \item Casting entre tipos EditText y View
%   \item Clase Nested R 
%   \item Enhanced for loop
% \end{itemize}
Adicionalmente, se describió que el compilador de Jif exige la declaración del
chequeo de excepciones tipo runtime, en programas que lo
requieran\ref{subsec:consVerPol}. Ahora, cumplir con los requisitos del
compilador de Jif y aplicar mecanismos que soporten tales limitaciones, implica
una serie de transformaciones en el código fuente del programa Android a
analizar, tanto en la etapa previa a la anotación como en la anotación
misma.\newline
% para soportar tales limitaciones y
% cumplir con los requisitos del compilador de Jif, el código fuente del programa
% Android a analizar, pasa por una serie de transformaciones tanto en la etapa
% previa a la anotación como en la anotación misma.\newline
Previa a la anotación el desarrollador debe garantizar dos cosas, primero debe
adicionar las runtime exceptions que su programa necesite, segundo cuando
requiera el uso de bucles for, debe usar la versión sencilla y no la versión
mejorada del mismo(Enhanced for loop).\newline 
Durante la anotación, además de aplicar los criterios de anotación definidos en
el diseño de la solución \ref{subsec:pasosSol}, se aplican los mecanismos
propuestos para soportar el Statement @Override, Casting entre tipos EditText y
View; y Clase Nested R. Así, en el caso de  la sentencia Statement @Override, se
comenta la línea que lo contenga; para el casting entre tipos EditText y View,
la información implicada en este tipo de casting es abstraída mediante un tipo
de dato String; finalmente para la clase nested R se crea una clase que define
los define los recursos utilizados por la aplicación a través de
variables.\newline 
Si bien, la idea que fundamenta el diseño de la solución consiste en generar la
versión Jif del aplicativo Android a analizar, lo cual se traduce en adicionar
las anotaciones correspondientes para evaluar determinada política de seguridad,
de modo que el compilador de Jif entienda el programa y permita analizar el
flujo de información en el mismo; no es suficiente con sólo anotar el código, en
otras palabras, sin los ajustes previamente mencionados, el compilador genera
error. Por otro lado, lo positivo es que tales ajustes no alteran la lógica del
programa.


\section{Trabajo Futuro}
Cómo puede ser extendido el trabajo y qué beneficios tendría esa
extensión\newline

Exceptuando las clases de la API que requieran características del lenguaje Java
que no son soportadas por Jif, se podría ampliar el setup de Jif para Android,
de modo que se brinde soporte mediante el sistema de anotaciones de Jif a la
mayor cantidad de clases posibles de la API. Esto permitiría hacer análisis de
flujo de información a una mayor cantidad de aplicativos Android.\newline
El esquema de anotación podría ser extendido para definir y evaluar políticas de
integridad y mecanismos de downgrading, de manera tal que el desarrolador tenga
un mayor rango de opciones para verificar si el aplicactivo que
implementa, cumple con políticas de seguriad.\newline

Tomado de la propuesta de solución:\newline
Por último, cabe anotar que aunque la presente propuesta está centrada en
verificar políticas de confidencialidad, en caso de contar con el tiempo
prudente, sería interesante analizar políticas adicionales como por ejemplo,
integridad y declasificación, pues estas son verificables mediante el modelo de
evaluación de JIF, modelo del que parte la herramienta de evaluación planteada.

\section{ConclusionesXXXXXXXXX}
Qué aprendimos con este trabajo.\newline

Buscando contribuir en la labor del desarrollador de aplicaciones Android, de
modo que este pueda garantizar el cumplimiento de determinadas políticas de
seguridad, en el presente trabajo se exploró el análisis de flujo de información
en aplicaciones Android, mediante técnicas de lenguajes tipados de seguridad,
específicamente mediante el sistema de anotaciones de Jif.\newline 
Acorde a la literatura científica consultada, el análisis de aplicaciones
Android mediante el sistema de anotaciones de Jif, es una opción no antes
explorada.\newline
Habiendo superado una serie de limitaciones técnicas, se implementó un
prototipo de anotación, y se comparó con otras herramientas de
análisis(FlowDroid y JoDroid).
Partiendo de los tipos de análisis y técnicas evaluadas, de sus ventajas y
desventajas, soportadas no solamente en los resultados de evaluación sino
también en la literatura científica, es posible decir que 



