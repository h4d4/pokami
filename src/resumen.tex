\label{ch:resumen}
\chapter*{
\begin{center}
	Resumen
\end{center} }

El presente trabajo de investigación plantea aplicar técnicas de análisis
basadas en control de flujo de información, con el fin de verificar la ausencia
de fugas de información en aplicaciones Android, desde su construcción. 
Puesto que, el manejo de información del usuario es una de las principales
preocupaciones de seguridad en aplicativos Android.\newline
Precisamente, un estudio reciente de seguridad en dispositivos móviles,
publicado por McAfee\cite{McAfeeReport}, señala que entre las aplicaciones que
invaden la privacidad del usuario, existen aplicaciones(no consideradas malware)
que comprometen información privada o sensible del usuario, puesto que, permiten
que información como: su ubicación, acciones en el dispositivo y sus contactos
telefónicos, sea enviada hacia servidores de compañías
publicitarias\cite{McAfeeReport}(Pag 10).\newline 
A lo anterior, se suma que en Android por defecto, el desarrollador no cuenta
con mecanismos para definir políticas de seguridad que
regulen el flujo de información de sus aplicaciones, careciendo de herramientas
que le garanticen la ausencia de flujos indeseados. Así, para el
desarrollador es complejo prevenir fugas de información del usuario. \newline 
Diferentes trabajos de investigación han abordado el problema de pérdida de
información en aplicativos Android, sin embargo, literatura científica existente
al respecto señala que: la mayoría de propuestas aplican técnicas data-flow
análisis a partir del bytecode(TaintDroid\cite{TaintDroid},
Flow-Droid\cite{FlowDroid-Thesis}, DidFail\cite{DidFail}, DroidForce\cite{DroidForce}).
Tales propuestas se enfocan en la precisión y eficiencia del
análisis para detectar fugas de datos en aplicaciones de terceros ya
implementadas. Por consiguiente, la finalidad del análisis no es garantizar el
cumplimiento de políticas de confidencialidad e integridad desde la construcción
del aplicativo.\newline 
Para contribuir en la solución de dicha problemática, en el presente trabajo de
investigación se plantea aplicar técnicas de análisis basadas en control de
flujo de información, con el fin de garantizar el cumplimiento de determinadas
políticas de seguridad desde su construcción.
Más específicamente, se propone una herramienta para análisis estático de flujo
de información, basada en el sistema de anotaciones de Jif.
De manera que, partiendo de las anotaciones de seguridad definidas por el
desarrollador en el código de su aplicación, se verifique si esta cumple 
con determinada política de seguridad.\newline
A su vez, la herramienta serviría como medio de verificación para el usuario, ya
que mediante la herramienta podría redefinir y verificar las políticas de
seguridad en el aplicativo que adquiere(siempre y cuando sea código de
abierto).\newline
El presente trabajo parte del diseño de solución ideal que se plantea
\ref{sec:sol-desig}, centrándose en un conjunto reducido de clases de la API
Android y un conjunto específico de sources y sinks, acorde a una política de
seguridad establecida.\newline 
De este modo, el desarrollador define la política de seguridad en su
aplicativo mediante el sistema de anotaciones, y verifica el cumplimiento de la
misma con el compilador de Jif.\newline  
No obstante, para efectos de evaluación, la anotación del desarrollador es
automatizada mediante un generador de anotaciones, que anota el aplicativo
acorde a la política de seguridad a evaluar.\newline
Adicionalmente para la evaluación, se parte del benchmark de aplicativos Android
DroidBench \cite{DroidBenchBenchmarks}, se define un conjunto de aplicaciones
evaluables, y se analizan con la herramienta de análisis propuesta, los
resultados obtenidos son comparados con FlowDroid \ref{sec:FlowDroid-Tool} y
JoDroid \ref{sec:jod}, dos herramientas de análisis estático basadas en
flujo de datos y flujo de información, respectivamente.

Los resultados de evaluación para la solución propuesta coinciden con las
hipótesis iniciales, puesto que, al estar basada en control de flujo de
información, el análisis es más rápido, menos preciso pero a la vez completo. Es
decir, se reportan más falsos positivos pero a la vez, se pierden menos fugas de
información.

Finalmente, además de identificar ventajas y desventajas en la solución
propuesta, se identifican una serie de retos, generados en gran medida por las
diferencias existentes entre una aplicación Android y una aplicación Java
convencional; y por las limitaciones propias del lenguaje de anotación Jif.

























