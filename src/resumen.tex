\label{ch:resumen}
\chapter*{
\begin{center}
	Resumen
\end{center} }
Breve resumen del trabajo : contexo, problema, solución propuesta, resultados
alcanzados.

El presente trabajo de investigación plantea aplicar técnicas de análisis
basadas en control de flujo de información, con el fin de verificar la ausencia
de fugas de información en aplicaciones Android, desde su construcción. Puesto
que, controlar el acceso y uso de la información, representa una de las
principales preocupaciones de seguridad en dichos aplicativos.\newline 
Un estudio reciente de seguridad en dispositivos móviles, publicado por
McAfee\cite{McAfeeReport}, revela que en el contexto de aplicativos Android:
80\% reúnen información de la ubicación, 82\% hacen seguimiento de alguna acción
en el dispositivo, 57\% registran la forma de uso del celular(mediante Wi-Fi o
mediante la red de telefonía), y 36\% conocen información de las cuentas de
usuario.\newline
Diferentes trabajos de investigación han abordado el problema de pérdida de
información en aplicativos Android, sin embargo, la literatura científica
existente al respecto, señala que la mayoría de trabajos aplican técnicas para
hacer data-flow análisis a partir del bytecode. De modo que, su finalidad es
detectar fugas de información en aplicativos ya implementados, y no,
en garantizar el cumplimiento de determinadas políticas de seguridad desde la
construcción del aplicativo. Así, el desarrollador de la aplicación carece de
herramientas de apoyo para verificar si la aplicación que implementa, cumple con
determinadas políticas de seguridad.\newline

Para contribuir en la verificación de propiedades de seguridad desde la
implementación de aplicativos Android, se propone una herramienta para análisis
estático de flujo de información basada en el sistema de anotaciones de Jif.
Del diseño de solución ideal que se plantea, el presente trabajo se centra en
anotar un conjunto reducido de clases de la API Android mediante anotaciones de
Jif, y en partir de un conjunto reducido de sources y sinks, acorde a una
política de seguridad previamente establecida.\newline
De este modo, el desarrollador define la propiedad de sedugirad en su
aplicativo, mediante el sistema de anotaciones y compila la respectiva versión
Jif para verificar el cumplimiento de la misma.\newline
No obstante, para efectos de evalución, la anotación del desarrollador es
automatizada mediante un generador de anotaciones, que anota la política de
seguridad a evaluar en el aplicativo.\\ 
La evaluación se centra en un conjunto de aplicaciones, estas son analizadas
mediante la herramienta de análisis porpuesta(Prototipo) y los resultados del
análisis son comparados frente FlowDroid y JoDroid, dos herramientas de
análisis estático basadas en flujo de datos y flujo de información
respectivamente.\newline
Las políticas de seguridad evaluables, se centran en políticas de
confidencialidad.\Newline

Los resultados de evaluación para la solución propuesta coindiden con lo
esperado, pues al estar basada en control de flujo de información, el anális es
más rápido, menos preciso pero a la vez completo. Es decir, genera más falsos
positivos pero reporta las fugas existentes.

























