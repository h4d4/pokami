\label{ch:resumen}
\chapter*{
\begin{center}
	Resumen
\end{center} }
Breve resumen del trabajo : contexo, problema, solución propuesta, resultados
alcanzados.

El presente trabajo de investigación plantea aplicar técnicas de análisis
basadas en control de flujo de información, con el fin de verificar la ausencia
de fugas de información en aplicaciones Android, desde su construcción. Puesto
que, controlar el acceso y uso de la información, representa una de las
principales preocupaciones de seguridad en dichos aplicativos.\newline 
Un estudio reciente de seguridad en dispositivos móviles, publicado por
McAfee\cite{McAfeeReport}, revela que en el contexto de aplicativos Android:
80\% reúnen información de la ubicación, 82\% hacen seguimiento de alguna acción
en el dispositivo, 57\% registran la forma de uso del celular(mediante Wi-Fi o
mediante la red de telefonía), y 36\% conocen información de las cuentas de
usuario.\newline
Diferentes trabajos de investigación han abordado el problema de pérdida de
información en aplicativos Android, sin embargo, literatura científica existente
al respecto señala que la mayoría de propuestas aplican técnicas data-flow
análisis a partir del bytecode. Enfocandose en detección de fugas de información
en aplicativos ya implementados, y no, en garantizar el cumplimiento de
determinadas políticas de seguridad desde la construcción del aplicativo. Por
lo tanto, el desarrollador de la aplicación carece de herramientas de apoyo para
verificar si la aplicación que implementa, cumple con determinadas políticas de
seguridad.\newline

Para contribuir en la verificación de políticas de seguridad desde la
implementación de aplicativos Android, se propone una herramienta para análisis
estático de flujo de información basada en el sistema de anotaciones de Jif. De
manera que, partiendo de las anotaciones de seguridad definidas por el
desarrollador en el código de su aplicación, se verifique si esta cumple 
con determinada política de seguridad.

El presente trabajo parte del diseño de solución ideal que se plantea
\ref{sec:sol-desig}, centrandose en un conjunto reducido de clases de la API
Android y un conjunto específico de sources y sinks, acorde a una política de
seguridad establecida.\newline 
De este modo, el desarrollador define la propiedad de segurirad en su
aplicativo, mediante el sistema de anotaciones y compila la respectiva versión
Jif para verificar el cumplimiento de la misma.\newline 
No obstante, para efectos de evalución, la anotación del desarrollador es
automatizada mediante un generador de anotaciones, que anota el aplicativo
acorde a la política de seguridad a evaluar.\\
Para la evaluación se especifica un conjunto de aplicaciones, estás son
analizadas con la la herramienta de análisis propuesta, y los resultados
obtenidos son comparados con FlowDroid y JoDroid, dos herramientas de análisis
estático basadas en flujo de datos y flujo de información, respectivamente.\newline

Los resultados de evaluación para la solución propuesta coindiden con las
hipótesis iniciales, puesto que, al estar basada en control de flujo de
información, el análisis es más rápido, menos preciso pero a la vez completo. Es
decir, reporta más falsos positivos pero identifica todas o la mayoría de fugas.

























