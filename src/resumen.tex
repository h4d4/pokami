\label{ch:resumen}
\chapter*{
\begin{center}
	Resumen
\end{center} }

% Breve resumen del trabajo: contexo, problema, solución propuesta, resultados
% alcanzados.
El presente trabajo de investigación plantea aplicar técnicas de análisis
basadas en control de flujo de información, con el fin de verificar la ausencia
de fugas de información en aplicaciones Android, desde su construcción. 
Puesto que, controlar el acceso y uso de la información, representa una de las
principales preocupaciones de seguridad en dichos aplicativos.\newline 
Un estudio reciente de seguridad en dispositivos móviles, publicado por
McAfee\cite{McAfeeReport}, revela que en el contexto de aplicativos Android:
80\% reúnen información de la ubicación, 82\% hacen seguimiento de alguna acción
en el dispositivo, 57\% registran la forma de uso del celular(mediante Wi-Fi o
mediante la red de telefonía), y 36\% conocen información de las cuentas de
usuario.\newline
Adicionalmente, el informe señala que una aplicación invasiva no necesariamente
contiene malware, y que su finalidad no siempre implica fraude; de las
aplicaciones que más vulneran la privacidad del usuario, 35\% contienen
malware.\newline 
Si bien, aplicaciones invasivas no necesariamente implican malware y/o acciones
delictivas, el cuestionamiento de fondo es la forma y finalidad con que una
aplicación manipula la información del usuario, y qué garantías puede ofrecer el
desarrollador para que tal manipulación sea consentida.

Diferentes trabajos de investigación han abordado el problema de pérdida de
información en aplicativos Android, sin embargo, literatura científica existente
al respecto señala que: la mayoría de propuestas aplican técnicas data-flow
análisis a partir del bytecode. Enfocándose en detección de fugas de información
en aplicativos ya implementados, y no, en garantizar el cumplimiento de
determinadas políticas de seguridad desde la construcción del aplicativo. Por
lo tanto, el desarrollador de la aplicación carece de herramientas de apoyo para
verificar si la aplicación que implementa, cumple con determinadas políticas de
seguridad.

Para contribuir en la verificación de políticas de seguridad desde la
implementación de aplicativos Android, se propone una herramienta para análisis
estático de flujo de información, basada en el sistema de anotaciones de Jif.
De manera que, partiendo de las anotaciones de seguridad definidas por el
desarrollador en el código de su aplicación, se verifique si esta cumple 
con determinada política de seguridad.\newline
El presente trabajo parte del diseño de solución ideal que se plantea
\ref{sec:sol-desig}, centrándose en un conjunto reducido de clases de la API
Android y un conjunto específico de sources y sinks, acorde a una política de
seguridad establecida.\newline 
De este modo, el desarrollador define la propiedad de seguridad en su
aplicativo mediante el sistema de anotaciones, y verifica el cumplimiento de la
misma con el compilador de Jif.\newline 
No obstante, para efectos de evaluación, la anotación del desarrollador es
automatizada mediante un generador de anotaciones, que anota el aplicativo
acorde a la política de seguridad a evaluar.\\
Adicionalmente para la evaluación, se especifica un conjunto de aplicaciones que
son analizadas con la la herramienta de análisis propuesta, los resultados
obtenidos son comparados con FlowDroid y JoDroid, dos herramientas de análisis
estático basadas en flujo de datos y flujo de información, respectivamente.

Los resultados de evaluación para la solución propuesta coinciden con las
hipótesis iniciales, puesto que, al estar basada en control de flujo de
información, el análisis es más rápido, menos preciso pero a la vez completo. Es
decir, se reportan más falsos positivos pero se identifican todas las fugas de
información.

























